\section{Union-Find decoder}\label{sec:unionfind}

The Union-Find decoder \cite{delfosse2017linear, delfosse2017almost} maps each element of the syndrome $\sigma$ to a so-called non-trivial vertex $v$ in a non-connected graph on the code lattice, and grows clusters $c$ that form a connected graph $c(\m{V}, \m{E})$ of vertices $\m{V}$ and edges $\m{E}$ locally, by iteratively adding a layer of edges and trivial vertices to existing clusters, until all clusters have an even number of non-trivial syndrome vertices. This process is described as the growth of a cluster or the growth of the vertices that lie on a cluster's boundary. Then, a spanning tree is built for each cluster, and leafs are conditionally peeled from the tree in a tail recursive breadth-first search until all non-trivial syndrome vertices are paired and linked by a path, which is the correcting operator $\mathcal{C}$. By growing the clusters of vertices in order of their sizes - the number of vertices in the cluster - the threshold is reported to increase from $9.2\%$ to $9.9\%$ in the \emph{weighted growth} variant of the decoder.

The complexity of the Union-Find decoder is driven by the merging between clusters. For this the algorithm uses the Union-Find or disjoint-set data structure \cite{tarjan1975efficiency}. The function \codefunc{Find} is used to travel in the vertex-tree of the cluster to the distinctive root element to find the parent cluster of a given vertex. If a newly added vertex has a different root, the vertex-trees are merged by \codefunc{Union}. 