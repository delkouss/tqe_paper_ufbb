\section{Union-Find decoder}\label{sec:unionfind}

The Union-Find decoder \cite{delfosse2017linear, delfosse2017almost} maps each syndrome $\sigma_i$ to a so-called non-trivial vertex $v_i$ in a non-connected graph on the code lattice, and grows clusters $c_j$ of vertices that form a connected graph locally, by adding iteratively a layer of edges and trivial vertices to existing clusters until all clusters have an even number of non-trivial syndrome vertices. Then, a spanning tree is built for each cluster, and every tree is traversed until all non-trivial syndrome vertices are paired and linked by a path, which is the correcting operator $\mathcal{C}$. By growing the clusters of vertices in order of their sizes, which is dubbed the \emph{weighted growth} version of the Union-Find decoder, the threshold is reported to increase from $9.2\%$ to $9.9\%$ compared to the non-weighted variant for the toric code. 

The complexity of the Union-Find decoder is driven by the merging between clusters. For this the algorithm uses the Union-Find or disjoint-set data structure \cite{tarjan1975efficiency}. The function \codefunc{Find} is used to traverse the vertex-tree of the cluster to the distinctive root element to find the parent cluster of a given vertex. If a newly added vertex has a different root, the vertex trees are merged by \codefunc{Union}. 