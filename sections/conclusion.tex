\section{Conclusion}\label{sec:conclusion}

In this thesis, we have thoroughly inspected the Union-Find decoder that has the advantage in running time and simplicity compared to other decoders. We have provided various implementations of the original Union-Find decoder with the bucket sort method for weighed growth and the dynamic forest variant of the decoder. We have introduced a modification of the original decoder that selectively grows regions of clusters based on the concept of a potential matching weight. The modified decoder, dubbed the Union-Find Balanced-Bloom (UFBB) decoder, relies on an additional node-tree data structure to facilitate the calculation of the potential matching weight. A direct comparison of the decoding performance of our decoder, the Union-Find decoder (with original data \cite{delfosse2017almost}), and the Minimum-Weight Perfect Decoder in Figure \ref{fig:directcomp}. We summarize our results in short:

\begin{itemize}
  % \item By maintaining a dynamic forest, the threshold of the Union-Find decoder can be slightly increased. 
  \item The UFBB decoder has an improved decoder threshold compared with the Union-Find decoder for all lattice sizes and a comparable threshold to the Minimum-Weight Perfect Matching decoder for small lattices.
  \item The UFBB decoder has an improved decoding rate at the threshold error rate compared with the Union-Find decoder for all lattice sizes. 
  % \item A degradation of the error threshold of the UFBB decoder on larger lattice sizes is expected due to the effects of parity inversion and increased node delays. 
  % \item The matching weight of the UFBB decoder is decreased from the matching weight of the Dynamic-forest Bucket Union-Find decoder by constant factor, depending on the error model. 
  % \item For a constant value of the equilibrium factor, $k_{eq}=\frac{1}{2}$ results in the best performance in the UFBB decoder. 
  \item The UFBB decoder has a worst-case time complexity of $\m{O}(N\log{N})$. 
\end{itemize}

Many of the work on recent decoders have focused on increasing the code threshold via, for example, the use of neural networks. However, these decoders have the disadvantage that they have a significant running time and bad scalability. The Union-Find decoder manages to decode fast and scale almost-linearly with the input system size. For this reason it may be a great candidate for physical applications in the near future. We manage to find a middle ground between the two objectives as the Union-Find Balanced-Bloom decoder has an improved decoding performance at the cost of a slight increase in complexity. 

% Recent work that includes the Union-Find decoder focuses on bringing the decoder algorithm to the hardware level. Most notably, a scalable decoder micro-architecture has been proposed with a fully pipelined hardware implementation \cite{das2020scalable}. Related work has shown that a reduction in bandwidth is possible provided qubits with a low physical error rate \cite{delfosse2020hierarchical}. Furthermore, another variant of the decoder dubbed the \emph{Weighted Union-Find} decoder, not to be confused with \emph{weighted growth}, promises to increase the code threshold under circuit-level noise \cite{huang2020fault}. This application relies on adopting the decoder to a \emph{weighted} graph. Every edge $e\in\m{E}$ may now have a different length value, and edges are not limited to the growth of half-edges per growth iteration. We believe that Union-Find Balanced-Bloom decoder and the Weighted Union-Find decoder are compatible. In the combined decoder, boundary edges in every node are grown with respect to their weights in the weighted graph. For these reasons, the Union-Find decoder has a bright future in the world of quantum algorithms.
