
\section{Complexity of Balanced-Bloom}\label{sec:complexity}

\Figure[htb](topskip=0pt, botskip=0pt, midskip=0pt){tikzfigs/tikz-figure6.pdf}{bla.).\label{fig6}}

In this section, we will find the worst-cast time complexity of the Union-Find Node-Suspension decoder. The addition cost of the original Union-Find decoder can be split in two parts: (A) the depth-first-searches (DFS's) related to the (re)calculation of the node parities and node delays in line \ref{algo:pdc}, and (B) the DFS related to the growth of a cluster in line \ref{algo:grow}. We dub these two parts the \textbf{suspension cost} and the \textbf{growth cost}, respectively. The \codefunc{Join} operation in lines \ref{algo:joina}-\ref{algo:joinb} only has a linear addition to the cost.

\subsection{Suspension cost}\label{sec:suscomplexity}

The cost of node suspension calculation is equal to the number of nodes traversed in the DFS's of the node parities and node delays. To find this number $N_{sus}(N)$ analytically, we utilize the rules of cluster growth and mergers. 
\begin{enumerate}
  \item Weight growth states that clusters are grown in the order of sizes of their vertex-trees. 
  \item Joins between the node-trees between odd and even clusters always retains the root node of the odd cluster.
  \item The DFS's of the node parity and delay calculation are always performed just before cluster growth in an even partition of the node-tree starting from the pointer saved at the root node .
\end{enumerate}
From rules 2 and 3, we conclude that $N_{sus}$ is proportional to the sum of sizes of all even node-trees during all mergers of the growth process on the lattice. Note that if many clusters merge within the same growth iteration, only the last even cluster counts towards the cost, since rule 3 ensures that the calculation is not performed on intermediate even partitions. To find the worst-case time complexity, we maximize $N_{sus}$, which is proportional to the computation time. We take a top-down approach of \textbf{cluster fragmentation}; starting from a single cluster that maximally occupies the lattice at the end of growth, and move back in time to find its ancestor clusters and their sizes. The maximization of $N_{sus}$ is in the repetitiveness of the recalculation over some parition of the final node-tree. 

\begin{definition}\label{def:fragmentation}
  Let the \emph{fragmentation} of an odd cluster with node-tree $\pre{k-1}\nset^o$ split it into a set of its ancestral node-trees. Here $k$ indicates the \emph{generation}, where larger $k$ indicates a more distant ancestor set of smaller node-trees. Let the fragmentation step $f$ be the combination of \emph{partial fragmentations (PF)} $f_o$, which fragments an odd node-tree into an even ancestor and an odd ancestor
  \begin{equation}\label{eq:pfe}
    f_o(\{\pre{k-1}\nset^o\}) = \m{F}^o_k = \{\pre{k}\nset^e_{-1}, \pre{k}\nset^o_0 \}, 
  \end{equation}
  and PF $f_e$ fragments even node-trees into $n_f$ odd ancestors
  \begin{equation}\label{eq:pfo}
    f_e(\{\pre{k}\nset^e_{-1}\}) = \m{F}^e_k=\{\pre{k}\nset^{o}_1,...,\pre{k}\nset^o_{n_f}\},
  \end{equation}
  such that a fragmentation step is
  \begin{equation}\label{eq:fstep}
    f(\{\pre{k-1}\nset^o\}) = \m{F}_k = \{\pre{k}\nset^o_0,\pre{k}\nset^{o}_1,...,\pre{k}\nset^{o}_{n_f}\}.
  \end{equation}
\end{definition}

A fragmentation step is only possible on an odd node-tree $\nset^o$ with $|{\nset^o}| \geq 3$. The fragmentation functions does not remove single-node node-trees from a set when applied, such that for an odd cluster of finite size, after $k_m$ fragmentation steps all odd ancestors in $\m{F}_{k_{m}}$ are single nodes. Using fragmentations, we can sum over the fragmentation generations to find the sum of even node-tree sizes
\begin{equation}\label{eq:npdc}
  N_{sus} = 2\sum_{k=1}^{k_{m}}{ \sum_j{ \left\{ \abs{\pre{k}\nset_j^e} \big| \pre{k}\nset_j^e \in \m{F}^o_k(\pre{0}\nset^o)\right\} } },
\end{equation}
for which we require (a) the number of generations $k_m$, (b) the number of odd ancestors $n_f$ in Equation \eqref{eq:pfo} and (c) the ratio between the node-tree sizes of an odd node-tree and its even ancestor. To find these values, we make two assumptions to simplify the sum: 
\begin{enumerate}
  \item Assume that all odd ancestor at generation $k_m$ are of a single node. 
  \item Assume that cluster does not increase in size (vertex-tree) when grown, such that only mergers occur between existing clusters. 
\end{enumerate}
As a result of assumption 2, nodes are effectively not allowed to increase in radius. While this is only conceptually impossible, it maximizes the number of nodes on the lattice such that the sum in Equation \eqref{eq:pfo} forms an upper bound to $N_{sus}$. 

\begin{theorem}\label{the:fragnumber}
  Fragmentation number $n_f=2$ maximizes $N_{sus}$ in equation \eqref{eq:npdc}.
\end{theorem}
\begin{proof}
  The sum of even node-tree sizes in each fragmentation step is constant per Lemma \ref{lem:equalevensum}. Thus, \eqref{eq:npdc} is maximized by having as many fragmentations steps as possible, or the largest possible $p$.  As $k_f$ increases the number of odd node-trees in each fragmentation step $f_o$, the average size of these odd node-trees has decreased. Consequently, the node-tree size decreases faster towards the minimum size of three nodes as more fragmentation steps are applied (Equation \eqref{eq:fstep}). As the sum of even node-tree sizes in each fragmentation step is the same, increasing $k_f$ decreases the number of fragmentation steps:
  \begin{equation}
    p \propto \frac{1}{k_f}.
  \end{equation}
  Hence, $N_{PDC}$ is maximized for the minimal value of $k_f$, which is $k_f = 2$.
\end{proof}


% \input{tikzfigs/fragmentation}

If partial fragmentation function $f_o$ is called on a set of node-trees $f_o(\{\nset^o, \nset^e, ...\})$, it fragments all odd node-trees in the set. Partial fragmentation function $f_e$ fragments all even node-trees. Along these lines, the entire set of odd node-trees $\m{F}_k$ can undergo the another fragmentation step into odd subsets, resulting in another set of ancestral node-trees $\m{F}_{k+1}$. We can do this some $p$ times on $\pre{0}\nset^o$, where we have selected $k-1=0$, until our resulting set of node-trees $\m{F}_{p}$ consists only of the smallest possible node subsets $\pre{p}\nset^o$ where $|\pre{p}\nset^o|=1$.


To find the worst-case complexity, we maximize $N_{PDC}$ or the cost of the partial calculations during the construction of the node-trees on the lattice. Let us assume the worst-case, when there are a maximal number of nodes in the node-trees just before the last round of growth. As the lattice is maximally occupied, this is a single odd node-tree $\pre{0}\nset^o$ in which a partial calculation is performed as part of the last round of growth. Node-tree $\pre{0}\nset^o$ has a maximal number of nodes if $|n.\vset|=1$ for all nodes $n$ in $\pre{0}\nset^o$. Thus, on a lattice of $N=|\vset|$ vertices, the node-tree $\pre{0}\nset^o$ has a maximal
\begin{equation}\label{eq:limitnsetsize}
  \abs{\pre{0}\nset^o} \leq N
\end{equation}
nodes. As the partial calculation is only executed on the even subtrees, $N_{PDC}$ is the sum of even node-trees sizes $|\pre{k}\nset^e|$, in all partial fragmentation sets $\m{F}^o_{k}$, during all fragmentation steps $k=[1,...,p]$, in the full fragmentation of $F(\pre{0}\nset^o)$. We add the factor 2 in Equation \eqref{eq:npdc} as both the parity calculation and delay calculations requires its own depth-first search. The sequence of fragmentations that maximizes the even node-tree sizes maximizes $N_{PDC}$.
% The worst-case delay complexity is computed by maximizing $N_{PDC}$ of the full fragmentation of $\pre{0}\nset^o$ with $S_{\pre{0}\nset^o} = N/2-1$.
\begin{definition}\label{def:fullfrag}
  Let the series of all $p$ fragmentation steps $f$ on $\pre{0}\nset^o$ be the \emph{full fragmentation} $F$, with
  \begin{equation}\label{eq:fullfrag}
    F(\pre{0}\nset^o) = \underbrace{f(f(...f(\pre{0}\nset^o)))}_\text{p times} = \{\pre{p}\nset^{o}_1, \pre{p}\nset^{o}_2,...,\pre{p}\nset^{o}_{N_\sigma} \} \hspace{.3cm} | \hspace{.3cm} \abs{\pre{p}\nset^{o}_i} = 1.
  \end{equation}
\end{definition}

\begin{definition}\label{lem:fragratio}
  Let the partial fragmentation ratio $R$ be the relative sizes of an ancestral node-tree $\pre{k+1}\nset$ and the fragmented node-tree $\pre{k}\nset$.
  \begin{equation}\label{eq:fragratio}
    zfragratio = \frac{\abs{\pre{k+1}\nset}}{\abs{\pre{k}\nset}}
  \end{equation}
\end{definition}
In $f_e$ there are a set of partial fragmentation ratios $\{R_{-1}, R_0\}$, and in $f_o$ are a set of partial fragmentation ratios $\{R_1,...,R_{k_f}\}$, where
\begin{align}
  R_{-1} +  R_0         & = 1  \\
  \sum_{i=1}^{k_f}{R_i} & = 1.
\end{align}

The problem of finding the sequence of even ancestral node-tree sizes to maximize the value of $N_{PDC}$ now becomes finding the partial fragmentation number $k_f$ and the set of partial fragmentation ratios $\{R_{-1},..., R_{k_f}\}$.

\begin{lemma}\label{lem:sumevenkf}
  For the same partial fragmentation ratios $\{R_{-1}, R_0\}$ in $f_o$, the sum of even ancestral node-tree sizes after a fragmentation step is not dependent on $k_f$ (see Figure \ref{fig:fragcorrect}).
\end{lemma}
\begin{proof}
  Let us consider an even node-tree $\pre{k}\nset^e$ that is first partially fragmented by $f_e$ to $\m{F}^e_k$. The fragmentation set $\m{F}^e_k$ is then partially fragmented by $f_o$ to $\m{F}_{k+1}^o$. Let us consider the two cases when $k_f=2$ and $k_f=4$. For $k_f=2$, the partial fragmentation $f_e$ splits $\pre{k}\nset^e$ into two odd ancestral node-trees in $\m{F}^e_k$ and four node-trees in $\m{F}_{k+1}^o$.
  % Let the size of $\pre{k-1}\nset^e$ be $|\pre{k-1}\nset^e| = K$. To find $n_o$, let us consider two cases where $n_o = 1$ or $n_o=2$. If an even node-tree $\pre{k-1}\nset^e$ is fragmented with $k_f=2$, a fragmentation step $f(\pre{k-1}\nset^e)=f_o(f_e(\pre{k-1}\nset^e))$ produces the following partial fragmentation sets:
  \begin{align*}
    % \nonumber % Remove numbering (before each equation)
    f_e(\pre{k}\nset^e)_{k_f = 2}
    = \m{F}^e_{k}|_{k_f = 2}
     & = \{ \pre{k} \nset^{o}_1, \pre{k} \nset^{o}_2\}                                                                             \\
    f_o(\m{F}^e_{k}|_{k_f = 2})
    = \m{F}^o_{k+1}|_{k_f = 2}
     & = \left\{ \{\pre{k+1}\nset^{o}_{0}, \pre{k+1}\nset^{e}\}^o_1 , \{\pre{k+1}\nset^{o}_{0}, \pre{k+1}\nset^{e} \}^o_2 \right\}
  \end{align*}
  The ratios of the sizes of fragmented node-trees in $f_e$ are
  \begin{equation*}
    \frac{\abs{\pre{k} \nset^{o}_1}}{\abs{\pre{k}\nset^e}} = R_1, \hspace{2em}
    \frac{\abs{\pre{k} \nset^{o}_2}}{\abs{\pre{k}\nset^e}} = R_2,
  \end{equation*}
  where $ R_1 + R_2 = 1$. The ratios of the sizes of fragmented node-trees in $f_o$ are
  \begin{equation*}
    \frac{\abs{\pre{k+1}\nset^{o}_0|^o_1}}{\abs{\pre{k} \nset^{o}_1}} =
    \frac{\abs{\pre{k+1}\nset^{o}_0|^o_2}}{\abs{\pre{k} \nset^{o}_2}} = R_0, \hspace{2em}
    \frac{\abs{\pre{k+1}\nset^{e}  |^o_1}}{\abs{\pre{k} \nset^{o}_1}} =
    \frac{\abs{\pre{k+1}\nset^{e}  |^o_2}}{\abs{\pre{k} \nset^{o}_2}} = R_{-1},
  \end{equation*}
  where $R_0 + R_{-1} = 1$. The sum of the sizes of even node-trees in the odd partial fragmentation set $\m{F}^o_{k+1}$ is thus
  \begin{equation*}
    R_1 R_{-1} \abs{\pre{k}\nset^e} + R_2 R_{-1} \abs{\pre{k}\nset^e} = (R_1 + R_2) R_{-1} \abs{\pre{k}\nset^e} = R_{-1} \abs{\pre{k}\nset^e}
  \end{equation*}

  For $k_f = 4$, the partial fragmentation sets are
  \begin{align*}
    % \nonumber % Remove numbering (before each equation)
    f_e(\pre{k}\nset^e)_{k_f = 4}
    = \m{F}^e_{k}|_{k_f = 4}
     & =\{ \pre{k}\nset^{o}_1, \pre{k}\nset^{o}_2,  \pre{k}\nset^{o}_3,\pre{k}\nset^{o}_4\}, \\
    f_o(\m{F}^e_{k}|_{k_f = 4})
    = \m{F}^o_{k+1} |_{k_f = 4}
     & = \big\{      \{ \pre{k+1}\nset^{o}_0, \pre{k+1}\nset^e\}^o_1,
    \{ \pre{k+1}\nset^{o}_0, \pre{k+1}\nset^e\}^o_2,                                         \\
     & \hspace{3em} \{ \pre{k+1}\nset^{o}_0, \pre{k+1}\nset^e\}^o_2,
    \{ \pre{k+1}\nset^{o}_0, \pre{k+1}\nset^e\}^o_4 \big\}.
  \end{align*}
  The ratios of the sizes of fragmented node-trees in $f_e$ are
  \begin{equation*}
    \frac{\abs{\pre{k} \nset^{o}_1}}{\abs{\pre{k}\nset^e}} = q_1, \hspace{2em}
    \frac{\abs{\pre{k} \nset^{o}_2}}{\abs{\pre{k}\nset^e}} = q_2, \hspace{2em}
    \frac{\abs{\pre{k} \nset^{o}_3}}{\abs{\pre{k}\nset^e}} = q_3, \hspace{2em}
    \frac{\abs{\pre{k} \nset^{o}_4}}{\abs{\pre{k}\nset^e}} = q_4,
  \end{equation*}
  where $ q_1 + q_2 + q_3 + q_4 = 1$. The ratios of the sizes of fragmented node-trees in $f_o$ are
  \begin{equation*}
    \frac{\abs{\pre{k+1}\nset^o_0|^o_1}}{\abs{\pre{k} \nset^{o}_1}} =
    \frac{\abs{\pre{k+1}\nset^o_0|^o_2}}{\abs{\pre{k} \nset^{o}_2}} =
    \frac{\abs{\pre{k+1}\nset^o_0|^o_3}}{\abs{\pre{k} \nset^{o}_3}} =
    \frac{\abs{\pre{k+1}\nset^o_0|^o_4}}{\abs{\pre{k} \nset^{o}_4}} = R_0,
  \end{equation*}
  and
  \begin{equation*}
    \frac{\abs{\pre{k+1}\nset^e|^o_1}}{\abs{\pre{k} \nset^{o}_1}} =
    \frac{\abs{\pre{k+1}\nset^e|^o_2}}{\abs{\pre{k} \nset^{o}_2}} =
    \frac{\abs{\pre{k+1}\nset^e|^o_3}}{\abs{\pre{k} \nset^{o}_3}} =
    \frac{\abs{\pre{k+1}\nset^e|^o_4}}{\abs{\pre{k} \nset^{o}_4}} = R_{-1},
  \end{equation*}
  where $R_0 + R_{-1} = 1$. The sum of the sizes of even node-trees in  the odd partial fragmentation set $\m{F}^o_{k+1}$ is thus
  \begin{align*}
    q_1 R_{-1} \abs{\pre{k}\nset^e} + q_2 R_{-1} \abs{\pre{k}\nset^e} + q_3 R_{-1} \abs{\pre{k}\nset^e} + q_4 R_{-1} \abs{\pre{k}\nset^e}
     & = (q_1 + q_2 + q_3 + q_4 ) R_{-1} \abs{\pre{k}\nset^e} \\
     & = R_{-1} \abs{\pre{k}\nset^e}.
  \end{align*}
  This is true for any $k_f = 2i, i\in \mathbb{N}^*$.
\end{proof}

\begin{lemma}\label{lem:equalevensum}
  The sum of even node-tree sizes in every fragmentation step $k$ is only dependent on partial fragmentation ratios $\{R_{-1}, R_0\}$.
  \begin{equation}\label{eq:equalevensum}
    \sum_j{ \left\{ \abs{\pre{k}\nset_j^e} \big| \pre{k}\nset_j^e \in \m{F}^o_k \right\} } = \text{constant}
    \hspace{1em} \bigg| \hspace{1em} \forall \m{F}_k^o \text{ during } F(\pre{0}\nset^o).
  \end{equation}
\end{lemma}
\begin{proof}
  Consider an odd node-tree $\pre{k-1}\nset^o$ that is partially fragmented as
  \begin{align*}
    f_o(\pre{k-1}\nset^o) = \m{F}^o_k     & = \{\pre{k}\nset^e_{-1}, \pre{k}\nset^o_0 \}                                                                                                                                 \\
    f_e(\m{F}^o_k)        = \m{F}^e_k     & = \left\{ \{\pre{k}\nset^o_i\ | i \in [1,..,k_f] \}^e_{-1}, \pre{k}\nset^{o}_0 \right\}                                                                                      \\
    f_o(\m{F}^e_k )       = \m{F}^o_{k+1} & = \left\{ \left\{ \{\pre{k+1}\nset^e_{-1}, \pre{k+1}\nset^o_0\}_i^o | i \in [1,..,k_f] \right\}^e_{-1}, \left\{\pre{k}\nset^e_{-1}, \pre{k}\nset^o_0 \right\}^{o}_0 \right\}
  \end{align*}

  The sum of even node-tree sizes in $\m{F}^o_k$ is simply the size of $\pre{k}\nset^e_{-1}$ and is equal to
  \begin{equation*}
    \sum_j{ \left\{ \abs{\pre{k}\nset_j^e} \big| \pre{k}\nset_j^e \in \m{F}^o_k \right\} } = \abs{\pre{k}\nset^e_{-1}} = R_{-1}\abs{\pre{k-1}\nset^o}.
  \end{equation*}

  The sum of even node-tree sizes in $\m{F}^o_{k+1}$ can be divided into two parts. The first part is the partial fragmentations $f_e f_o$ of $\pre{k}\nset^e_{-1}$, which we know from Lemma \ref{lem:sumevenkf} is $R_{-1}|\pre{k}\nset^e_{-1}|$ regardless of the choice for $k_f$. The second part is the partial fragmentation $f_o$ of $\pre{k}\nset^o_0$, which is $R_{-1}|\pre{k}\nset^o_0|$. Hence, the sum is
  \begin{equation*}
    \sum_j{ \left\{ \abs{\pre{k}\nset_j^e} \big| \pre{k}\nset_j^e \in \m{F}^o_{k+1} \right\} } = R_{-1} \left( \abs{\pre{k}\nset^e_{-1}} + \abs{\pre{k}\nset^o_0} \right) = R_{-1}\abs{\pre{k-1}\nset^o}.
  \end{equation*}
\end{proof}

\begin{theorem}\label{the:fragnumber}
  For the fragmentation number $k_f=2$, $N_{PDC}$ of Definition \ref{def:npdc} and Equation \eqref{eq:npdc} is maximized (see Figure \ref{fig:fragexamples}).
\end{theorem}
\begin{proof}
  The sum of even node-tree sizes in each fragmentation step is constant per Lemma \ref{lem:equalevensum}. Thus, \eqref{eq:npdc} is maximized by having as many fragmentations steps as possible, or the largest possible $p$.  As $k_f$ increases the number of odd node-trees in each fragmentation step $f_o$, the average size of these odd node-trees has decreased. Consequently, the node-tree size decreases faster towards the minimum size of three nodes as more fragmentation steps are applied (Equation \eqref{eq:fstep}). As the sum of even node-tree sizes in each fragmentation step is the same, increasing $k_f$ decreases the number of fragmentation steps:
  \begin{equation}
    p \propto \frac{1}{k_f}.
  \end{equation}
  Hence, $N_{PDC}$ is maximized for the minimal value of $k_f$, which is $k_f = 2$.
\end{proof}

% \input{tikzfigs/fragnumber}

The search for the fragmentation ratios has now been reduced to finding $\{R_{-1}, R_0\}$ of $f_o$ and $\{R_1, R_2\}$ of $f_e$ since $k_f = 2$. A partial fragmentation $f_e$ of $\pre{k}\nset^e_{-1}$ and fragmentation step $f$ of $\pre{k-1}\nset^o$ are now
\begin{align}
  f_e(\pre{k}\nset^e_{-1}) & = \m{F}^e_k  =\{\pre{k}\nset^{o}_1,\pre{k}\nset^o_2\} \label{eq:newpfe}                      \\
  f(\pre{k-1}\nset^o)      & = \m{F}_k    = \{\pre{k}\nset^o_0, \pre{k}\nset^o_1, \pre{k}\nset^o_1\}. \label{eq:newfstep}
\end{align}
The sizes of the ancestral odd node-trees in a fragmentation step $f$ are related to the joined node-tree by
\begin{equation}
  \frac{\abs{\pre{k}\nset^o_0}}{\abs{\pre{k-1}\nset^o}} = R_0, \hspace{2em}
  \frac{\abs{\pre{k}\nset^o_1}}{\abs{\pre{k-1}\nset^o}} = R_1, \hspace{2em}
  \frac{\abs{\pre{k}\nset^o_2}}{\abs{\pre{k-1}\nset^o}} = R_2,
\end{equation}
where
\begin{align}
  \nonumber  \tilde{R}_1                     & = R_{-1}R_1                      \\
  \tilde{R}_2                                & = R_{-1}R_2 \label{eq:bigratios} \\
  \nonumber  R_0 + \tilde{R}_1 + \tilde{R}_2 & = 1.
\end{align}

This fragmentation number does not come unexpectedly. If $k_f=2$, a fragmentation $f_e$ outputs two ancestral node-trees. This is equivalent to a single even-join. If $k_f>2$, the fragmentation $f_e$ will be equivalent to several even-joins and intermediate odd-joins. Recall from Lemma \ref{lem:delaywhengrown} that the partial calculation of every intermediate odd-join is skipped. Thus, these partial calculations are ``lost'' from the maximization of $N_{PDC}$.

Let us now try to maximize $N_{PDC}$ of Equation \eqref{eq:npdc}, not from the perspective of fragmentations, but from the perspective of cluster growth. During a growth iteration, some $N_v$ vertices are added to the cluster $c_j$, and some other clusters merge with $c_j$ that also require the join of their respective node-trees. If no join operations occur, the node-tree stays unchanged. The cluster is allowed to continue to grow without delay calculations per Lemma \ref{lem:calconce}. To maximize $N_{PDC}$, $N_v$ must be minimized, as every added vertex here is one that could have been part of a node in some other node-tree, and thus does not add to $N_{PDC}$.


\begin{lemma}\label{lem:localmax}
  Local maximization of the fragmentation ratio of the even subtree $R_{-1}=\tilde{R}_1+\tilde{R}_2$ leads to global maximization of $N_{PDC}$.
\end{lemma}
\begin{proof}
  Recall from Equation \eqref{eq:pfe} that in the partial fragmentation $f_o$ of $\pre{k-1}\nset^o$, the size of the fragmented even ancestral subtree is
  \begin{equation*}
    \abs{\pre{k}\nset^e_{-1}} = R_{-1}\abs{\pre{k-1}\nset^o},
  \end{equation*}
  which counts towards $N_{PDC}$ in the full fragmentation. In the fragmentation step $k$, $\pre{k-1}\nset^o$ is fragmented according to fragmentation ratios $\{R_0, \tilde{R}_1, \tilde{R}_2\}_k$. In the next fragmentation step $k+1$, the odd node-trees of $\m{F}_k=\{\pre{k}\nset^o_0, \pre{k}\nset^o_1, \pre{k}\nset^o_1\}$ are to fragmented into ancestral node-trees. The framentation ratios for each of the fragmentations $f(\pre{k}\nset^o_0), f(\pre{k}\nset^o_1), f(\pre{k}\nset^o_1)$ are not dependant on $\{R_0, \tilde{R}_1, \tilde{R}_2\}_k$. Thus global maximization of $N_{PDC}$ is achieved by local maximization of $R_{-1}=\tilde{R}_1+\tilde{R}_2$ in every fragmentation.
\end{proof}

\begin{theorem}\label{the:fragratio}
  For the fragmentation ratios $R_0 = \tilde{R}_1 = \tilde{R}_2 = \frac{1}{3}$, $N_{PDC}$ of Definition \ref{def:npdc} and Equation \eqref{eq:npdc} is maximized in a Union-Find Balanced-Bloom decoder with weighted growth (see Section \ref{sec:bucketwg}).
\end{theorem}
\begin{proof}
  Take the partial fragmentation $f_o$ of $\pre{k-1}\nset^o$ of Equation \eqref{eq:pfo} and $f_e$ of Equation \eqref{eq:newpfe}, which are equivalent to a final odd-join between $\nset^e_{-1}, \nset^o_0$ and even-join between $\nset^o_1, \nset^o_2$, respectively.

  For $f_e$ that is equivalent to the even-join to even-parity cluster $c_{-1}$ between the odd-parity clusters $c_1, c_2$ with node-trees $\nset^o_1, \nset^o_2$, the clusters must have relatively equal vertex-tree sizes
  \begin{equation*}
    \abs{\vset_1} \approx \abs{\vset_2}.
  \end{equation*}
  If not, clusters $c_1, c_2$ may be allowed to grow multiple iterations before merging, sorted by weighted growth. In each iteration, some $N_v$ vertices are added to the cluster. Let the growth iteration in which the even-join occurred be labeled as $i_e$

  For $f_o$ equivalent to the final odd-join between even $c_{-1}$ and odd $c_0$, the final odd-join must strictly occur after the even-join of $f_e$. This odd-join can either be initiated by odd-parity $c_0$ in some growth iteration $i_o > i_e$, when $c_2$ is the only odd-parity cluster, or initiated by either $c_1$ or $c_2$ in growth iteration $i_o = i_e$ (see Figure \ref{fig:fragfratio}). Determined by weighted growth, the vertex-tree sizes are related as
  \begin{equation*}
    \abs{\vset_0} \geq \abs{\vset_1} \approx \abs{\vset_2}.
  \end{equation*}
  Recall from equation \eqref{eq:sets} that $|\nset| \leq |\vset|$. We assume the largest possible node-tree size $|\nset| = |\vset|$ to find that
  \begin{equation*}
    \abs{\nset^o_0} \geq \abs{\nset^o_1} \approx \abs{\nset^o_2},
  \end{equation*}
  and hence
  \begin{equation*}
    R_0 \geq \tilde{R}1 \approx \tilde{R}2,
  \end{equation*}
  To maximize $N_{PDC}$, we want to maximize $\abs{\nset^e_{-1}} = \abs{\nset^o_1} + \abs{\nset^o_2}$ or $R_{-1}=\tilde{R}1 + \tilde{R}2$ (per Lemma \ref{lem:localmax}\footnote{Added in revision}). Since \eqref{eq:bigratios}, $R_0$ must be as small as possible, and thus $R_0 = \tilde{R}1 = \tilde{R}2 = \frac{1}{3}$.
\end{proof}

% \input{tikzfigs/fragratio}

The last unknown parameter for the maximization of $N_{PDC}$ in Equation \eqref{eq:npdc} is $p$, the total number of fragmentation steps. If we assume that in every growth step, not a single non-node vertex is added $N_v = 0$, the full fragmentation of odd node-tree $\pre{0}\nset^o$ is just the continuous division of the set in 3 parts per Theorem \ref{lem:fragratio}, which can be calculated easily.
\begin{equation}\label{eq:numfrag}
  p \leq \log_3(\abs{\pre{0}\nset^o})
\end{equation}
In every partial fragmentation set $\m{F}^o_k$, the sum of even node-tree sizes is
\begin{equation}\label{eq:sumevensetsize}
  \sum_j \left\{ \abs{\pre{k}\nset_j^e} \big| \pre{k}\nset_j^e \in \m{F}^o_k \right\} \leq \frac{2}{3}\abs{\pre{0}\nset^o},
\end{equation}
as $R_1+R_2 = \frac{2}{3}$ per Theorem \ref{the:fragratio}, and this value is constant for every fragmentation step per Lemma \ref{lem:equalevensum}. This is an inequality as we have assumed $|\nset| = |\vset|$ and $N_v=0$ in Theorem \ref{the:fragratio}. Filling in equation \eqref{eq:numfrag} and \eqref{eq:sumevensetsize} in \eqref{eq:npdc}, we find that
\begin{align}
  % \nonumber % Remove numbering (before each equation)
  \nonumber N_{PDC} & \leq \sum_{k=1}^{p} \sum_j \left\{ \abs{\pre{k}\nset_j^e} \big| \pre{k}\nset_j^e \in \m{F}^o_k \right\} \\
  \nonumber         & \leq \sum_{k=1}^{\log_3(\abs{\pre{0}\nset^o})} \frac{2}{3}\abs{\pre{0}\nset^o}                          \\
                    & \leq \frac{2}{3}\abs{\pre{0}\nset^o}\log{\abs{\pre{0}\nset^o}}.
\end{align}

Recall from Equation \eqref{eq:limitnsetsize} that the node-tree size is bounded by the lattice size $|\pre{0}\nset^o| \leq N$. The worst-case time complexity of the delay computation is thus bounded by $\m{O}(N\log{N})$. The average-case complexity is even lower as it is almost certain that not all vertices are nodes such that $|\nset| < |\vset|$ and $N_v \neq 0$.

\subsection{Bloom complexity}\label{sec:bloomcomplexity}

To grow a cluster represented by a node-tree $\nset$, a depth-first search is performed on the node-tree to iterate over each boundary list that are stored at the nodes.
\begin{definition}\label{def:nbloom}
  Let the total number of times nodes are bloomed with \codefunc{Bloom} be $N_{bloom}$.
\end{definition}

Similar to the previous section, we assume a maximum number of nodes on the lattice where, in each cluster $|\nset| = |\vset|$ and $N_v = 0$. For a fragmentation step of $f(\pre{k-1}\nset^o)$ to $\{\pre{k}\nset^o_0, \pre{k}\nset^o_1, \pre{k}\nset^o_2\}$, $N_{bloom}$ is maximized if all three ancestral node-trees are grown. As the growth of every set $\nset$ adds $|\nset|$ to $N_{bloom}$, the total number of bloom can be found similarly to $N_{PDC}$ in Equation \eqref{eq:npdc}. The sum is now on all odd node-tree sizes in all $p$ fragmentation steps $\m{F}_k$:
\begin{equation}\label{eq:nnode}
  N_{bloom} \leq \sum^{p}_{k=1}\sum_j \left\{ \abs{\pre{k}\nset^o_j} \big| \pre{k}\nset^o_j \in \m{F}_k \right\}
\end{equation}
For a full fragmentation of $\nset$ of size $|\nset|$, the sum of all set sizes in each fragmentation set $\m{F}$ is
\begin{equation}\label{eq:sumsetsfrag}
  \sum_j \left\{ \abs{\pre{k}\nset_j} \big| \pre{k}\nset^o_j \in \m{F}_k \right\} = \abs{\pre{0}\nset^o}.
\end{equation}
By filling in $p$ from \eqref{eq:numfrag}, we find that
\begin{align}
  % \nonumber % Remove numbering (before each equation)
  \nonumber & N_{bloom} & \leq \sum^{p}_{k=1}\sum_j \left\{ \abs{\pre{k}\nset^o_j} \big| \pre{k}\nset^o_j \in \m{F}_k \right\} & \\
  \nonumber &           & \leq \sum_{k=1}^{\log_3{\abs{\pre{0}\nset^o}}} \abs{\pre{0}\nset^o}                                  & \\
            &           & \leq \abs{\pre{0}\nset^o}\log_3{\abs{\pre{0}\nset^o}},                                               &
\end{align}
which again corresponds to a worst-case time complexity that is bounded by $\m{O}(N\log{N})$.
