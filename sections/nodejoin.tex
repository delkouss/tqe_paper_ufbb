\subsection{Joining node-trees}\label{sec:nodejoin}

\Figure[htb](topskip=0pt, botskip=0pt, midskip=0pt){tikzfigs/tikz-figure4.pdf}{(a) An odd cluster $\nset^o=\{s_1, s_2, s_3\}$ with root $n^o_r = s_1$ joins with an even cluster $\nset^e=\{s_4, s_5\}$ with root $n^e_r=s_4$ on nodes $s_3, s_5$, respectively, to a new set $\nset$ with subsets $'\nset^e$ and $'\nset^o$.  If we choose to (b), make $s_5$ a child of $s_3$, the parities and delays in $'\nset^o$ can be reused, and we only have perform partial parity and delay calculations over $'\nset^e$. If we choose to (c), make $s_3$ a child of $s_5$, parities and delays have to be recalculated over both $'\nset^2e$ and $'\nset^o$. \label{fig4}}

In the Union-Find algorithm, clusters of an odd number of non-trivial vertices or syndromes grow in size iteratively and merge with other clusters until all clusters are of even parity. During this process, the node-trees of the Node-Suspension data structure must also be merged. Let us now first make a clear distinction between the merging procotols of the underlying data structures; the vertex-trees of the UF data structure are merged with \codefunc{Union}, whereas the node-trees are merged with \codefunc{Join}. Once node-trees are joined, the node suspensions within the combined node-tree changes. The focus of the \codefunc{Join} protocol is therefore to minimize the DFS's of the recalculation of the node parity and delays in the combined node-tree. 

First, node that if a cluster of even parity has a even number of non-trivial vertices or syndromes, its node-tree has an even number of syndrome-nodes. For an even node-tree there are no PMW's as the matching can be made within the node-tree. Consequently, node suspension, parity and delays are undefined for an even node-tree. Thus, if two odd clusters merge into an even cluster, we don't know and do not care about its node suspensions. 

The second type is the merge between an even and an odd cluster. The combined cluster is odd and it will continue to grow in size, thus its node suspensions must be computed. Now, consider the example of odd node-tree $\nset_o$ and even node-tree $\nset_e$ that are to be joined on nodes $n_o\in \nset_o$ and $n_e \in \nset_e$ (Figure \ref{fig4}\emph{a}). If the root of $\nset_o$ is kept as the root of the joined node-tree (Figure \ref{fig4}\emph{b}), $n_e$ is to be a child node of $n_o$. As $\nset_e$ contains an even number of syndrome-nodes, the node parities in $\nset_e$ do not change. The DFS of the node parity recalculation is only necessary in $\nset_e$, which now has $n_e$ as sub-root. Furthermore, as the node delay is only dependant on the properties of a node and of its parent node, the DFS of the node delay recalculation is also only required from node $n_e$ and is performed within $\nset_e$. These paritial DFS's of the even sub-tree are exactly what was required as the node parity and delays in $\nset_e$ were undefined. If the root of $\nset_e$ takes the role of the root of the combined tree (Figure \ref{fig4}\emph{c}), an odd number of syndrome-nodes are attached to $n_e$, such that the parities of nodes on the path from $n_e$ to the root are changed. Such a join would require the DFS's for the entire combined node-tree of the recalculation of node parities and delays. A simple rules is thus to always keep the root of the odd node-tree. 

Finally, a cluster can be subjected to merges with multiple other clusters within the same growth iteration, duing which the merged cluster may switch parity multiple times. The DFS's related to the recalculation of the node parities and delay must for this reason not be initiated directly of the the joining of node-trees. Instead, a pointer to the sub-root of the even sub-tree in the last odd-even join is stored at the root of the node-tree. The recalculation is then initiated just before cluster growth to prevent multiple recalculations over the same partitions of the node-tree. 

% to even -> node suspension undefined


% \Figure[htb](topskip=0pt, botskip=0pt, midskip=0pt){tikzfigs/tikz-figure5.pdf}{bla\label{fig5}}

% Equilibruim