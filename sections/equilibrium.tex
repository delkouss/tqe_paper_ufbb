\subsection{Parity inversion}

\Figure[htb](topskip=0pt, botskip=0pt, midskip=0pt){tikzfigs/tikz-figure5.pdf}{
    (a) The delay values $n_i.d$ and the equilibrium-states $(I:k_{eq}M)$ for 3 odd clusters $\{\nset_1, \nset_2, \nset_3\}$ of 3 nodes that grow and join into a size-9 cluster. (Top) Initially, parity and delay calculations  are performed with delay equation \eqref{eq:2ddelay} on each odd cluster which have equilibrium-states $(0:2)$, with delay $2$ in the middle node. (Middle) The clusters are grown, where the middle node is delayed, such that it's delay value decreases to $1$, and the clusters have equilibrium-states $(1:2)$. (Bottom) The clusters join to a single odd cluster, which is selected for growth. Hence, parity and delay calculation is performed again, and the equilibrium-state is $(0:4)$, thus requiring 4 growth iterations before equal potential matching weight is reached in all nodes.  (b) The same clusters, growths and joins, but now with delay equation \eqref{eq:delayequation} for $k_{eq} = 1/2$. With the equilibrium factor, $(k_{eq}M:k_{eq}M)$ can be reached in fewer growth iterations; e.g. after 1 round (middle), $(1:1)$ is reached. Also, after join to a single cluster (bottom), fewer iterations are needed (2 compared to 4 in Figure \ref{fig:kbloom}).\label{fig5}}