\section{Minimizing matching weight}\label{sec:matchingweight}

In the following we give some intuition into how the UFBB decoder works. Consider the cluster of vertices $\mathcal{V}=\{v_0,v_1,v_2\}$ in Figure \ref{fig0}. Now let us investigate the weights of a matching if an additional vertex $v'$ is connected to the cluster. 
If $v'$ is connected to $v_0$ or to $v_2$, then the resulting matching has a total weight of 2: $(v',v_0)$ and $(v_1,v_2)$, or $(v_0,v_1)$ and $(v_2,v')$. However, if $v'$ is connected to vertex $v_2$, then the total weight is 3: $(v', v_1)$ and $(v_0, v_2)$. Inspired by this idea, we introduce the concept of potential matching weight (PMW) of a vertex. 

\Figure[h](topskip=0pt, botskip=0pt, midskip=0pt){tikzfigs/tikz-figure0.pdf}{Unbalanced matching weight in cluster vertex set $\mathcal{V}$. The matching edges (dashed) correspond to the position of $v'$.\label{fig0}}

% \tikzfading[name=fade right, left color=transparent!0, right color=transparent!100]

% \begin{figure}[htb]
%   \centering
  
%   \caption{Unbalanced matching weight in cluster vertex set $\mathcal{V}$. The matching edges (dashed) correspond to the position of $v'$. }\label{fig1}
% \end{figure}


% The calculation of the PMW requires to introduce a new data structure that we call the node-set of a cluster. The node-set of a cluster is a partition of the cluster such that each element of the partition consists of a set of adjacent vertices that are either interior elements in the associated graph or have equal PMW (Figure \ref{fig2}). The calculation of the PMW is performed by two depth-first searches of the node-set (Figure \ref{fig3}), where the PMW is translated into a node delay that determines the priority for cluster growth, such that the growth of a cluster moves towards equal PMW in the cluster (Balanced Bloom).

% In this chapter, we describe a modification of the Union-Find decoder, dubbed the \emph{Union-Find Balanced-Bloom} decoder, that increases the Union-Find decoder's code threshold by improving its heuristic for minimum-weight matching. We show that the modified decoder retains a relatively low time-complexity. 

% Within the vanilla Union-Find decoder, not all odd-parity clusters are grown at the same time. Larger clusters relatively add more ``incorrect edges'' to themselves than compared to a smaller cluster (lemma \ref{lem:incorrectedges}). The Union-Find decoder therefore applies \emph{weighted growth} of clusters, where the order of cluster growth is sorted based on the cluster sizes. We have shown a linear-time implementation utilizing \emph{bucket sort} in Section \ref{sec:bucketwg}. With the addition of weighted growth, the error threshold of the Union-Find decoder is reported to increase from $9.2\%$ to $9.9\%$ for a 2D toric lattice \cite{delfosse2017almost}. However, we measured an increase from $9.716\%$ to $9.984\%$ in our implementation of the decoder (Section \ref{sec:ufperformance}). We showed that by always maintaining a dynamic forest where all clusters are connected acyclic graphs (Section \ref{sec:dynamicforest}), a slight increase in the code threshold and reduced running time can be obtained. 

% From the simulated results of several of our implementations of the Union-Find decoder (Section \ref{sec:ufimplementations}), we grew the intuition that a decreased weight is, to some extent, a heuristic for an increased code threshold (formalized in Proposition \ref{prop:mw2}). In fact, the instruction of the Union-Find decoder mostly tries to obtain a low weight matching; by growing the odd-parity clusters on their boundaries a single layer at the time, unions between odd-parity clusters mostly occur on nearest neighbors. The lattice's discrete coordinates limit the number of growth iterations to a constant proportional to the lattice. But also due to this discreteness, there may be many unions within each growth iteration, and nearest-neighbor unions between clusters may not result in a minimum-weight matching between syndromes. Especially as the clusters increase in size, also does their boundaries, and an increasingly larger amount of ``incorrect edges'' are added to the cluster. Weighted growth reduces the number of large-cluster growths, but does not decrease the number of ``incorrect edges'' if a large cluster is grown. This leaves us with the question: Should all boundary edges of a cluster be grown simultaneously?

% We suspect that the error threshold of the Union-Find decoder can be increased by improving the heuristic for minimum-weight matching. In this chapter, we accomplish this by sorting the growth of specific subsets of a cluster according to a parameter that we dub the \emph{potential matching weight}, explained in \ref{sec:PMW}. We introduce a new data structure that we call the \emph{node-tree} of a cluster in \ref{sec:nodeset}. We compute the node \emph{parity} and \emph{delay} within this node-tree in \ref{sec:nodedelay}, which set the order of boundary edge growth. In \ref{sec:growingcluster} through \ref{sec:nodejoin}, we cover the rules for growth and join operations for the node-trees, which are more complex than those of the Union-Find decoder. The modified decoder, the Union-Find Balanced-Bloom decoder, still has a relatively low worst-case quasilinear-time complexity, approximated in \ref{sec:ufbbcomplexity}. 

% \section{A potential matching weight}\label{sec:PMW}

% The Minimum-Weight Perfect Matching decoder finds the minimum-weight subset of edges by constructing a fully connected graph between all vertices (Chapter \ref{ch:mwpm}). By computing on the entire lattice, we denote such a decoder as a \emph{global} decoder. The Union-Find decoder is a \emph{local} decoder, as each cluster is grown individually, oblivious about its surrounding neighbors until it merges into them. In this section, we introduce the concept of a \emph{potential matching weight} of an odd-parity cluster, and we show that its value is not constant across the vertices of a cluster. Recall from Definition \ref{def:cluster} that a cluster with index $i$ is defined as an object $c_i$ with an edge set $\m{E}_i$ and vertex-tree $\m{V}_i$.

% \begin{definition}\label{def:pmw}
%   Consider an odd-parity cluster $c_i$ containing a vertex $v$. The Potential Matching Weight (PMW) of the vertex $v$ is the matching weight in the subset of edges of an odd-parity cluster $c_i$ in a hypothetical union with another cluster $c_j$ in the next growth iteration, where the merging boundary edge is supported by $v$. 
%   \begin{equation}
%     PMW(v) = \abs{\m{C} \cap \m{E}_{i}} + 1 \text{ if } c_i, c_j \text{ merged by } \codefunc{Union}(v,u) | v \in \delta\m{V}_{i}, u \in \delta\m{V}_{j}, 
%   \end{equation}
%   In other words, the potential matching weight is a vertex-specific predictive heuristic to the matching weight assuming a union in the next growth iteration. 
% \end{definition}

% Note that for this reason, the potential matching weight is not defined for a vertex that is not in the boundary of a cluster. Let us first consider an example. Cluster $c_e$ is defined by vertex-tree $\vset_e = \{v_1, v_2, v_3\}$, where each vertex is a syndrome-vertex $\vset_e \subset \sigma$ (Figure \ref{fig:PMW}). The vertices lie on a horizontal line, distance 1 from each other, where each vertex has grown a single iteration of half-edges. The cluster has odd parity and is queued for growth. Let us investigate the weights of a matching if an additional vertex $v_o$ is connected to the cluster. If $v_o$ is connected to $v_1$ or to $v_3$, then the resulting matching has a total weight of 2: $\{(v_o,v_1), (v_2,v_3)\}$ or $\{(v_o,v_3),(v_1,v_2)\}$, respectively. However, if $v_o$ is connected to vertex $v_2$, then the total weight is 3: $\{(v_o, v_2),(v_1, v_3)\}$, where $(v_1,v_3)$ has weight 2. 

% From the above example, we can see that even for a minimal-sized odd cluster that is not a single vertex, the potential matching weight is not equal for all vertices in the cluster. Therefore, it would not be optimal to grow all boundary edges simultaneously, as boundaries connected to vertices with a high potential matching weight potentially result in a higher matching weight. The growth of these high potential matching weight boundaries should thus be delayed for some iterations. When the PMV across the cluster reaches an equilibrium, there is no benefit of growing some boundaries before others, and simultaneous growth is allowed again.

% \input{tikzfigs/potentialmatchingweight}

However, the calculation of the potential matching weight is seemingly not as straight forward, especially for clusters of increasingly larger size. Furthermore, if the potential matching weight is to be calculated for every vertex with boundary edges in all clusters in every growth iteration, the algorithm's time complexity would increase dramatically. Luckily, we can reduce these calculations to be performed on a set of \emph{nodes} in each cluster, which we clarify in the next section.
