\documentclass[twocolumn,pra,nofootinbib]{revtex4}

\usepackage[utf8]{inputenc}
\usepackage[T1]{fontenc}
\usepackage[english]{babel}
\usepackage{amsfonts}
\usepackage{amsmath}
\usepackage{amssymb}
\usepackage{amsthm}
\usepackage{amsbsy}
\usepackage{placeins}
\usepackage{setspace}
\usepackage{tabularx}
\usepackage{hhline}
\usepackage{multirow}
\usepackage{enumitem}
\usepackage{textcomp}
\usepackage{cleveref}
\usepackage{nicefrac}
\usepackage{mathtools}
\usepackage[linesnumbered, ruled, vlined]{algorithm2e}

% Set table
\newcolumntype{L}[1]{>{\hsize=#1\hsize\raggedright\arraybackslash}X}%
\newcolumntype{R}[1]{>{\hsize=#1\hsize\raggedleft\arraybackslash}X}%
\newcolumntype{C}[1]{>{\hsize=#1\hsize\centering\arraybackslash}X}%
\newcommand{\gc}{\cellcolor[gray]{0.9}}

% Set theorems
\newtheorem{definition}{Definition}[section]
\newtheorem{lemma}{Lemma}[section]
\newtheorem{theorem}{Theorem}[section]
\newtheorem{proposition}{Proposition}[section]

% Set new commands
\let\oldemptyset\emptyset
\let\emptyset\varnothing
\newcommand{\m}[1]{\mathcal{#1}}
\newcommand{\n}[1]{\mathscr{#1}}
\newcommand{\bound}{\mathscr{B}}
\newcommand{\akker}{\mathscr{A}}
\newcommand{\nset}{\mathcal{N}}
\newcommand{\vset}{\mathcal{V}}
\newcommand{\pr}[1]{ {}^{#1} }
\newcommand{\ceil}[1]{{\left \lceil #1 \right \rceil }}
\newcommand{\floor}[1]{{\left \lfloor #1 \right \rfloor }}
\DeclarePairedDelimiter{\abs}{\lvert}{\rvert}
\DeclareMathOperator{\Find}{Find}
\DeclareMathOperator{\Union}{Union}
\DeclareMathOperator{\Nodejoin}{Join}
\DeclareMathOperator{\Support}{Support}

% Set algorithm2e package settings
\SetAlgoCaptionLayout{centerline}
\setlength{\algoheightrule}{1pt}
\setlength{\algotitleheightrule}{1pt}
\setlength{\interspacetitleboxruled}{.5em}
\SetStartEndCondition{ }{}{}
\SetKwProg{Fn}{def}{\string:}{}
\SetKw{KwTo}{in}
\SetKwFor{For}{for}{\string:}{}
\SetKwIF{If}{ElseIf}{Else}{if}{ then}{else if}{else}{}
\SetKwFor{While}{while}{ do}{}
\SetInd{0.1em}{0.5em}
\SetAlgoNoEnd\DontPrintSemicolon
\SetAlFnt{\small}


\makeatletter
    \def\thefigure{\@arabic\c@figure}
    \def\fps@figure{tbp}
    \def\ftype@figure{1}
    \def\ext@figure{lof}
    \def\figurename{FIGURE}
    \def\fnum@figure{\figurename~\thefigure}
    \define@key{figkeys}{topskip}{\def\fig@topskip{#1}} % skip above image
    \define@key{figkeys}{midskip}{\def\fig@midskip{#1}} % skip between image and caption
    \define@key{figkeys}{botskip}{\def\fig@botskip{#1}} % skip below caption
    \setkeys{figkeys}{topskip=0pt,midskip=0pt,botskip=0pt}%
    \newsavebox\figbox
    \def\Figure{\@ifnextchar[{\@Figure}{\@Figure[t!]}}
    \def\@Figure[#1]{\@ifnextchar({\@@Figure[#1]}{\@@Figure[#1](topskip=0pt)}}
    \def\@@Figure[#1](#2){\@ifnextchar[{\@@@Figure[#1](#2)}{\@@@Figure[#1](#2)[scale=1]}}
    %%%%%%%%%%%%
    \def\@@@Figure[#1](#2)[#3]#4#5{%
        \setkeys{figkeys}{#2}% Set new keys
        \setbox\figbox\hbox{\includegraphics[#3]{#4}}%
        \xdef\xfigwd{\the\wd\figbox}%
        \ifdim\wd\figbox<\columnwidth% Columnwidth Figure
            \begin{figure}[#1]%
            \centering%
            \vspace*{5pt}%
            \vspace*{\fig@topskip}%
            \includegraphics[#3]{#4}%
            \vspace*{\fig@midskip}%
            \caption{#5}%
            \vspace*{\fig@botskip}%
            \end{figure}
        \else% Wide Figure
            \begin{figure*}[#1]%
            \centering%
            \vspace*{5pt}%
            \vspace*{\fig@topskip}%
            \includegraphics[#3]{#4}%
            \vspace*{\fig@midskip}%
            \caption{#5}%
            \vspace*{\fig@botskip}%
            \end{figure*}
        \fi%
    }
\makeatother

\begin{document}
    \title{Quasilinear-Time Decoding Algorithm for Topological Codes with High Decoding Performance}
    \author{Shui Hu}
    \affiliation{Department of Physics, Delft University of Technology}
    \email{watermarkhu@outlook.com}

    \author{David Elkouss}
    \affiliation{QuTech, Delft University of Technology, Lorentzweg 1, 2628CJ Delft, The Netherlands}


    \begin{abstract}
        Quantum computing has the potential to transcend the information technology as we know it. Small scale quantum systems are already possible today, and the goal is to scale up these quantum architectures to build practical quantum devices. A major limiting factor in quantum computing is the accumulation of errors that may be caused by various sources. A solution is to encode the logical information in a larger amount of physical qubits to increase resilience, thereafter to decode the information with a decoder. A recently proposed decoder dubbed the Union-Find (UF) decoder is fast and almost linear in its worst-case time complexity, but has a reduced performance compared with the Minimum-Weight Perfect Matching (MWPM) decoder. We propose a modification of the UF decoder that aims to further decrease the weight of the correction operator, similarly to the MWPM decoder. The modified decoder, dubbed the Union-Find Node-Suspension (UFNS) decoder, manages to have an improved performance compared to the UF decoder on any lattice size. For small lattice sizes and also for low error rates, the UFNS decoder performs near identically to the MWPM decoder. We manage to maintain a quasilinear worst-case time complexity of $\m{O}(n\log{n})$. 
    \end{abstract}

    \maketitle

    \section{Introduction}\label{sec:introduction}
One of the most promising approaches for fault-tolerant quantum computation is based on surface quantum error-correcting codes \cite{dennis2002topological, kitaev2003fault}. With surface codes, error correction only requires the measurement of local operators on a 2-dimensional lattice of qubits. The measurement outcome, called the syndrome, is passed to the decoding algorithm to deduct the error that has occurred and to supply a correction operator. The resilience against errors can be improved by increasing the system size while the physical error rate is below a threshold value $p_{th}$. For this, it is essential that the decoder has low time complexity; if the clock-rate of the quantum computer becomes limited by the decoder, the advantages of increasing the system size could be compromised.

Many decoding algorithms have been developed that either aim to improve the threshold and lower logical error rates \cite{wang2003confinement, raussendorf2007faulttolerant, fowler2012towards, fowler2013minimum, heim2016optimal, duclos2010fast, duclos2013fault, bravyi2014efficient, darmawan2018linear}, or to perform under more realistic noise models \cite{tuckett2020fault, hutter2015improved, bravyi2013quantum,  nickerson2019analysing, wootton2012high, huang2020fault}, including a new class of neural network decoders \cite{baireuther2019neural, chamberland2018deep, liu2019neural, nautrup2019optimizing, torlai2017neural, varsamopoulos2017decoding, varsamopoulos2020decoding}, and other types \cite{bombin2012universal, herold2015cellular, horsman2012surface, kubica2019cellular, watson2015fast}. Identifying the optimal decoder of the code depends critically on the noise model. The most popular decoder for surface codes is unarguably still the Minimum-Weight Perfect Matching (MWPM) decoder. It performs near-optimal for a bit-flip noise model \cite{dennis2002topological} on a standard non-bounded surface code with a threshold of $p_{th} = 10.3\%$, and for a phenomenological noise model \cite{wang2003confinement}, which includes faulty measurements, with $p_{th} = 2.9\%$. This approach's basic principle is to identify the \emph{lowest weight} error configuration that can produce the syndrome. The minimum-weight matching is found by constructing a fully connected graph between nodes of the syndrome, which leads to a cubic worst-case time complexity of $\mathcal{O}(n^3)$, where $n$ is the number of qubits in the system \cite{kolmogorov2009blossom}. Fowler has proved that the matching problem can be solved in average $\mathcal{O}(1)$ time, but only at sufficiently low error rates, and the worst-case complexity remains significant \cite{fowler2013minimum}. 

In this work, we build on top of a recently proposed decoder called the Union-Find (UF) decoder. It combines a very low worst-case time complexity with a high threshold \cite{delfosse2017almost}, making it a practical solution for real devices. The UF decoder's thresholds on the toric code with independent bit-flip noise and phenomenological noise are $9.9\%$ and $2.7\%$, and its worst-case time complexity is $\mathcal{O}(n\alpha(n))$, where $\alpha$ is the inverse of Ackermann's function \cite{tarjan1975efficiency}. For any physically feasible amount of qubits, this value is $\alpha(n) \leq 3$, leading to an ``almost-linear'' time complexity.

We propose here a modification of the UF decoder that improves the heuristic for minimum-weight matching. The modified decoder, which we dub the \emph{Union-Find Node-Suspension decoder} (UFNS), achieves near MWPM performance while retaining a quasilinear time complexity. In sections \Cref{sec:surfacecode,sec:unionfind}, we introduce the surface code and the UF decoder. In \Cref{sec:ufbb}, we describe the modified algorithm and its motivation. We discuss the algorithm's complexity in \Cref{sec:complexity} and compare its performance with other decoders in \Cref{sec:performance}.  

\section{The Surface Code}\label{sec:surfacecode}

The Union-Find Node-Suspension decoder proposed here, similarly to its parent decoder, applies to any surface code of any genus, with or without boundary, and to color codes \cite{delfosse2017almost}. For simplicity, we only describe the standard implementation of the surface code without boundary.

The \emph{toric code}, a topological code introduced by Kitaev \cite{kitaev2003fault}, is defined by arranging qubits on the edges of a square lattice with periodic boundary conditions. The code is denoted by $V,E,F$, respectively the set of vertices, set of edges, and the set of faces on the lattice. The toric code is defined to be the ground state of the Hamiltonian 
\begin{equation}
    H = -\sum_{v \in V} X_v -\sum_{f \in F} Z_f, 
\end{equation}
where operator $X_v$ is the product of Pauli $X$ operators on the qubits located on edges forming the vertex $v$, \emph{i.e.}, $X_v = \prod_{e \in v} X_e$, and $Z_f = \prod_{e \in f} Z_e$ is the product of Pauli $Z$ operators on the qubits located on edges of face $f$. The code space is spanned by the simultaneous ``+1'' eigenstate of all operators $X_v$ and $Z_f$. Together with any possible product of them, these operators are the \emph{stabilizers} of the code and form the stabilizer group $S$. The torus' non-trivial cycles encode the logical operators. Below a certain threshold, errors will only introduce local effects and do not change these cycles.

For simplicity, we consider independent or non-correlated noise caused by i.i.d. bit-flip errors, where each qubit is subjected to a Pauli $X$ error with probability $p_X$. Due to \emph{lattice duality}, the error detection and correction of phase-flip errors are identical. The phenomenological noise model adds noisy measurements with the probability of error during each measurement equal to $p_X$. 
%Additionally, any qubit may be \emph{erased} from the system with probability $p_e$. The set of erased qubits is denoted with $\varepsilon$. This \emph{erasure} is detectable, such that we can replace or reinitiate all erased qubits, which corresponds to a random Pauli error after measurement. 

Error correction is proceeded by measuring a set of independent stabilizers of the code, \emph{i.e.}, the operators $X_v$ and $Z_f$. For a set of phase-flip errors $E_Z = \{I,Z\}^{\otimes n}$, the stabilizers $X_v$ that anticommute with the error return a non-trivial outcome. The set of non-trivial eigenvalues of the stabilizers is called the syndrome $\sigma$ of the code. Given the measured $\sigma$, it is the task of the decoder to find the correction operator $\mathcal{C}(\sigma)$. When the correction operator is applied, the code is returned to the code space, \emph{i.e.}, $\mathcal{C}(\sigma)E_Z \in S$. The error is corrected up to a stabilizer. The mapping of the measured syndrome to the correction is thus not one-to-one. It is up to the decoder to choose the correction most similar to the error. 

\section{Union-Find decoder}\label{sec:unionfind}

The Union-Find decoder \cite{delfosse2017almost} maps each element of the syndrome $\sigma$ to a so-called non-trivial vertex $v$ in a non-connected graph on the code lattice, and grows clusters that form a connected graph $G(V_i, E_i)$ of vertices $V_i\in V$ and edges $E_i \in E$ locally, by repeatedly adding a layer of edges and trivial vertices to existing clusters, until all clusters have an even number of non-trivial syndrome vertices. This process is described as the growth of a cluster or the growth of the vertices that lie on a cluster's boundary. Then, a spanning tree $F$ is built, such that each cluster is a connected acyclic graph., Leaves of the trees are conditionally peeled in a tail-recursive breadth-first search until all non-trivial syndrome vertices are paired and linked by a path within $F$, which is the correcting operator $\mathcal{C}$ \cite{delfosse2017linear}. By growing the clusters of vertices in order of their sizes --- the number of vertices in the cluster --- the threshold is reported to increase from $9.2\%$ to $9.9\%$ for bit-flip noise in this \textbf{Weighted Growth} variant of the decoder.

The merging between clusters drives the complexity of the Union-Find decoder. For this, the algorithm uses the Union-Find or disjoint-set data structure \cite{tarjan1975efficiency}. The function $\Find(v)$ is used to travel in the cluster-tree --- a disjoint set of the cluster's vertices --- from vertex $v$ to the representative root element $r_v$, to identify the cluster to which $v$ belongs. When two vertices $u, v$ are connected on a newly added edge $(u,v)$, $\Find(u), \Find(v)$ output the roots $r_u, r_v$. If $r_u \neq r_v$, the two cluster-trees are not connected. The two cluster are then mergeds by $\Union(r_u, r_v)$ by pointing one tree's root to another root. \textbf{Weighted Union} is performed by pointing the smaller tree to the larger to reduce the cost of future calls to $\Find$.
    \section{Union-Find Node-Suspension decoder}\label{sec:ufbb}

In this section, we describe the \emph{Union-Find Node-Suspension} decoder, which increases the Union-Find decoder's performance by improving its heuristic for minimum-weight matching. We first introduce the concept of the potential matching weight in \Cref{sec:matchingweight}. We describe the data structure required for this decoder in \Cref{sec:nodeset}, and the necessary calculations performed on this data structure in \Cref{sec:paritydelaysus,sec:nodejoin,sec:inversion}. The pseudocode is included in \Cref{sec:pseudocode}. 

\Figure[htb](topskip=0pt, botskip=0pt, midskip=0pt){figures/tikz-figure0.pdf}{
    A cluster with vertices $\{v_0, v_1, v_2\}$ with potential matching weights $\{2, 3, 2\}$. The line style and color of the colored edges correspond to the matching in the hypothetical union with an external vertex $v'$ of the same line style and color.\label{fig0}}

\subsection{Potential matching weight}\label{sec:matchingweight}
%In the following we give some intuition into the improvement of the Union-Find Balanced Bloom decoder upon the original Union-Find decoder. 
% We compared the ratio of the matchings between the MWPM decoder and our own implementation of the UF decoder, averaged over many simulations, and found that UF matching weight has a constant prefactor of $\sim 1.043$ over the minimum weight for the toric code (\Cref{comp_weight}). From this, we suspected that a decreased matching weight is a heuristic for an increased threshold. Within the context of the UF decoder, the matching weight may be decreased by prioritizing the growth of vertices with low PWM's within the cluster. 

Consider the cluster with index $i$ containing the set of non-trivial vertices $V_i=\{v_0,v_1,v_2\}$ and set of edges $E_i=\{(v_0,v_1), (v_1, v_2)\}$ of \Cref{fig0}. Now let us investigate the weight of a matching if an additional non-trivial vertex $v'$ is connected to the cluster. If $v'$ is connected to $v_0$ or to $v_2$, then the resulting matching has a total weight of 2: $(v',v_0)$ and $(v_1,v_2)$, or $(v_0,v_1)$ and $(v_2,v')$. However, if $v'$ is connected to vertex $v_2$, then the total weight is 3: $(v', v_1)$ and $(v_0, v_2)$. Inspired by this idea, we introduce the concept of potential matching weight (PMW) of a vertex. 

\begin{definition}\label{def:pmw}
    % For, the hypothetical merger with another odd-parity cluster $V_j, E_j$ on the edge $(v_i, v_j)$, with $v_i\in V_i$ and  $v_j \in V_j$, outputs an even-parity cluster with edges $E_{ij} = E_i \cup E_j \cup (v_i, v_j)$ in which there exists a matching $\m{C}_{(v,v')} \subseteq E_{ij}$ between syndromes internal to the cluster.
    % Let the Potential Matching Weight (PMW) of vertex $v_\alpha \in V_\alpha$ in an odd-parity cluster $\alpha$ with vertices $V_\alpha$ and edges $E_\alpha$ be
    % \begin{equation}
    %   PMW(v_\alpha) = \abs{\m{C}_{(v_\alpha,v_\beta)} \cap E_\alpha} + 1,
    % \end{equation}
    % where matching $\m{C}_{(v,v')} \subseteq E_{ij}$ is between the syndrome vertices internal to the even-parity cluster with edges $E_{ij} = E_\alpha \cup E_\beta \cup (v_\alpha, v_\beta)$, after a hypothetical merger of cluster $\alpha$ with another odd-parity cluster $V_\beta, E_\beta$ on the edge $(v_\alpha, v_\beta)$, with $v_\alpha\in V_\alpha$ and  $v_\beta \in V_\beta$
    Let there be a hypothetical merger between odd cluster $\alpha$ of vertices $V_\alpha$ and edges $E_\alpha$, and odd cluster $\beta$ of $V_\beta$ and $E_\beta$, on the edge $(v_\alpha, v_\beta)$, where $v_\alpha \in V_\alpha$ and $v_\beta \in V_\beta$. In the merged even cluster with edges $E_{\gamma} = E_\alpha \cup E_\beta \cup (v_\alpha, v_\beta)$, there is a matching $\m{C}_{(v_\alpha,v_\beta)} \subseteq E_{\gamma}$  between the syndrome vertices internal to the cluster. The \textbf{Potential Matching Weight} (PMW) of vertex $v_\alpha$ is then defined as
    \begin{equation}
      PMW(v_\alpha) = \abs{\m{C}_{(v_\alpha,v_\beta)} \cap E_\alpha} + 1.
    \end{equation}
\end{definition}

In other words, the PMW is a vertex-specific predictive heuristic to the matching weight, assuming a union occurs in the next growth iteration. The PMW can be utilized by prioritizing the growth of vertices with low PMW such that there is an increased probability of mergers between clusters on edges connected to these vertices, and there is an increased probability in a lower matching weight. However, the PMWs' calculation within a cluster is not a trivial task, especially for clusters of increasingly larger size, as all edges of a cluster must be considered in its calculation. Furthermore, the PMWs within a cluster change due to cluster growth and mergers, both of which occur more frequently as the system size increases. For this reason, the scaling of the PMW computation is vital to the decoder. 

\Figure[htb](topskip=0pt, botskip=0pt, midskip=0pt){figures/tikz-figure1.pdf}{
    The cluster of \Cref{fig0} after two rounds of prioritized growth of $v_0$ and $v_2$. There are regions of vertices that are either interior elements or have equal potential matching weights, represented as nodes with different node radii in the node-tree $\nset$. \label{fig1}}

\subsection{Node-Suspension data structure}\label{sec:nodeset}

Fortunately, the PMWs' calculation is quite efficient by the introduction of a new data structure. Consider the cluster of non-trivial vertices $V_i=\{v_0,v_1,v_2\}$ and edges $E_i = \{(v_0,v_1), (v_1, v_2)\}$ from \Cref{fig0}. We had found previously that vertices $v_0, v_2$ have a lower PMW compared to $v_1$ by 1 edge. The growth of $v_0$ and $v_2$ are thus prioritized, such that new vertices are added to the cluster on the boundary of $v_0$ and $v_2$. If all newly added vertices are trivial, the cluster is now as in \Cref{fig1}. If we repeat the PMW calculation, we now find that the PMWs in the new vertices connected to $v_0$ are equal. The same is true for vertices connected to $v_2$. 
\begin{definition}\label{def:vertextree}
    Let the vertex-tree $\vset_i$ be a connected acyclic subgraph of the graph of a cluster $G(V_i, E_i)$.   The vertex-tree $\vset_i$ includes all vertices $V_i$ and a minimum number of edges in $E^\vset_i \subseteq E_i$. 
\end{definition}
\begin{definition}
  Let the node-tree $\nset_i$ be a partition of the vertex-tree $\vset_i$, such that each element of the partition --- a \textbf{node} $n$ --- consists of a set of adjacent vertices that lie within a certain distance --- the \textbf{node radius} $r$ --- from the \textbf{primer vertex}, which initializes the node and lies at its center. The node-tree is a directed acyclic graph, and its edges $\m{E}_i$ have lengths equal to the distance between the primer vertices of neighboring nodes. 
\end{definition}

The concept of primer vertices is easily understood when considering non-trivial vertices of the syndrome $\sigma$. Suppose every non-trivial vertex is the primer of a node, the weight of a matching in $\vset_i$ equal to the weight of the same matching in $\nset_i$. Furthermore, for every node of the node-tree, all vertices that lie at distance $r$ to the primer vertex are either boundary vertices to the cluster and have equal PMW, or lie within the radius of another node. For the example in \Cref{fig1}, the PMW of all boundary vertices of $n_0$, for simplicity just the PMW of $n_0$, is $\floor{r_0} + (n_1, n_2) + 1$. The partition from $\vset$ to $\nset$ thus allows us to compute the PMW on a reduced tree. 

\Figure[htb](topskip=0pt, botskip=0pt, midskip=0pt){figures/tikz-figure2.pdf}{
    Two different types of nodes. Syndrome-nodes $s$ have a non-trivial vertex or syndrome at its center. Vertices that lie on the radii of two existing nodes initialize a junction-node $j$ in the node-tree.\label{fig2}}

\Figure[hbt](topskip=0pt, botskip=0pt, midskip=0pt){figures/tikz-figure3.pdf}{
    The relevant data structures. \emph{(a)} The cluster-tree of the Union-Find data structure. The path from a vertex to the root of the cluster-tree is traversed to find the root element in order to differentiate between clusters. The root node of the node-tree is now additionally stored at the root of the cluster-tree. \emph{(b)} The vertex-tree $\vset$ with 9 non-trivial vertices. As $\vset$ is strictly acyclic, the cluster's edges must be maintained such that no cycles are created. This is done during growth by removing edges (red dotted lines) if a cycle is detected. \emph{(c)} The node-tree $\nset$, which currently has the same number of elements as $\vset$, as all vertices are non-trivial. Two depth-first searches are required to compute node parities (head recursively) and delays (tail recursively) in $\nset$.\label{fig3}}

All non-trivial vertices serve as primers for nodes that are called \textbf{syndrome-nodes} $s$. However, not all primer vertices are non-trivial vertices of the syndrome. If two non-trivial vertices are located an even Manhattan distance on the lattice, the growth of their clusters can simultaneously reach some vertex that lies on equal radii of the associated nodes, such as in \Cref{fig2}. For this reason, such vertices serve as primers of a different type of node --- a \textbf{junction-node} $j$ --- in the merged node-tree. 

The calculation of the PMW on the node-tree $\nset$ rather than the vertex-tree $\vset$ offers a reduction in the cost. However, it is still no trivial task as the entire tree must be considered for the calculation in every node. Instead, we will compute for the \textbf{node suspension} $\m{s}(n)$ --- the number of growth iterations needed for a node to reach the maximum PMW in the node-tree --- which relates closely to the PMW. For example, the node suspension for the nodes $\{n_0, n_1, n_2\}$ associated with the vertices $\{v_0, v_1, v_2\}$ in \Cref{fig0} is $\{0, 2, 0\}$, and $\{0, 1, 0\}$ in \Cref{fig1}.

The Node-Suspension data structure does not replace but coexists with the Union-Find data structure. Additional to the the Union-Find data structure's cluster-trees of distinct roots, we store for every cluster the node-tree $\nset_i$ by its root node. For this, we need to maintain the reduced set of edges $E^\vset_i \subseteq E_i$ of the vertex-trees $\vset_i$ for every cluster, which can be done in constant time (see Algorithm \ref{algo:ufbb}). In the UF decoder, vertex-trees $\m{V}_i$ are not maintained, such that the graph associated with each cluster is not acyclic \cite{delfosse2017almost}. Instead, a spanning forest $F$ of all clusters is created \cite{delfosse2017linear} after growth. Each connected element within $F$ is also an acyclic graph. The difference is that while a single depth-first search or breath-first-search creates $F$, $\vset$ is equivalent to multiple breadth-first searches from each non-trivial vertex within the cluster, where the search of every breadth occurs during a growth iteration. The relevant data structures are depicted in \Cref{fig3}. 


\subsection{Node parity, delay, and suspension}\label{sec:paritydelaysus}

The Node-Suspension data structure allows for calculating the node suspension of all nodes in a node-tree $\nset$ by two intermediate steps. In each step, a depth-first-search (DFS) of $\nset$ is applied from its root node $r$ (\Cref{fig3}c).

In the first DFS, we calculate for the \textbf{node parity} $n_p$ --- the number of descendant syndrome-nodes of a node modulo 2 --- via a tail-recursive function, which is only dependent on the node parities of the children nodes of a node. The node parity is defined per node type:
\begin{align}\label{eq:nodeparity}
    s_p &= \hspace{.6cm}\big( \sum_{\mathclap{n \in \text{ children of } s}} (1-n_p) \big ) \bmod 2,\\
    j_p &= 1 - \big(\sum_{\mathclap{n \in \text{ children of } j}} (1-n_p) \big) \bmod 2.
\end{align}

In the second DFS, we calculate for the difference in node suspension of a node $n$ with its parent $m$; $\delta = \m{s}(n) - \m{s}(m)$. We can choose an arbitrary \textbf{node delay} $n_d$ --- the node suspension minus the maximum node suspension in the node-tree --- for the root node $r$ such as $r_d=0$ and add the suspension difference $\delta$ during each step to obtain $n_d$ for every node. This node delay of a node $n$ is only dependent on the node radii of itself and its parent $m$, the length of edge $(n,m)$, and its parity $n_p$. 
\begin{multline}\label{eq:delayequation}
    n_d = m_d + \bigg \lceil 2C\big(\ceil{n_r} - \floor{m_r + n_r \bmod 1}\\
    - (-1)^{n_p}\abs{(n,m)}\big) - 2(n_r - m_r) \bmod2 \bigg \rceil
\end{multline}
Here, the \textbf{inversion constant} $C$ deals with the inversion of node parities in a node-tree during merges of clusters explained in \ref{sec:nodejoin}. The node suspension is then related to the node delay by
\begin{equation*}
    \m{s}(n) = n_d - \max_{x \in \nset}{x_d}. 
\end{equation*}
The maximum node delay in $\nset$ can be maintained during the second DFS of the node-tree, and the node suspension itself is calculated during cluster growth. A single growth iteration, which is applied in the UF decoder by adding half-edges to all boundary vertices of the cluster, is now replaced by another DFS of $\nset$. During this DFS, we calculate the suspension $\m{s}(n)$ for a node, and conditionally grow it --- adding half-edges to the boundary vertices in the current node and adding 1 to its radius $n_r$ --- if $\m{s}(n) = 0$. This requires us to save the list of boundary vertices to each node (\Cref{fig3}c). When all $\m{s}(n)$ in $\nset$ are zero, all nodes are grown simultaneously within the same iteration. 

If the node-tree does not change after a growth iteration, which is the case if no mergers occur between clusters, the node suspensions decrease in an expected manner: For all nodes that are not suspended from growth, their node suspensions decrease with 1 in the next growth iteration. Due to this behavior, we can reuse the node delays $n_d$ to calculate $\m{s}(n)$ for the next growth iteration by introducing another node parameter $n_w$, the number of iterations a node has \textbf{waited}. Each time a node is suspended from growth, we add 1 to $n_w$. The node suspension in subsequent iterations is then
\begin{equation}\label{eq:suspension}
    \m{s}(n) = n_d - \max_{x \in \nset}{x_d} - n_w. 
\end{equation}
Note that we have not stated which node in $\nset$ should be the root node. In fact, any node in $\nset$ could have been picked as the root of the node-tree. The only requirement is that the DFS of cluster growth must be performed along the same direction as the DFSs of the parity and delay calculations. If no cluster mergers occur, the node delays can be reused in the node suspension calculation prior to node growth. The node-tree is constructed by storing all neighbors of a node to a list. This way, the DFSs' direction can be determined by simply saving the root node, the starting point of the DFSs, to the cluster. All node variables are depicted in \Cref{fig3}c. 
%In the next section, we expand upon this idea of "reusing" some intermediate parameters to calculate the node suspensions after a cluster merger.  


\subsection{Joining node-trees}\label{sec:nodejoin}

In the Union-Find (UF) algorithm, odd parity clusters of an odd number of non-trivial vertices, which are elements of $\sigma$, grow in size repeatedly and merge with other clusters until all clusters are even. During these mergers, the node-trees of the Node-Suspension data structure must also be combined. Let us now first make a clear distinction between the merging protocols of the underlying data structures; the clusters-trees of the UF data structure are merged with the $\Union$ function, whereas the node-trees are merged with a separate $\Nodejoin$ function. After a join of multiple node-trees, the node suspensions within the combined node-tree change. Therefore, the $\Nodejoin$ protocol's focus is to minimize the DFSs of the recalculation of the node parity and delays in the combined node-tree. 

First, note that as a cluster of even parity has an even number of non-trivial vertices, its node-tree has an even number of syndrome-nodes. For these even node-trees, the concept of PMW does not exist, as the matching can be made within the node-tree. Consequently, node suspension, parity, and delays are undefined when two odd node-trees join to an even node-tree. 
%Thus, if two odd clusters merge into an even cluster, we don't know and do not care about its node suspensions. 

\Figure[hbt](topskip=0pt, botskip=0pt, midskip=0pt){figures/tikz-figure4.pdf}{
    \emph{(a)} An odd cluster $\nset_o=\{n_1, n_2, n_o\}$ with root $n_1$ joins with an even cluster $\nset_e=\{n_3, n_e\}$ with root $n_3$ on nodes $n_o, n_e$, respectively, to a joined node-tree. If we choose to \emph{(b)}, make $n_e$ a child of $n_o$, the parities and delays the subtree of $\nset_o$ can unchanged, and we only have to perform partial parity and delay calculations over the subtree of $\nset_e$. If we choose to \emph{(c)}, make $n_o$ a child of $n_e$, parities and delays have to be recalculated in the entire joined node-tree. \label{fig4}}

The second type of merger is between an even and an odd cluster. The combined cluster is odd, and its growth is continued. Thus, its node suspensions must be computed. Consider the example of odd node-tree $\nset_o$ and even node-tree $\nset_e$ that are to be joined on nodes $n_o\in \nset_o$ and $n_e \in \nset_e$ (\Cref{fig4}\emph{a}). If $\nset_o$'s root is kept as the root of the joined node-tree (\Cref{fig4}\emph{b}), $n_e$ is to be a child node of $n_o$. As $\nset_e$ contains an even number of syndrome-nodes, the node parities in $\nset_o$ do not change. Hence, the node parity DFS is only necessary in the subtree $\nset_e$, which now has $n_e$ as subroot. Furthermore, as the node delay is only dependent on its own properties and its parent's, the node delay DFS is also only required from node $n_e$ and within the subtree of $\nset_e$. These so-called \textbf{partial} DFSs of the node-tree are precisely what was required, as the node parity and delays in $\nset_e$ were undefined. Alternatively, if $\nset_e$'s root becomes the root of the combined tree (\Cref{fig4}\emph{c}), an odd number of syndrome-nodes are attached to $n_e$, such that the parities of nodes on the path from $n_e$ to the root are changed. Such a join would require the DFSs on the entire combined node-tree to calculate for node parities and delays. Thus, a simple rule is always to keep the root of the odd node-tree, which we dub \textbf{Odd-Rooted Join}.

In addition, a cluster can be subjected to multiple mergers within the same growth iteration, during which the parity of the merged cluster changes dependent on the number of mergers and the parities of the clusters involved. The DFSs related to the parity and delay calculations must, for this reason, not be initiated directly after the joining of node-trees. After all, it may be possible for the cluster to merge again such that the parities and delays become invalid. To prevent these redundant calculations, subroots of the even subtrees are stored to a list $\m{R}$ at the root of the node-tree (\Cref{fig3}\emph{c}). When multiple mergers occur, the root node that stores the now redundant subroots is replaced by a new root with new $\m{R}$. If a cluster is selected for growth, we check for the subroots in $\m{R}$ at the new root node and initiate the DFSs from these subroots. We call this the \textbf{Root List Replacement}. 

\Figure[htb](topskip=0pt, botskip=0pt, midskip=0pt){figures/tikz-figure5.pdf}{
    The node suspension values for nodes for 3 odd node-trees $\{\nset_1, \nset_2, \nset_3\}$ of 3 nodes that grow and join into a single node-tree. \emph{(a)} Node suspensions are calculated by setting $C=1$ in \Cref{eq:delayequation}. In step 1, the growth in each of the three node-trees' outer nodes is prioritized, and the node-trees merge. In step 2, the recalculation of the joined node-tree is performed. Parities within the subtree of $\nset_2$ are now inverted, and the suspension in these nodes have doubled. \emph{(b)} Node suspensions are calculated by setting $C=\nicefrac{1}{2}$. Now the increase in node suspensions after parity inversion is halved.\label{fig5}}

\subsection{Parity Inversion}\label{sec:inversion}
An unfortunate effect of the Node-Suspension data structure, which we dub \textbf{Parity Inversion}, causes a decrease in the algorithm's performance as the lattice size is increased. We will demonstrate this effect through the example in \Cref{fig5}\emph{a}. Consider three instances of the node-tree of \Cref{fig0}; $\nset_a, \nset_b, \nset_c$, positioned near each other on the lattice. For each node-tree, if the middle node is suspended from growth for two iterations, all nodes have the same Potential Matching Weight. However, in the current example, the node-trees $\nset_a, \nset_b, \nset_c$ merge after 1 iteration. The combined node-tree is odd. Thus, we recalculate the node parities and delays to find that the parities in the partition of the node-tree containing the nodes of $\nset_b$ have been inverted, and the node suspensions in this partition have doubled from before node suspensions before the merger. If the next merging event occurs on the node with the doubled node suspension, the matching weight may be larger compared to the original UF decoder, which defies the goal of Node-Suspension to decrease the matching weight.

Parity Inversion defines a trade-off in the Node-Suspension data structure: `A node must wait as many iterations as it is suspended to reach equilibrium in Potential Matching Weight in the node-tree. However, after Parity Inversion, the node suspension for previously prioritized nodes increases linearly with the number of iterations waited by the suspended nodes pre-inversion.' As a compromise, we redefine the node suspension as \textbf{half} the number of growth iterations needed for all nodes in the node-tree to reach equal PMW. This can be done by setting $C=0.5$ in \Cref{eq:delayequation}. Nevertheless, as more inversions occur, the maximum node suspension in the node-tree increases, and it becomes more and more unlikely for a cluster to actually reach zero node suspension in all nodes. The number of inversions is directly related to the number of merging events, and thus to the size of the lattice. The performance to improve the heuristic for minimum weight matching thus decreases for larger lattices. 


    \subsection{Pseudocode}\label{sec:pseudocode}
\begin{algorithm}[htb]
  \BlankLine
  \KwData{A graph $G=(V,E)$, and syndrome $\sigma \subseteq V$}
  \KwResult{Correction set $\m{C}$}
  \BlankLine
  Initialize cluster vertex-trees, node-trees and table $\Support$.\;\label{algo:B1a}
  Create the list $\m{L}$ of odd clusters.\;
  \While(){$\m{L}$ is not empty}{
    Initialize the fusion list $\m{F}$ as an empty list.\;\label{algo:B1b}
    \For(){cluster $\in\m{L}$ \label{algo:B2a}}{
      \For(){$n \in\m{R}$ at node root $r$}{
        Apply DFSs to calculate node parities and delays (Equations \eqref{eq:nodeparity}, \eqref{eq:delayequation}) from $n$ to descendant nodes. Keep track of the largest value for $n_d$ encountered during the delay DFS.\;\label{algo:pdc}
      }
      Apply DFS from root $r$ to all descendants. At each node during the DFS, if $\m{s}(n)=0$ (\Cref{eq:suspension}), grow all boundary edges of vertices in the node a half-edge per the Union-Find decoder, such that grown edges are added to $\m{F}$, and apply $n_r=n_r+\nicefrac{1}{2}$. If $\m{s}(n)\neq0$, apply $n_w=n_w+1$ and continue the DFS.\;\label{algo:grow}
    }
    \For(){edge $(u,v) \in \m{F}$\label{algo:B3a}}{
      \eIf(){$\Find(u)\neq\Find(v)$}{
        Merge vertex-trees by Weighted $\Union$.\;
        \eIf(){$u \in n_u$ and $v \in n_v$\label{algo:joina}}{
          Merge node-trees by Odd-Rooted $\Nodejoin$. If resulting node-tree with root $r$ is odd, add even subroot (either $n_u$ or $n_v$) to list $\m{R}$ at $r$.\;
        }($u$ or $v$ does not belong to a cluster){
          Add $u$ to $n_v$ or $v$ to $n_u$.\;\label{algo:joinb}
        }
      }($u,v$ in same cluster.\label{algo:dfa}){
        Subtract 1 from $(u,v)$ in $\Support$.\;\label{algo:dfb}
      }
    }
    Update $\m{L}$ with odd clusters\; \label{algo:B3b}
  }
  Apply the peeling decoder \cite{delfosse2017linear}.\label{algo:B4a}
  \caption{Union-Find Node-Suspension decoder}\label{algo:ufbb}
\end{algorithm}

\Figure[hbt](topskip=0pt, botskip=0pt, midskip=0pt){figures/tikz-figure6.pdf}{
  The \emph{fragmentation} of node-tree $\pr{k-1}\omega$ into its ancestor node-trees, where the prefix $k-1$ indicates the \emph{fragmentation generation}. The fragmentation $f(\pr{k-1}\omega)$ returns $\nu+1$ odd ancestors of generation $k$. Fragmentation function $f$ can be separated into the \emph{partial fragmentation} $f_\omega(\pr{k-1}\omega)$ that returns an odd ancestor $\pr{k}\omega_0$ and even ancestor $\pr{k}\epsilon$, and subsequently the partial fragmentation $f_\epsilon(\pr{k}\epsilon)$, which returns $\nu$ odd ancestors of generation $k$. With $\nu=2$, each partial fragmentation is the opposite of a join operation between two node-trees. Furthermore, note that if $\abs{\vset_0} = \abs{\vset_1} = \abs{\vset_2}$, the vertex-tree sizes related to node trees $\pr{k}\omega_0, \pr{k}\omega_1, \pr{k}\omega_2$, each generation is the opposite of a growth iteration. \label{fig6}}

The full version of the algorithm we have described is given in Algorithm \ref{algo:ufbb}. Note that this pseudocode includes instructions that are shortened versions of the pseudocode of the Union-Find decoder \cite{delfosse2017almost}. This is done for clarity on the modified additions of the Node-Suspension data structure and protocols on top of the Union-Find pseudocode. The first block of lines \ref{algo:B1a}-\ref{algo:B1b} initializes the clusters and describes the loop of cluster growth. Block 2 contains lines \ref{algo:B2a}-\ref{algo:grow} and describes the DFSs related to calculating the node parities and delays from all even subroots stored at the root node, and the DFS of the cluster growth. Block 3 contains lines \ref{algo:B3a}-\ref{algo:B3b} and describes the combined merging protocols of the Union-Find and Node-Suspension data structures. Note that lines \ref{algo:dfa}-\ref{algo:dfb} contain an extra step that maintains acyclic vertex-trees. The final block in line \ref{algo:B4a} is the peeling decoder \cite{delfosse2017linear}, which now does not have to create the spanning forest of the grown clusters. Similarly to the Union-Find decoder, Weighted Growth is applied such that the smaller cluster is always grown first. 

    
\section{Complexity of Node-Suspension}\label{sec:complexity}

In this section, we will find the worst-case time complexity of the Union-Find Node-Suspension decoder. The additional cost of the original Union-Find decoder can be split in two parts: (A) the depth-first-searches (DFSs) related to the (re)calculation of the node parities and node delays in line \ref{algo:pdc}, and (B) the DFS related to the growth of a cluster in line \ref{algo:grow}. We dub the two parts the \textbf{suspension cost} and the \textbf{growth cost}, respectively. The $\Nodejoin$ operation in lines \ref{algo:joina}-\ref{algo:joinb} only has a linear addition to the cost.

\subsection{Suspension cost}\label{sec:suscomplexity}

The cost of node suspension calculation is proportional to $N_{\text{sus}}$, the number of nodes traversed in the DFSs of the node parities and node delays. As a result of Odd-Rooted Join, $N_{\text{sus}}$ is proportional to the sum of sizes of all even node-trees. Root List Replacement decreases the sum to the most recent even node-trees that are ancestors of grown odd node-trees, which we dub $\mathbf{\Delta}$ \textbf{node-trees}. To find the worst-case time complexity, we maximize $N_{\text{sus}}$, which is proportional to the computation time. We take a time-reversed approach of analyzing a cluster; starting from a single cluster that maximally occupies the lattice at the end of growth, and move back in time to find its ancestor clusters. In this process that we call \textbf{cluster fragmentation}, we aim to find the combination of cluster mergers that maximizes the number and sizes of $\Delta$ node-trees. 

% The maximization of $N_{\text{sus}}$ is in the repetitiveness of the recalculation over some parition of the final node-tree. 


% \begin{enumerate}[label=P\arabic*,ref=P\arabic*]
%   \item Joins between the node-trees of odd and even clusters retain the root node of the odd cluster. \label{rjoin}
%   \item The DFSs of the node parity and delay calculation are performed just before cluster growth in an even partition of the node-tree starting from the pointer saved at the root node. \label{rgrow}
%   \item The smallest clusters, measured by the number of verties in $\vset$, are grown first. \label{rweight}
% \end{enumerate}

% From properties \ref{rjoin} and \ref{rgrow}, it can be $N_{\text{sus}}$ is proportional to the sum of sizes of all even node-trees during all mergers of the growth process on the lattice. Note that if many clusters merge within the same growth iteration, only the last even cluster counts towards the cost, since \ref{rgrow} ensures that the calculation is not performed on intermediate even partitions.

\begin{definition}\label{def:fragmentation}
  Let the \textbf{fragmentation} function $F$ on an odd node-tree $\omega$ return $\nu + 1$ of its most recent odd ancestor node-trees on which suspension calculations were performed. Let the prefix on the node-tree indicate the \textbf{fragmentation generation}, such that $F(\pr{k-1}\omega)$ returns $\nu + 1$ node-trees of generation $k$:
  \begin{equation}\label{eq:fstep}
    F(\{\pr{k-1}\omega\}) = \{\pr{k}\omega_0,\pr{k}\omega_1,...,\pr{k}\omega_{\nu}\}.
  \end{equation}
  Let $F$ be the combination of \textbf{partial fragmentation} functions $F_\omega$ and $F_\epsilon$, where $F_\omega$ fragments an odd node-tree $\pr{k-1}\omega$ into an odd ancestor $\pr{k}\omega_0$ and an even ancestor node-tree $\pr{k}\epsilon$: 
  \begin{equation}\label{eq:pfo}
    F_\omega(\{\pr{k-1}\omega\}) = \{\pr{k}\epsilon, \pr{k}\omega_0 \},
  \end{equation}
  and where $F_\epsilon$ further fragments $\pr{k}\epsilon$ into $\nu$ odd ancestors
  \begin{equation}\label{eq:pfe}
    F_\epsilon(\{\pr{k}\epsilon\}) =\{\pr{k}\omega_1,...,\pr{k}\omega_{\nu}\}.
  \end{equation}
  % of an odd cluster with node-tree $\pr{k-1}\omega$ split it into a set of its ancester node-trees. Here $k-1$ indicates the \emph{fragmentation step number}, where larger step number $k$ refers to an ancestor node-tree of smaller size. Let the fragmentation $F$ be the combination of \emph{intermediate fragmentations} (IF) $F_\omega$, which fragments an odd node-tree into an even ancestor and an odd ancestor

  % and $F_\epsilon$, which fragments even node-trees into $\nu$ odd ancestors
  % \begin{equation}\label{eq:pfe}
  %   f_\epsilon(\{\pr{k}\epsilon\}) = \m{F}^e_k=\{\pr{k}\omega_1,...,\pr{k}\omega_{\nu}\},
  % \end{equation}
  % such that a fragmentation is
\end{definition}

The fragmentation function can be applied consecutively, such that a set of odd node-trees of generation $k$ is fragmented to ancestors of generation $k+1$. We use the notation $F^{(2)}(\pr{k-1}\omega)$ to indicate that two fragmentations are applied on $\pr{k-1}\omega$ to obtain the ancestors of generation $k+1$. Furthermore, let $F_\omega^{(i)}$ and be equivalent to $F_\omega F^{(i-1)}$. 

As a result of Odd-Rooted Join and Root List Replacement, if the cluster of node-tree $\pr{k-1}\omega$ is grown, the DFSs of the parity and delay calculations are performed within $\pr{k}\epsilon$. Therefore, every even node-tree returned by $F_\omega$ is a $\Delta$ node-tree. The value of $N_{\text{sus}}$ can thus be obtained by taking the sum of sizes of all even node-tree in $F_\omega^{(k)}$ over a series of $\mu$ fragmentations of some odd node-tree $\Omega = \pr{0}\omega$ until there are no more ancestor node-trees:
\begin{equation}\label{eq:npdc}
  N_{\text{sus}} = 2\sum_{k=1}^\mu{ \sum_{ \pr{k}\epsilon \in f_\omega^{(k)}(\Omega) }{ \abs{\pr{k}\epsilon}} }.
\end{equation}

To find $N_{\text{sus}}$, we will make two assumptions to simplify \Cref{eq:npdc}. Junction-nodes are initiated on the tangent of two node radii belonging to separate node-trees when merging into one. For increasing fragmentation generation, the total number of nodes in the fragmented set must therefore decrease. By neglecting their existence, \Cref{eq:npdc} becomes
\begin{equation}\label{eq:npdc2}
  N_{\text{sus}} \leq 2\sum_{k=1}^\mu{ \sum_{ \pr{k}\epsilon \in f_\omega^{(k)}(\Omega) }{ \abs{\pr{k}\epsilon}} }.
\end{equation}
Furthermore, if only syndrome-nodes exist, $\pr{k-1}\omega$'s size must equal the sum of sizes of its ancestors $F(\pr{k-1}\omega)$. In other words, the size of each of the $\nu+1$ ancestors $\pr{k}\omega_i$ can be represented by the \textbf{fragmentation ratio}
\begin{equation}\label{eq:ratio}
  \pr{k}R_i = \frac{\abs{\pr{k}\omega_i}}{\abs{\pr{k-1}\omega}}, \hspace{0.5cm} \sum_{i=0}^{\nu}{\pr{k}R_i} = 1,
\end{equation}
Secondly, we assume that vertex-trees do not increase in size, such that $\abs{\nset}=\abs{\vset}$. Normally, the number of nodes in a cluster is bounded by the number of vertices $\abs{\nset}\leq \abs{\vset}$, as non-trivial vertices can be added to the node, which increases the node radius. By this assumption, the vertex-tree can only increase in size due to a merger between clusters, and nodes are effectively not allowed to increase in radius. While this is not possible during realistic cluster growth, using this assumption simplifies \Cref{eq:npdc2}, as we will see later, while not compromising its upper bound. 

To find the upper bound in $N_{\text{sus}}$, we are now tasked to find: (a) $\nu$, the number of ancestors in $F_\epsilon$, (b) the fragmentation ratios $\{R_0, ..., R_\nu\}$, and (c) the number of fragmentation generations $\mu$. 

\begin{lemma}\label{lem:evenconstant}
  For constant fragmentation ratios $\pr{k}R_i = R_i$, the sum $\sum_{ \pr{k}\epsilon \in f_\omega^{(k)}(\Omega) }{ \abs{\pr{k}\epsilon}}$ is constant for every $k$. 
\end{lemma}
\begin{proof}
  For $k=1$, there is a single even ancestor $\pr{1}\epsilon$ of size 
  \begin{equation*}
    \abs{\pr{1}\epsilon} = (1 - R_0)\abs{\Omega}.
  \end{equation*}
  For $k=2$, every odd node-tree in $F(\Omega)=\{\pr{1}\omega_0,...,\pr{1}\omega_\nu\}$ is fragmented by $F_\omega$ to an even ancestor $\pr{2}\epsilon_i$ for $i \in \{0,...,\nu \}$, such that 
  \begin{equation*}
    \sum_{i=0}^{\nu}{\abs{\pr{2}\epsilon_i}}  = \sum_{i=0}^{\nu}{R_i(1 - R_0)\abs{\Omega}}= (1 - R_0)\abs{\Omega}.
  \end{equation*}
  The same is true for all subsequent generations $k$. 
\end{proof}

\begin{theorem}\label{the:fragnumber}
  The upper bound for $N_{\text{sus}}$ is obtained by setting $\nu=2$ in \Cref{eq:pfe}. 
\end{theorem}
\begin{proof}
  The sum of even node-tree sizes in every generation is constant per \Cref{lem:evenconstant}. Thus, the upper bound in \Cref{eq:npdc2} is obtained by the largest possible $\mu$. As $\nu$ increases the number of odd node-trees in each $F^{(k)}_\omega$, the average size of these odd node-trees decreases. Since the size of a node-tree is proportional to the number of ancestor generations, we find that 
  \begin{equation*}
    \mu \propto \frac{1}{\nu}. 
  \end{equation*}
  Hence, the upper bound in \Cref{eq:npdc2} exists in the minimal value of $\nu$, which is $\nu = 2$.
\end{proof}

Using \Cref{the:fragnumber}, we now find that a fragmentation on an odd cluster $F(\pr{k-1}\omega)$ returns $\{\pr{k}\omega_0, \pr{k}\omega_1, \pr{k}\omega_2\}$, where $\pr{k}\omega_1, \pr{k}\omega_2$ are ancestors of the even node-tree $\pr{k}\epsilon$ returned by $F_\omega(\pr{k-1}\omega)$. 

\begin{lemma}\label{lem:chrono}
  The node-tree size of $\pr{k}\omega_0$ must be smaller than $\pr{k}\omega_1, \pr{k}\omega_2$, such that $R_1 \geq R_0 \leq R_2$. 
\end{lemma}
\begin{proof}
  The partial fragmentations must occur in the order of first \Cref{eq:pfo}, then \eqref{eq:pfe}, as \eqref{eq:pfe} requires an even node-tree that is returned by \eqref{eq:pfo}. In terms of cluster growth, the vertex-trees $\vset_1, \vset_2$, corresponding to $\pr{k}\omega_1, \pr{k}\omega_2$, must merge before the combined vertex-tree can merge with $\vset_0$, which corresponds to $\pr{k}\nset_0$. As a result of Weighted Growth, $\abs{\vset_1}$ and $\abs{\vset_2}$ must be smaller or equal to $\abs{\vset_0}$, such that 
  \begin{equation*}
    \abs{\vset_1}\geq \vset_0 \leq\abs{\vset_2}.
  \end{equation*}
  If this condition is not met, the cluster of $\vset_0$ grows first and merges with either $\vset_1$ or $\vset_2$, and the chronology of events is disturbed. Since we assumed $\abs{\nset}=\abs{\vset}$, this can be translated to 
  \begin{equation*}
    \abs{\nset_1}\geq \nset_0 \leq\abs{\nset_2},
  \end{equation*}
  and subsequently to the fragmentation ratios.
\end{proof}

\begin{theorem}\label{the:ratios}
  The upper bound for $N_{\text{sus}}$ is obtained via the fragmentation ratios $R_0 = R_1 = R_2 = \nicefrac{1}{3}$.
\end{theorem}
\begin{proof}
  The ratios $\{R_0, R_1, R_2\}$ can be found by maximizing the size of the even node-tree $\pr{k}\epsilon$ in each fragmentation, which is 
  \begin{equation*}
    \abs{\pr{k}\epsilon} = (R_1 + R_2)\abs{\pr{k-1}\omega}.
  \end{equation*}
  Since $ R_1 \geq R_0 \leq R_2$ per \Cref{lem:chrono}, the largest values for $R_1, R_2$ possible are equal to $R_0$.
\end{proof}

The last unknown parameter in finding the upper bound of $N_{\text{sus}}$ in \Cref{eq:npdc2} is $\mu$.

\begin{theorem}\label{the:km}
  For $\nu = 2$ and $R_i = \{\nicefrac{1}{3},\nicefrac{1}{3},\nicefrac{1}{3}\}$, the maximum number of fragmentation generations is $\mu = \log_3{\abs{\Omega}}$.
\end{theorem}
\begin{proof}
  In every generation, all node-trees are fragmented into 3 ancestors that are $\nicefrac{1}{3}$ the size of their common descendant. The series of $\mu$ fragmentations is thus simply $\mu$ divisions of the node-tree $\Omega$ in 3 parts until all ancestors have size 1, at which point a node-tree cannot be fragmented.
\end{proof}

Collecting \Cref{the:fragnumber,the:ratios,the:km} and filling in \Cref{eq:npdc2}, we find that

\begin{align*}
  \nonumber N_{\text{sus}} &\leq 2\sum_{k=1}^\mu{ \sum_{ \pr{k}\epsilon \in f_\omega^{(k)}(\Omega) }{ \abs{\pr{k}\epsilon}}  } \\
  \nonumber         &\leq 2\sum_{k=1}^{\log_3{\abs{\Omega}}} \frac{2}{3}\abs{\Omega}\\
                    &\leq \frac{4}{3}\abs{\Omega}\log_3{\abs{\Omega}}.
\end{align*}

The maximum size of the odd node-tree $\Omega$ is bounded by the system size $n = \abs{V}$. The worst-case time complexity of the suspension calculation is thus $\m{O}(n\log{n})$. 

\subsection{Growth cost}\label{sec:growthcost}

To grow a cluster represented by a node-tree $\nset$, a depth-first search (DFS) is performed on the node-tree to find all nodes with zero suspension. The total cost of these DFSs is proportional to the total number of nodes encountered during these DFSs, which we dub $N_{\text{grow}}$. Using the definition of fragmentations of \Cref{def:fragmentation}, the cost of growth is proportional to
\begin{equation}\label{eq:ngrow}
  N_{\text{grow}} = 2\sum_{k=1}^\mu{ \sum_{ \pr{k}\omega \in f^{(k)}(\Omega) }{ \abs{\pr{k}\omega}} }.
\end{equation}
Again, we assume that no trivial vertices are added to a cluster or $|\nset| = |\vset|$ such that \Cref{eq:ngrow} becomes an upper bound. As a result of Odd-Rooted Join and Root List Replacement, the upper bound is obtained if there are as many fragmentation generations. This is again achieved through $\nu = 2$. For every fragmentation of some odd node-tree $\pr{k-1}\omega$ into $\{\pr{k}\omega_0, \pr{k}\omega_1, \pr{k}\omega_2\}$, all three ancestors add to $N_{\text{grow}}$ if they have grown. As a result of Weighted Growth, this is the case when $\abs{\vset_0}\approx \abs{\vset_1}\approx\abs{\vset_2}$ such that $R_0 \approx R_1 \approx R_2\approx \nicefrac{1}{3}$. For these values of $\nu$ and $R$, we can apply \Cref{the:km} for $\mu$. For $|\nset| = |\vset|$, the sum of node-tree sizes in every fragmented set $\m{F}_k$ is exactly $\abs{\Omega}$, and we find that
\begin{align*}
  \nonumber N_{\text{grow}} &\leq 2\sum_{k=1}^\mu{ \sum_{ \pr{k}\omega \in f^{(k)}(\Omega) }{ \abs{\pr{k}\omega}}  } \\
  \nonumber         &\leq 2\sum_{k=1}^{\log_3{\abs{\Omega}}} \abs{\Omega}\\
                    &\leq 2\abs{\Omega}\log_3{\abs{\Omega}},
\end{align*}
which again corresponds to a worst-case time complexity $\m{O}(n\log{n})$.
    \section{Performance}\label{sec:performance}

We benchmark the performance of our own variants of the Union-Find decoder, as well as the Union-Find Node-Suspension (UFNS) decoder of Algorithm \ref{algo:ufbb}. Decoding rate $d$ for a given lattice size $L$ and physical error rate is acquired by Monte Carlo simulations. We compare the performance under the independent noise model of only i.i.d. bit-flip errors with chance $p_X$ and the phenomenological noise model, which adds faulty syndrome measurements occurring at chance $p_X$. Code thresholds $p_{th}$ are obtained by curve fitting for the crossing point of decoding rates plotted in $(p_X, d)$ space, for a range of values for $L$ and $p_X$ \cite{wang2003confinement}. We also use the decoding rate at the threshold $d(p_{th})= d_{th}$ as a metric to compare decoders. 

\subsection{Union-Find decoder variants}

First, we show that $p_{th}$ can be increased, and the matching weight $\abs{\m{C}}$ can be decreased within the Union-Find decoder \textbf{without} the Node-Suspension data structure. We compare distinct variants of our implementation of the Union-Find decoder, that either implements Weighted Growth via bucket sort or no Weighted Growth, and either constructs a forest $F$ of grown clusters post-growth, which is the case in the original Union-Find decoder, or maintains acyclic vertex-trees $\vset$ during cluster growth. The latter is achieved by applying lines \ref{algo:dfa}-\ref{algo:dfb} of Algorithm \ref{algo:ufbb} in the UF decoder. The labels used for each variant are listed in \Cref{tab:uftable}. The full descriptions for each of the variants can be found in \cite{markthesis}.

\begin{table}[htbp]
  \centering
  \begin{tabularx}{\linewidth} { | R{2} || C{.5} | C{.5} | }
    \hline
    & $\mathbf{F}$ &  $\pmb{\vset}$\\
    \hhline{|=::=:=|}
    No Weighted Growth & fUF  & vUF \\
    \hline
    \textbf{B}ucket sort Weighted Growth & bfUF & bvUF \\
    \hline
  \end{tabularx}
  \caption{Abbreviated names for the variants of the Union-Find decoder.}\label{tab:uftable}
\end{table}

\Figure[b!](topskip=0pt, botskip=0pt, midskip=0pt){figures/comp_matching_weight.pdf}{
  The matching weight $\abs{\m{C}}$ of the Union-Find decoder variants (Table \ref{tab:uftable}) and the UFNS decoder, normalized to the minimum weight $\min{\abs{\m{C}}}$ of the Minimum-Weight Perfect Matching decoder. All weights are obtained by Monte Carlo simulations on $p_X=0.098$ with a minimum of $100.000$ samples. The x-axis scales linearly with $N = L^2$. \label{comp_weight}}

\Figure[htb](topskip=0pt, botskip=0pt, midskip=0pt){figures/threshold_ufbb.pdf}{
  The decoding rates $d$ of \emph{(a)} the UFNS decoder and \emph{(c)} the bvUF and MWPM decoders, obtained via Monte Carlo simulations with a minimum of 100.000 samples per lattice size per error rate. \emph{(a)} The decoding rates of the UFNS decoder do not cross in a single point, such that there is no apparent code threshold. \emph{(b)} The intersections of threshold curves of subsequent lattice sizes are the so-called threshold coordinates, which follow a trend in $(p_X, d)$ space as the lattice sizes increase. \emph{(c)} The threshold coordinates of the UFNS decoder occupy the region in $(p_X, d)$ space of the MWPM decoder. \label{threshold_ufbb}}

The values for $p_{th}$ and $d_{th}$ for each variant, including the Minimum-Weight Perfect Matching (MWPM) decoder, are listed in \Cref{tab:ufndfwug}. Weighted Growth has the expected behavior of increasing $p_{th}$. While there is no major increase in $p_{th}$ from the $\vset$-variants over the $F$-variants, a significant increase in $d_{th}$ can be observed. We suspect that the acyclic graphs of $\vset$ have shorter branches in between junctions compared to $F$, which leads to a decreased matching weight and increased $d_{th}$. We plot the matching weight $\abs{\m{C}}$ of the UF variants, normalized to the minimum weight of the MWPM decoder for $p_X = 0.098$ in \Cref{comp_weight}. Here we see a correlation between a decrease in $\abs{\m{C}}$ and increase in performance: both Weighted Growth and maintaining $\vset$ during growth increase performance and decrease matching weight. Furthermore, the matching weight in $\vset$-variants have a relatively low and constant factor over the minimum weight, which improves upon the $L$-dependent behavior of $F$-variants.

\begin{table}[htbp]
  \centering
  \begin{tabularx}{\linewidth} { | R{1} || C{1} | C{1} | C{1} | C{1} | }
    \hline
    \multirow{2}{*}{} & \multicolumn{2}{c|}{Independent}& \multicolumn{2}{c|}{Phenomenological} \\
    \cline{2-5}
     & $p_{th}$ & $d_{th}$ & $p_{th}$ & $d_{th}$ \\
    \hhline{|=::=:=:=:=|}
    fUF & $9.72\%$ & $73.34\%$ & $ 2.53\%$ & $92.39\%$ \\
    \hline
    vUF & $9.79\%$ & $74.32\%$ & $2.56\%$ & $93.64\%$ \\
    \hline
    bfUF & $9.98\%$ & $72.71\%$ & $2.68\%$ & $91.32\%$ \\
    \hline
    bvUF & $10.01\%$ & $72.86\%$ & $2.69\%$ & $92.08\%$ \\
    \hline
    MWPM & $10.35\%$ & $71.58\%$ & $2.97\%$ & $90.24\%$\\
    \hline
  \end{tabularx}
  \caption{Threshold error rates $p_{th}$ and threshold decoding success rates $d_{th}$ for the implementations of the  Union-Find decoder of \Cref{tab:uftable}.}\label{tab:ufndfwug}
\end{table}

\subsection{Union-Find Node-Suspension decoder}
We benchmark the performance of the Union-Find Node-Suspension (UFNS) decoder of Algorithm \ref{algo:ufbb}. The decoding rates are plotted in \Cref{threshold_ufbb}\emph{a} per lattice size. We discover that the curves related to the decoding rates do not cross in a single point, such that there is no clear threshold $p_{th}$. In fact, the intersections of two curves of subsequent lattice sizes, which we dub threshold coordinates, follow a trend where larger input lattice sizes results in a decrease in $p_{th}$ but an increase in $d_{th}$. We ascribe the degradation of the threshold coordinate for larger lattices to the Parity Inversion effect. These threshold coordinates $(p_{th}, d_{th})$ are plotted in \Cref{threshold_ufbb}\emph{b}. When these coordinates are plotted together with the bvUF and MWPM decoders' decoding rates, such as in \Cref{threshold_ufbb}\emph{c}, we see that the threshold coordinate for small lattice sizes is similar to the MWPM threshold coordinate. For larger lattice sizes, UFNS threshold coordinates move towards the bvUF threshold on the $p_X$ axis, but still has an increased performance due to the increased $d_{th}$. Overall, the threshold coordinates occupy a region in $(p_X, d)$ space previously reserved to the MWPM decoder. 

\Figure[b!](topskip=0pt, botskip=0pt, midskip=0pt){figures/comp_time.pdf}{
  The mean computation time of the UFNS, bvUF, and MWPM decoders near the threshold error rate. All weights are obtained by Monte Carlo simulations for $p_X=0.098$ with a minimum of $100.000$ samples. The x-axis scales linearly with $N = L^2$.\label{comp_time}}

The matching weight $\abs{\m{C}}$ of the UFNS decoder is successfully decreased compared to all UF variants of \Cref{tab:uftable}. For $p_X = 0.098$, an error rate close to $p_{th}$ for all decoders, the normalized matching weight is halved compared to the bvUF decoder (\Cref{comp_weight}). We compare the average running time of Monte Carlo simulations to obtain a matching between the UFNS, bvUF, and MWPM decoders. A comparison for a physical error rate near the threshold is included in \Cref{comp_time}. The average running time of UFNS, while not behaving according to worst-case $\m{O}(n \log{n})$, is also not linear. 

Finally, we show in \Cref{comp_lowerror} the performance of the UFNS decoder in the low-error regime with phenomenological noise. The decoding rate of the UFNS decoder is improved from the bvUF decoder and behaves similarly to the MWPM decoder. The mean computation time of the decoders in this regime, plotted in \Cref{comp_lowerror_time}, shows that the UFNS decoder performs at about the same speed as the bvUF decoder.\par

The simulator for the surface code, the Union-Find decoder variants, and the Union-Find Node-Suspension decoder have all been implemented in Python using our application \cite{OpenSurfaceSim}. The MWPM decoder utilizes the C implementation of BlossomV \cite{kolmogorov2009blossom} due to substantial slow performances of Python implementations. Simulations were initially performed on a single 3.20 GHz Intel Core i5 CPU but later parallelized on all 24 threads of 3.60 GHz Intel Xeon E5 CPU's. 
    \section{Conclusion}\label{sec:conclusion}

In this paper, we have introduced a modification of the Union-Find (UF) decoder \cite{delfosse2017almost} that selectively grows regions of clusters based on the concept of a potential matching weight. The modified decoder, dubbed the Union-Find Node-Suspension (UFNS) decoder, relies on an additional data structure to facilitate the calculation of the potential matching weight. We have proved analytically that the UFNS decoder has a worst-case time complexity of $\m{O}(n\log{n})$. 

Through Monte Carlo simulations on various decoder types, we have found that the UFNS decoder improves upon the performance of the UF decoder for all tested physical error rates and system sizes. Unfortunately, there is no fixed error threshold due to the Parity Inversion effect, which affects the performance at larger lattice sizes. Nevertheless, the UFNS decoder manages to occupy a region in $(p_X, d)$ space previously reserved to the Minimum Weight Perfect Matching (MWPM) decoder. For the low-error regime, the UFNS combines the advantages of the MWPM decoder's high decoding rates and the UF decoder's low computation time. Future work should focus on finding a way around the Parity Inversion effect and testing the decoder for other error types, such as erasure errors. 

Recent work that includes the Union-Find decoder focuses on bringing the decoder algorithm to the hardware level. Most notably, a scalable decoder micro-architecture has been proposed with a fully pipelined hardware implementation \cite{das2020scalable}. Related work has shown that a reduction in bandwidth is possible provided qubits with a low physical error rate \cite{delfosse2020hierarchical}. Furthermore, another variant of the decoder, dubbed the \emph{Weighted Union-Find} decoder, not to be confused with \emph{Weighted Growth}, promises to increase the code threshold under circuit-level noise \cite{huang2020fault}. This application relies on adopting the decoder to a \emph{weighted} graph. Every edge $e\in\m{E}$ may now have a different length value, and edges are not limited to the growth of half-edges per growth iteration. We believe that the Union-Find Node-Suspension decoder and the Weighted Union-Find decoder are compatible. In the combined decoder, boundary edges in every node are grown with respect to their weights in the weighted graph. 

The Union-Find decoder manages to decode fast and scale almost-linearly with the input system size. However, these speed-ups come at the cost of a decreased decoding performance. With the Union-Find Node-Suspension decoder, we manage to find a middle ground between the two objectives; high decoding performance that runs in worst-case quasilinear time. For these reasons, it may be a great candidate for physical applications in the near future.

\Figure[b!](topskip=0pt, botskip=0pt, midskip=0pt){figures/comp_lowerror.pdf}{
  The decoding rate $d$ for the low-error regime of phenomenological noise for the MWPM, UFNS and bvUF decoders. The UFNS decoding rates are improved from the UF variants and are very similar to MWPM. All $d$ are obtained by Monte Carlo simulations with a minimum of $100.000$ samples. The x-axis scales linearly with $N = L^3$.\label{comp_lowerror}}

\Figure[b!](topskip=0pt, botskip=0pt, midskip=0pt){figures/comp_lowerror_time.pdf}{
  The mean computation time of the UFNS, bvUF, and MWPM decoders in the low error regime for phenomenological noise for $p_X = \{0.5\%, 1.2\%, 2\%\}$ of the same simulation as in \Cref{comp_lowerror}. In this regime, the UFNS computation times are very comparable to the bvUF decoder. The x-axis scales linearly with $N = L^3$. \label{comp_lowerror_time}}


\Figure[htb](topskip=0pt, botskip=0pt, midskip=0pt){figures/threshold_comparison.pdf}{
  Direct comparison of the performance of various decoders covered in this thesis. The data of the original Union-Find (UF) decoder is taken from its publication \cite{delfosse2017almost}. Using the same range of lattice sizes and error rates, we simulate and plot the performance of \emph{(1)} our implementation of the Union-Find decoder with Weighted Growth applied via bucket sort and acyclic vertex-trees maintained during growth, the bvUF decoder, \emph{(2)} the Union-Find Node-Suspension decoder (UFNS), and  \emph{(3)} the Minimum-Weight Perfect Matching (MWPM) decoder.\label{thres_comp}}
    \FloatBarrier
    \bibliographystyle{apsrev}
    \bibliography{cit}
    % \printbibliography
\end{document}