\documentclass{ieeeaccess}
\usepackage[OT1]{fontenc} 
\usepackage{amsfonts, amsmath, amssymb, amsthm}
\usepackage{algorithmic}
\usepackage{graphicx}
\usepackage{caption}
\usepackage{placeins}
% \DeclareCaptionFont{ieeeblue}{\color{accessblue}}
% \DeclareCaptionLabelFormat{myformat}{\figcapfont{\textbf{#1}\textbf{#2}}}
% \captionsetup{labelfont={bf,ieeeblue},labelformat=myformat}
\usepackage{setspace}
\usepackage[font={sf,scriptsize,stretch=0.84}, labelfont={bf,color=accessblue}]{caption}

\usepackage{tabularx, hhline, multirow}
\newcolumntype{L}[1]{>{\hsize=#1\hsize\raggedright\arraybackslash}X}%
\newcolumntype{R}[1]{>{\hsize=#1\hsize\raggedleft\arraybackslash}X}%
\newcolumntype{C}[1]{>{\hsize=#1\hsize\centering\arraybackslash}X}%
\newcommand{\gc}{\cellcolor[gray]{0.9}}

\usepackage{enumitem}
\usepackage{textcomp}
\usepackage{nicefrac}
\usepackage{mathtools}
\DeclarePairedDelimiter{\abs}{\lvert}{\rvert}
% \def\BibTeX{{\rm B\kern-.05em{\sc i\kern-.025em b}\kern-.08em
%     T\kern-.1667em\lower.7ex\hbox{E}\kern-.125emX}}
\usepackage[style=ieee, sorting=none]{biblatex}
\addbibresource{citations/*.bib}

% \usepackage[table]{xcolor}
% % Define colors

\newtheorem{definition}{Definition}[section]
\newtheorem{lemma}{Lemma}[section]
\newtheorem{theorem}{Theorem}[section]
\newtheorem{proposition}{Proposition}[section]

% Set new commands
\newcommand{\codeword}[1]{\texttt{\textcolor{MidnightBlue}{#1}}}
%\newcommand{\codefunc}[1]{\texttt{\textcolor{OliveGreen}{#1}}}
\let\oldemptyset\emptyset
\let\emptyset\varnothing
\newcommand{\codefunc}[1]{\texttt{#1}}
\newcommand{\m}[1]{\mathcal{#1}}
\newcommand{\n}[1]{\mathscr{#1}}
\newcommand{\bound}{\mathscr{B}}
\newcommand{\akker}{\mathscr{A}}
\newcommand{\nset}{\mathcal{N}}
\newcommand{\vset}{\mathcal{V}}
\newcommand{\pre}[1]{ {}^{#1} }
\newcommand{\ceil}[1]{{\left \lceil #1 \right \rceil }}
\newcommand{\floor}[1]{{\left \lfloor #1 \right \rfloor }}

% Set algorithm2e package settings
\usepackage[linesnumbered, ruled, vlined]{algorithm2e}
\SetAlgoCaptionLayout{centerline}
\setlength{\algoheightrule}{1pt}
\setlength{\algotitleheightrule}{1pt}
\setlength{\interspacetitleboxruled}{.5em}
\SetStartEndCondition{ }{}{}
\SetKwProg{Fn}{def}{\string:}{}
\SetKw{KwTo}{in}
\SetKwFor{For}{for}{\string:}{}
\SetKwIF{If}{ElseIf}{Else}{if}{ then}{else if}{else}{}
\SetKwFor{While}{while}{ do}{}
\SetInd{0.1em}{0.5em}
\SetAlgoNoEnd\DontPrintSemicolon
\SetAlFnt{\small}


\begin{document}
\history{Date of publication xxxx 00, 0000, date of current version xxxx 00, 0000.}
\doi{10.1109/TQE.2020.DOI}

\title{Quasilinear Time Decoding Algorithm for \\Topological Codes with High Error Threshold}
\author{
    \uppercase{S. Hu}\authorrefmark{1},
    \uppercase{and D. Elkouss\authorrefmark{2}}
}
\address[1]{Department of Physics, Delft University of Technology (email: watermarkhu@outlook.com)}
\address[2]{QuTech, Delft University of Technology, Lorentzweg 1
2628CJ Delft, The Netherlands (email: d.elkousscoronas@qutech.nl)}

\tfootnote{This work was partially supported by the Netherlands Organization for Scientific Research (NWO/OCW), as part of the Quantum Software Consortium program (project number 024.003.037/3368), and the Unitary Fund.}

\markboth
{S. Hu \headeretal: IEEE Transactions on Quantum Engineering}
{S. Hu \headeretal: IEEE Transactions on Quantum Engineering}

\corresp{Corresponding author: First A. Author (email: author@ boulder.nist.gov).}

\begin{abstract}
    Quantum computing has the potential to transcend the information technology as we know it. Small scale quantum systems are already possible today, and the goal is to scale up these quantum architectures to build practical quantum devices. A major limiting factor in quantum computing is the accumulation of errors that may be caused by various sources. A solution is to encode the logical information in a larger amount of physical qubits to increase resilience and to decode the information with a decoder. A recently proposed decoder dubbed the Union-Find (UF) decoder utilizes the identically titled data structure to keep track of mergers between growing regions to find the appropiate correction. This decoder is fast and almost linear in its worst-case time complexity but has a reduced perfomance compared with the Minimum-Weight Perfect Matching (MWPM) decoder. We propose a modification of the UF decoder that aims to further decrease the weight of the correction operator, similarly to the MWPM decoder. The modified decoder, dubbed the Union-Find Node-Suspension (UFNS) decoder, manages to have an improved performance compared to the UF decoder of any lattice size. For small lattice sizes and also for low error rates, the UFNS decoder performs very comparable with the MWPM decoder. We manage to maintain a quasilinear worst-case time complexity of $\m{O}(n\log{n})$. 
\end{abstract}

\begin{keywords}
    Quantum Computing, Quantum Error Correction, Surface Code
\end{keywords}

\titlepgskip=-15pt
\maketitle

\section{Introduction}\label{sec:introduction}
One of the most promising approaches for fault-tolerant quantum computation is based on surface quantum error correcting codes \cite{dennis2002topological, kitaev2003fault}. With surface codes, error correction only requires the measurement of local operators on a 2-dimensional lattice of qubits. The measurement outcome, called the syndrome, is passed to the decoding algorithm to deduct the error that has occurred and to supply a correction operator. The resilience against errors can be improved by increasing the system size whilst the physical error rate is below a threshold value $p_{th}$. For this, it is essential that the decoder has low time complexity; if the clock-rate of the quantum computer becomes limited by the decoder, the advantages of increasing the system size could be compromised.

Arguably, the most popular decoder for surface codes is the Minimum-Weight Perfect Matching (MWPM) decoder \cite{dennis2002topological}. The basic principle behind this approach is to identify the \emph{lowest weight} error configuration that can produce the syndrome. In general this is a good approximation to the optimal maximum likelihood decoder \cite{bravyi2014efficient}. For a toric code that only suffers random Pauli noise, the optimal code threshold is $p_{th} = 10.9\%$, whereas the MWPM decoder has $p_{th} = 10.3\%$. The minimum-weight matchings are found by constructing a fully connected graph between nodes of the syndrome, which leads to a cubic worst-case time complexity of $\mathcal{O}(n^3)$, where $n$ is the number of qubits in the system \cite{kolmogorov2009blossom}. Fowler has proved that the matching problem can be solved in average$\mathcal{O}(1)$ time, but only at sufficiently low error rates, and the worst-case complexity remains significant \cite{fowler2013minimum}. 

Many other decoding algorithms have been developed \cites{duclos2013fault, hutter2015improved, watson2015fast, tuckett2018ultrahigh, kubica2019cellular, torlai2017neural, varsamopoulos2017decoding}. Here, we build on top of a recently proposed decoder called the Union-Find (UF) decoder. It combines a very low time complexity with a high threshold \cite{delfosse2017linear, delfosse2017almost} making it a practical solution for real devices. 
The UF decoder maps each syndrome to a vertex in a non-connected graph on the code lattice, and grows clusters of vertices locally by adding iteratively a layer of edges and vertices to existing clusters until all clusters have an even number of non-trivial syndrome vertices. It then trims the clusters until all non-trivial syndrome vertices are paired and linked by a path, which is the correcting operator. By growing the clusters of vertices in order of their sizes, the UF-decoder can be regarded as a heuristic for minimum-weight matching, and has a threshold of $p_{th} = 9.9\%$ for the toric code. The complexity of the UF decoder is driven by the merging between clusters. For this the algorithm uses the Union-Find or disjoint-set data structure \cite{tarjan1975efficiency}, which has worst-cast time complexity $\mathcal{O}(n\alpha(n))$, where $\alpha$ is the inverse of Ackermann's function. For any physical feasible amount of qubits, this value is $\alpha(n) \leq 3$, leading to an ``almost-linear'' time complexity.

We propose here a modification of the UF decoder that improves the heuristic for minimum-weight matching. The modified decoder, which we dub the \emph{Union-Find Balanced Bloom decoder} (UFBB), achieves near MWPM thresholds while retaining a quasilinear time complexity. In section \ref{sec:surfacecode} we introduce the surface code. In sections \ref{sec:matchingweight} and \ref{sec:ufbb} we describe the modified algorithm and its motivation. We discuss the complexity of the algorithm in section \ref{sec:complexity} and compare its performance with other decoders in section \ref{sec:performance}.  
\section{Union-Find Node-Suspension decoder}\label{sec:ufbb}

In this section, we describe the \emph{Union-Find Node-Suspension} decoder, which increases the Union-Find decoder's performance by improving its heuristic for minimum-weight matching. We first introduce the concept of the potential matching weight in \Cref{sec:matchingweight}. We describe the data structure required for this decoder in \Cref{sec:nodeset}, and the necessary calculations performed on this data structure in \Cref{sec:paritydelaysus,sec:nodejoin,sec:inversion}. The pseudocode is included in \Cref{sec:pseudocode}. 

\Figure[htb](topskip=0pt, botskip=0pt, midskip=0pt){tikzfigs/tikz-figure0.pdf}{
    A cluster with vertices $\{v_0, v_1, v_2\}$ with potential matching weights $\{2, 3, 2\}$. The line style and color of the colored edges correspond to the matching in the hypothetical union with an external vertex $v'$ of the same line style and color.\label{fig0}}

\subsection{Potential matching weight}\label{sec:matchingweight}
%In the following we give some intuition into the improvement of the Union-Find Balanced Bloom decoder upon the original Union-Find decoder. 
% We compared the ratio of the matchings between the MWPM decoder and our own implementation of the UF decoder, averaged over many simulations, and found that UF matching weight has a constant prefactor of $\sim 1.043$ over the minimum weight for the toric code (\Cref{comp_weight}). From this, we suspected that a decreased matching weight is a heuristic for an increased threshold. Within the context of the UF decoder, the matching weight may be decreased by prioritizing the growth of vertices with low PWM's within the cluster. 

Consider the cluster with index $i$ containing the set of non-trivial vertices $V_i=\{v_0,v_1,v_2\}$ and set of edges $E_i=\{(v_0,v_1), (v_1, v_2)\}$ of \Cref{fig0}. Now let us investigate the weight of a matching if an additional non-trivial vertex $v'$ is connected to the cluster. If $v'$ is connected to $v_0$ or to $v_2$, then the resulting matching has a total weight of 2: $(v',v_0)$ and $(v_1,v_2)$, or $(v_0,v_1)$ and $(v_2,v')$. However, if $v'$ is connected to vertex $v_2$, then the total weight is 3: $(v', v_1)$ and $(v_0, v_2)$. Inspired by this idea, we introduce the concept of potential matching weight (PMW) of a vertex. 

\begin{definition}\label{def:pmw}
    % For, the hypothetical merger with another odd-parity cluster $V_j, E_j$ on the edge $(v_i, v_j)$, with $v_i\in V_i$ and  $v_j \in V_j$, outputs an even-parity cluster with edges $E_{ij} = E_i \cup E_j \cup (v_i, v_j)$ in which there exists a matching $\m{C}_{(v,v')} \subseteq E_{ij}$ between syndromes internal to the cluster.
    % Let the Potential Matching Weight (PMW) of vertex $v_\alpha \in V_\alpha$ in an odd-parity cluster $\alpha$ with vertices $V_\alpha$ and edges $E_\alpha$ be
    % \begin{equation}
    %   PMW(v_\alpha) = \abs{\m{C}_{(v_\alpha,v_\beta)} \cap E_\alpha} + 1,
    % \end{equation}
    % where matching $\m{C}_{(v,v')} \subseteq E_{ij}$ is between the syndrome vertices internal to the even-parity cluster with edges $E_{ij} = E_\alpha \cup E_\beta \cup (v_\alpha, v_\beta)$, after a hypothetical merger of cluster $\alpha$ with another odd-parity cluster $V_\beta, E_\beta$ on the edge $(v_\alpha, v_\beta)$, with $v_\alpha\in V_\alpha$ and  $v_\beta \in V_\beta$
    Let there be a hypothetical merger between odd cluster $\alpha$ of vertices $V_\alpha$ and edges $E_\alpha$, and odd cluster $\beta$ of $V_\beta$ and $E_\beta$, on the edge $(v_\alpha, v_\beta)$, where $v_\alpha \in V_\alpha$ and $v_\beta \in V_\beta$. In the merged even cluster with edges $E_{\gamma} = E_\alpha \cup E_\beta \cup (v_\alpha, v_\beta)$, there is a matching $\m{C}_{(v,v')} \subseteq E_{\gamma}$  between the syndrome vertices internal to the cluster. The \textbf{Potential Matching Weight} (PMW) of vertex $v_\alpha$ is then defined as
    \begin{equation}
      PMW(v_\alpha) = \abs{\m{C}_{(v_\alpha,v_\beta)} \cap E_\alpha} + 1.
    \end{equation}
\end{definition}

In other words, the PMW is a vertex-specific predictive heuristic to the matching weight, assuming a union occurs in the next growth iteration. The PMW can be utilized by prioritizing the growth of vertices with low PMW such that there is an increased probability of mergers between clusters on edges connected to these vertices, and there is an increased probability in a lower matching weight. However, the PMWs' calculation within a cluster is not a trivial task, especially for clusters of increasingly larger size, as all edges of a cluster must be considered in its calculation. Furthermore, the PMWs within a cluster change due to cluster growth and mergers, both of which occur more frequently as the system size increases. For this reason, the scaling of the PMW computation is vital to the decoder. 

\Figure[htb](topskip=0pt, botskip=0pt, midskip=0pt){tikzfigs/tikz-figure1.pdf}{
    The cluster of \Cref{fig0} after two rounds of prioritized growth of $v_0$ and $v_2$. There are regions of vertices that are either interior elements or have equal potential matching weights, represented as nodes with different node radii in the node-tree $\nset$. \label{fig1}}

\subsection{Node-Suspension data structure}\label{sec:nodeset}

Fortunately, the PMW calculation is quite efficient by the introduction of a new data structure. Consider the cluster of non-trivial vertices $V_i=\{v_0,v_1,v_2\}$ and edges $E_i = \{(v_0,v_1), (v_1, v_2)\}$ from \Cref{fig0}. We had found previously that vertices $v_0, v_2$ have a lower PMW compared to $v_1$ by 1 edge. The growth of $v_0$ and $v_2$ are thus prioritized, such that new vertices are added to the cluster on the boundary of $v_0$ and $v_2$. If all newly added vertices are trivial, the cluster is now as in \Cref{fig1}. If we repeat the PMW calculation, we now find that the PMWs in the new vertices connected to $v_0$ are equal. The same is true for vertices connected to $v_2$. 
\begin{definition}\label{def:vertextree}
    Let the vertex-tree $\vset_i$ be a connected acyclic subgraph of the graph of a cluster $G(V_i, E_i)$.   The vertex-tree $\vset_i$ includes all vertices $V_i$ and a minimum number of edges in $E^\vset_i \subseteq E_i$. 
\end{definition}
\begin{definition}
  Let the node-tree $\nset_i$ be a partition of the vertex-tree $\vset_i$, such that each element of the partition --- a \textbf{node} $n$ --- consists of a set of adjacent vertices that lie within a certain distance --- the \textbf{node radius} $r$ --- from the \textbf{primer vertex}, which initializes the node and lies at its center. The node-tree is a directed acyclic graph, and its edges $\m{E}_i$ have lengths equal to the distance between the primer vertices of neighboring nodes. 
\end{definition}

The concept of primer vertices is easily understood when considering non-trivial vertices of the syndrome $\sigma$. Suppose every non-trivial vertex is the primer of a node, the weight of a matching in $\vset_i$ equal to the weight of the same matching in $\nset_i$. Furthermore, for every node of the node-tree, all vertices that lie at distance $r$ to the primer vertex are either boundary vertices to the cluster and have equal PMW, or lie within the radius of another node. For the example in \Cref{fig1}, the PMW of all boundary vertices of $n_0$, for simplicity just the PMW of $n_0$, is $\floor{r_0} + (n_1, n_2) + 1$. The partition from $\vset$ to $\nset$ thus allows us to compute the PMW on a reduced tree. 

\Figure[htb](topskip=0pt, botskip=0pt, midskip=0pt){tikzfigs/tikz-figure2.pdf}{
    Two different types of nodes. Syndrome-nodes $s$ have a non-trivial vertex or syndrome at its center. Vertices that lie on the radii of two existing nodes initialize a junction-node $j$ in the node-tree.\label{fig2}}

\Figure[hbt](topskip=0pt, botskip=0pt, midskip=0pt){tikzfigs/tikz-figure3.pdf}{
    The relevant data structures. \emph{(a)} The cluster-tree of the Union-Find data structure. The path from a vertex to the root of the cluster-tree is traversed to find the root element in order to differentiate between clusters. The root node of the node-tree is now additionally stored at the root of the cluster-tree. \emph{(b)} The vertex-tree $\vset$ with 9 non-trivial vertices. As $\vset$ is strictly acyclic, the cluster's edges must be maintained such that no cycles are created. This is done during growth by removing edges (red dotted lines) if a cycle is detected. \emph{(c)} The node-tree $\nset$, which currently has the same number of elements as $\vset$, as all vertices are non-trivial. Two depth-first searches are required to compute node parities (head recursively) and delays (tail recursively) in $\nset$.\label{fig3}}

All non-trivial vertices serve as primers for nodes that are called \textbf{syndrome-nodes} $s$. However, not all primer vertices are non-trivial vertices of the syndrome. If two non-trivial vertices are located an even Manhattan distance on the lattice, the growth of their clusters can simultaneously reach some vertex that lies on equal radii of the associated nodes, such as in \Cref{fig2}. For this reason, such vertices serve as primers of a different type of node --- a \textbf{junction-node} $j$ --- in the merged node-tree. 

The calculation of the PMW on the node-tree $\nset$ rather than the vertex-tree $\vset$ offers a reduction in the cost. However, it is still no trivial task as the entire tree must be considered for the calculation in every node. Instead, we will compute for the \textbf{node suspension} $n_s$ --- the number of growth iterations needed for a node to reach the maximum PMW in the node-tree --- which relates closely to the PMW. For example, the node suspension for the nodes $\{n_0, n_1, n_2\}$ associated with the vertices $\{v_0, v_1, v_2\}$ in \Cref{fig0} is $\{0, 2, 0\}$, and $\{0, 1, 0\}$ in \Cref{fig1}.

The Node-Suspension data structure does not replace but coexists with the Union-Find data structure. Additional to the the Union-Find data structure's cluster-trees of distinct roots, we store for every cluster the node-tree $\nset_i$ by its root node. For this, we need to maintain the reduced set of edges $E^\vset_i \subseteq E_i$ of the vertex-trees $\vset_i$ for every cluster, which can be done in constant time (see Algorithm \ref{algo:ufbb}). In the UF decoder, vertex-trees $\m{V}_i$ are not maintained, such that the graph associated with each cluster is not acyclic \cite{delfosse2017almost}. Instead, a spanning forest $F$ of all clusters is created \cite{delfosse2017linear} after growth. Each connected element within $F$ is also an acyclic graph. The difference is that while a single depth-first search or breath-first-search creates $F$, $\vset$ is equivalent to multiple breadth-first searches from each non-trivial vertex within the cluster, where the search of every breadth occurs during a growth iteration. The relevant data structures are depicted in \Cref{fig3}. 


\subsection{Node parity, delay, and suspension}\label{sec:paritydelaysus}

The Node-Suspension data structure allows for calculating the node suspension of all nodes in a node-tree $\nset$ by two intermediate steps. In each step, a depth-first-search (DFS) of $\nset$ is applied from its root node $r$ (\Cref{fig3}c).

In the first DFS, we calculate for the \textbf{node parity} $n_p$ --- the number of descendant syndrome-nodes of a node modulo 2 --- via a tail-recursive function, which is only dependent on the node parities of the children nodes of a node. The node parity is defined differently per node type:
\begin{align}\label{eq:nodeparity}
    s_p &= \hspace{.6cm}\big( \sum_{\mathclap{n \in \text{ children of } s}} (1-n_p) \big ) \bmod 2,\\
    j_p &= 1 - \big(\sum_{\mathclap{n \in \text{ children of } j}} (1-n_p) \big) \bmod 2.
\end{align}

In the second DFS, we calculate for the difference in node suspension of a node $n$ with its parent $m$; $\delta = n_s - m_s$. We can choose an arbitrary \textbf{node delay} $n_d$ --- the node suspension minus the maximum node suspension in the node-tree --- for the root node $r$ such as $r_d=0$ and add the suspension difference $\delta$ during each step to obtain $n_d$ for every node. This node delay of a node $n$ is only dependent on the node radii of itself and its parent $m$, the length of edge $(n,m)$, and its parity $n_p$. 
\begin{multline}\label{eq:delayequation}
    n_d = m_d + \bigg \lceil 2C\big(\ceil{n_r} - \floor{m_r + n_r \bmod 1}\\
    - (-1)^{n_p}\abs{(n,m)}\big) - 2(n_r - m_r) \bmod2 \bigg \rceil
\end{multline}
Here, the \textbf{inversion constant} $C$ deals with the inversion of node parities in a node-tree during merges of clusters explained in \ref{sec:nodejoin}. The node suspension is then related to the node delay by
\begin{equation*}
    n_s = n_d - \max_{x \in \nset}{x_d}. 
\end{equation*}
The maximum node delay can be maintained during the second DFS of the node-tree, and the node suspension itself is calculated during cluster growth. A single growth iteration, which is applied in the UF decoder by adding half-edges to all boundary vertices of the cluster, is now replaced by another DFS of $\nset$. During this DFS, we calculate the suspension $n_s$ for a node, and conditionally grow it - adding half-edges to the boundary vertices in the current node and adding 1 to its radius $n_r$ --- if $n_s = 0$. This requires us to save the list of boundary vertices to each node (\Cref{fig3}c). When all $n_s$ in $\nset$ are zero, all nodes are grown simultaneously within the same iteration. 

If the node-tree does not change after a growth iteration, which is the case if no mergers occur between clusters, the node suspensions decrease in an expected manner: For all nodes that are not suspended from growth, their node suspensions decrease with 1 in the next growth iteration. Due to this behavior, we can reuse the node delays $n_d$ to calculate $n_s$ for the next growth iteration by introducing another node parameter $n_w$, the number of iterations a node has \textbf{waited}. Each time a node is suspended from growth, we add 1 to $n_w$. The node suspension in subsequent iterations is then
\begin{equation}\label{eq:suspension}
    n_s = n_d - \max_{x \in \nset}{x_d} - n_w. 
\end{equation}
Note that we have not stated which node in $\nset$ should be the root node. In fact, any node in $\nset$ could have been picked as the root of the node-tree. As long as the DFS of cluster growth is performed in the same direction as the DFSs of the parity and delay calculations. If no cluster mergers occur, the node delays can be reused in the node suspension calculation prior to node growth. 
The node-tree is constructed by storing all neighbors of a node to a list. This way, the DFSs' direction can be determined by simply saving the root node, the starting point of the DFSs, to the cluster. All node variables are depicted in \Cref{fig3}c. 
%In the next section, we expand upon this idea of "reusing" some intermediate parameters to calculate the node suspensions after a cluster merger.  


\subsection{Joining node-trees}\label{sec:nodejoin}

In the Union-Find (UF) algorithm, odd parity clusters of an odd number of non-trivial vertices, --- elements of $\sigma$ --- grow in size repeatedly and merge with other clusters until all clusters are even. During these mergers, the node-trees of the Node-Suspension data structure must also be combined. Let us now first make a clear distinction between the merging protocols of the underlying data structures; the clusters-trees of the UF data structure are merged with the \codefunc{Union} function, whereas the node-trees are merged with a separate \codefunc{Join} function. After a join of multiple node-trees, the node suspensions within the combined node-tree change. Therefore, \codefunc{Join} protocol's focus is to minimize the DFSs of the recalculation of the node parity and delays in the combined node-tree. 

First, note that as a cluster of even parity has an even number of non-trivial vertices, its node-tree has an even number of syndrome-nodes. For these even node-trees, the concept of PMW does not exist, as the matching can be made within the node-tree. Consequently, node suspension, parity, and delays are undefined when two odd node-trees join to an even node-tree. 
%Thus, if two odd clusters merge into an even cluster, we don't know and do not care about its node suspensions. 

\Figure[hbt](topskip=0pt, botskip=0pt, midskip=0pt){tikzfigs/tikz-figure4.pdf}{
    \emph{(a)} An odd cluster $\nset_o=\{n_1, n_2, n_o\}$ with root $n_1$ joins with an even cluster $\nset_e=\{n_3, n_e\}$ with root $n_3$ on nodes $n_o, n_e$, respectively, to a joined node-tree. If we choose to \emph{(b)}, make $n_e$ a child of $n_o$, the parities and delays the sub-tree of $\nset_o$ can unchanged, and we only have to perform partial parity and delay calculations over the sub-tree of $\nset_e$. If we choose to \emph{(c)}, make $n_o$ a child of $n_e$, parities and delays have to be recalculated in the entire joined node-tree. \label{fig4}}

The second type of merger is between an even and an odd cluster. The combined cluster is odd, and its growth is continued. Thus its node suspensions must be computed. Consider the example of odd node-tree $\nset_o$ and even node-tree $\nset_e$ that are to be joined on nodes $n_o\in \nset_o$ and $n_e \in \nset_e$ (\Cref{fig4}\emph{a}). If $\nset_o$'s root is kept as the root of the joined node-tree (\Cref{fig4}\emph{b}), $n_e$ is to be a child node of $n_o$. As $\nset_e$ contains an even number of syndrome-nodes, the node parities in $\nset_o$ do not change. Hence, the node parity DFS is only necessary in the sub-tree $\nset_e$, which now has $n_e$ as sub-root. Furthermore, as the node delay is only dependent on its own properties and its parent's, the node delay DFS is also only required from node $n_e$ and within the sub-tree of $\nset_e$. These so-called \textbf{partial} DFSs of the node-tree are precisely what was required, as the node parity and delays in $\nset_e$ were undefined. Alternatively, if $\nset_e$'s root becomes the root of the combined tree (\Cref{fig4}\emph{c}), an odd number of syndrome-nodes are attached to $n_e$, such that the parities of nodes on the path from $n_e$ to the root are changed. Such a join would require the DFSs on the entire combined node-tree to calculate for node parities and delays. Thus, a simple rule is always to keep the root of the odd node-tree, which we dub \textbf{Odd-Rooted Join}.

In addition, a cluster can be subjected to multiple mergers within the same growth iteration, during which the parity of the merged cluster changes dependent on the number of mergers and the parities of the clusters involved. The DFSs related to the parity and delay calculations must, for this reason, not be initiated directly after the joining of node-trees. After all, it may be possible for the cluster to merge again such that the parities and delays become invalid. To prevent these redundant calculations, sub-roots of the even sub-trees are stored to a list $\m{S}$ at the root of the node-tree (\Cref{fig3}\emph{c}). When multiple mergers occur, the root node that stores the now redundant sub-roots is replaced by a new root with new $\m{S}$. If a cluster is selected for growth, we check for the sub-roots in $\m{S}$ at the new root node and initiate the DFSs from these sub-roots. We call this the \textbf{Root List $\m{S}$ Replacement}. 

\Figure[htb](topskip=0pt, botskip=0pt, midskip=0pt){tikzfigs/tikz-figure5.pdf}{
    The node suspension values for nodes for 3 odd node-trees $\{\nset_1, \nset_2, \nset_3\}$ of 3 nodes that grow and join into a single node-tree. \emph{(a)} Node suspensions are calculated by setting $C=1$ in equation \eqref{eq:delayequation}. In step 1, the growth in each of the three node-trees' outer nodes is prioritized, and the node-trees merge. In step 2, the recalculation of the joined node-tree is performed. Parities within the sub-tree of $\nset_2$ are now inverted, and the suspension in these nodes have doubled. \emph{(b)} Node suspensions are calculated by setting $C=\nicefrac{1}{2}$. Now the increase in node suspensions after parity inversion is halved.\label{fig5}}

\subsection{Parity inversion}\label{sec:inversion}
An unfortunate effect of the Node-Suspension data structure, which we dub \textbf{Parity Inversion}, causes a decrease in the algorithm's performance as the lattice size is increased. We will demonstrate this effect through the example in \Cref{fig5}\emph{a}. Consider three instances of the node-tree of \Cref{fig0}; $\nset_a, \nset_b, \nset_c$, positioned near each other on the lattice. For each node-tree, if the middle node is suspended from growth for two iterations, all nodes have the same Potential Matching Weight. However, in the current example, the node-trees $\nset_a, \nset_b, \nset_c$ merge after 1 iteration. The combined node-tree is odd. Thus, we recalculate the node parities and delays to find that the parities in the partition of the node-tree containing the nodes of $\nset_b$ have been inverted, and the node suspensions in this partition have doubled from before node suspensions before the merger. If the next merging event occurs on the node with the doubled node suspension, the matching weight may be larger compared to the original UF decoder, which defies the goal of Node-Suspension to decrease the weight.

This defines a trade-off in the Node-Suspension data structure; a node must wait as many iterations as it is suspended to reach equilibrium in Potential Matching Weight in the node-tree, but after Parity Inversion, the node suspension for previously prioritized nodes increases linearly with the number of iterations waited by the suspended nodes pre-inversion. As a compromise, we redefine the node suspension as \textbf{half} the number of growth iterations needed for all nodes in the node-tree to reach equal PMW. This can be done by setting $C=0.5$ in Equation \eqref{eq:delayequation}. Nevertheless, as more inversions occur, the maximum node suspension in the node-tree increases, and it becomes more and more unlikely for a cluster to actually reach zero node suspension in all nodes. The number of inversions is directly related to the number of merging events, and thus the size of the lattice. The performance to improve the heuristic for minimum weight matching thus decreases for larger lattices. 


\subsection{Pseudocode}\label{sec:pseudocode}
\FloatBarrier
The full version of the algorithm we have described is given in Algorithm \ref{algo:ufbb}. Note that this pseudocode includes instructions that are shortened versions of the pseudocode of the Union-Find decoder \cite{delfosse2017almost}. This is done for clarity on the additions of the Node-Suspension data structure and protocols on top of the Union-Find pseudocode. The first block 1-4 initializes the clusters and described the loop of cluster growth. Block 2 contains lines 5-8 and describes the DFS's related to the calculation of node parities and delays, and the DFS of the cluster growth. Block 3 contains lines 9-17 described the combined merging protocols of the Union-Find and Node-Suspension data structures. Node that lines 16-17 contain an extra step to ensure that the vertex-trees are always acyclic. The final block in line 17 is the peeling decoder \cite{delfosse2017linear}. 

\begin{algorithm}[htb]
    \BlankLine
    \KwData{A graph $G=(\m{V},\m{E})$, an erasure $\m{R} \subseteq \m{E}$ and syndrome $\sigma \subseteq \m{V}$}
    \KwResult{Correction set $\m{C}$}
    \BlankLine
    Initialize cluster vertex-trees, node-trees and other of UF.\;
    Create the list $\m{L}$ of odd clusters.\;
    \While(){$\m{L}$ is not empty}{
      Initialize the fusion list $\m{F}$ as an empty list.\;
      \For(){cluster $c\in\m{L}$}{
        \If(){pointer to even sub-root $n_e$ exists}{
          Apply DFS's to calculate node parities and delays (Equations \eqref{eq:nodeparity}, \eqref{eq:delayequation}) in partition of even sub-tree;
        }
        Apply DFS from root $n_r$ to all leafs. If $ns=0$ (Equation \eqref{eq:suspension}), grow all boundary edges of vertices in the node a half-edge per the Union-Find decoder, such that grown edges are added to $\m{F}$. If $ns\neq0$, apply $nw=nw+1$ and continue the DFS.\;
      }
      \For(){edge $(u,v) \in \m{F}$}{
        \eIf(){$\codefunc{Find}(u)\neq\codefunc{Find}(v)$}{
          Merge vertex-trees by weighted \codefunc{Union}.\;
          \eIf(){$u \in n_v$ and $v \in n_v$}{
            Merge node-trees by \codefunc{Join}.\;
          }(){
            Add $u$ or $v$ to $n_v$ or $n_u$, respectively.\;
          }
        }(){
          Subtract 1 from $(u,v)$ in \emph{Support}.\;
        }
      }
    }
    Apply the peeling decoder \cite{delfosse2017linear}.
    \caption{Union-Find Node-Suspension decoder}\label{algo:ufbb}
  \end{algorithm}
    

\section{Complexity of Node-Suspension}\label{sec:complexity}

In this section, we will find the worst-cast time complexity of the Union-Find Node-Suspension decoder. The addition cost of the original Union-Find decoder can be split in two parts: (A) the depth-first-searches (DFS's) related to the (re)calculation of the node parities and node delays in line \ref{algo:pdc}, and (B) the DFS related to the growth of a cluster in line \ref{algo:grow}. We dub the two parts the \textbf{suspension cost} and the \textbf{growth cost}, respectively. The \codefunc{Join} operation in lines \ref{algo:joina}-\ref{algo:joinb} only has a linear addition to the cost.

\subsection{Suspension cost}\label{sec:suscomplexity}

The cost of node suspension calculation is proportional to $N_{\text{sus}}$, the number of nodes traversed in the DFS's of the node parities and node delays. As a result of Odd-Rooted Join, $N_{\text{sus}}$ is proportional to the sum of sizes of all even node-trees. Root List Replacement decreases the sum to the most recent even node-trees that are ancestors of grown odd node-trees, which we dub $\mathbf{\Delta}$ \textbf{node-trees}. To find the worst-case time complexity, we maximize $N_{\text{sus}}$, which is proportional to the computation time. We take a time-reversed approach of analyzing a cluster; starting from a single cluster that maximally occupies the lattice at the end of growth, and move back in time to find its ancestor clusters. In this process that we call \textbf{cluster fragmentation}, we aim to find the set of cluster mergers that maximizes the number and sizes of $\Delta$ node-trees. 

% The maximization of $N_{\text{sus}}$ is in the repetitiveness of the recalculation over some parition of the final node-tree. 


% \begin{enumerate}[label=P\arabic*,ref=P\arabic*]
%   \item Joins between the node-trees of odd and even clusters retain the root node of the odd cluster. \label{rjoin}
%   \item The DFS's of the node parity and delay calculation are performed just before cluster growth in an even partition of the node-tree starting from the pointer saved at the root node. \label{rgrow}
%   \item The smallest clusters, measured by the number of verties in $\vset$, are grown first. \label{rweight}
% \end{enumerate}

% From properties \ref{rjoin} and \ref{rgrow}, it can be $N_{\text{sus}}$ is proportional to the sum of sizes of all even node-trees during all mergers of the growth process on the lattice. Note that if many clusters merge within the same growth iteration, only the last even cluster counts towards the cost, since \ref{rgrow} ensures that the calculation is not performed on intermediate even partitions.

\begin{definition}\label{def:fragmentation}
  Let the \textbf{fragmentation} function $f$ on an odd node-tree $\omega$ return the set of $\nu + 1$ of its most recent odd ancestor node-trees on which suspension calculations were performed. Let the prefix on the node-tree indicate the \textbf{fragmentation generation}, such that $f(\pre{k-1}\omega)$ returns $\nu + 1$ node-trees of generation $k$, denoted by $\pre{k}\m{F}$:
  \begin{equation}\label{eq:fstep}
    f(\{\pre{k-1}\omega\}) = \{\pre{k}\omega_0,\pre{k}\omega_1,...,\pre{k}\omega_{\nu}\}.
  \end{equation}
  Let $f$ be the combination of \textbf{partial fragmentation} functions $f_\omega$ and $f_\epsilon$, where $f_\omega$ fragments an odd node-tree $\pre{k-1}\omega$ into an odd ancestor $\pre{k}\omega_0$ and an even ancestor node-tree $\pre{k}\epsilon$: 
  \begin{equation}\label{eq:pfo}
    f_\omega(\{\pre{k-1}\omega\}) = \{\pre{k}\epsilon, \pre{k}\omega_0 \},
  \end{equation}
  and where $f_\epsilon$ further fragments $\pre{k}\epsilon$ into $\nu$ odd ancestors
  \begin{equation}\label{eq:pfe}
    f_\epsilon(\{\pre{k}\epsilon\}) =\{\pre{k}\omega_1,...,\pre{k}\omega_{\nu}\}.
  \end{equation}
  % of an odd cluster with node-tree $\pre{k-1}\omega$ split it into a set of its ancester node-trees. Here $k-1$ indicates the \emph{fragmentation step number}, where larger step number $k$ refers to an ancestor node-tree of smaller size. Let the fragmentation $f$ be the combination of \emph{intermediate fragmentations} (IF) $f_\omega$, which fragments an odd node-tree into an even ancestor and an odd ancestor

  % and $f_\epsilon$, which fragments even node-trees into $\nu$ odd ancestors
  % \begin{equation}\label{eq:pfe}
  %   f_\epsilon(\{\pre{k}\epsilon\}) = \m{F}^e_k=\{\pre{k}\omega_1,...,\pre{k}\omega_{\nu}\},
  % \end{equation}
  % such that a fragmentation is
\end{definition}

The fragmentation function can be applied consecutively, such that a set of odd node-trees of generation $k$ is fragmented to ancestors of generation $k+1$. We use the notation $f^{(2)}(\pre{k-1}\omega)$ to indicate that two fragmentations are applied on $\pre{k-1}\omega$ to obtain ancestors of generation $k+1$. Furthermore, let $f_\omega^{(i)}$ and be equivalent to $f_\omega f^{(i-1)}$. 

According to Odd-Rooted Join and Root List Replacement, if the cluster of node-tree $\pre{k-1}\omega$ is grown, the DFS's of the parity and delay calculations are performed within $\pre{k}\epsilon$. Therefore, every even node-tree returned by $f_\omega$ is a $\Delta$ node-tree. The value of $N_{\text{sus}}$ can thus be obtained by taking the sum of sizes of all even node-tree in $\pre{k}\m{F}_\omega$ over a series of $\mu$ fragmentations of some odd node-tree $\Omega = \pre{0}\omega$ until there are no more ancestor node-trees:
\begin{equation}\label{eq:npdc}
  N_{\text{sus}} = 2\sum_{k=1}^\mu{ \sum_{ \pre{k}\epsilon \in f_\omega^{(k)}(\Omega) }{ \abs{\pre{k}\epsilon}} }.
\end{equation}

To find $N_{\text{sus}}$, we are going to make two assumptions to simplify \eqref{eq:npdc}. Junction-nodes are initiated on the tangent of two node radii belonging to separate node-trees when merging into one. For increasing fragmentation generation, the total number of nodes in the fragmented set must therefore decrease. By neglecting their existence, \eqref{eq:npdc} becomes
\begin{equation}\label{eq:npdc2}
  N_{\text{sus}} \leq 2\sum_{k=1}^\mu{ \sum_{ \pre{k}\epsilon \in f_\omega^{(k)}(\Omega) }{ \abs{\pre{k}\epsilon}} }.
\end{equation}
Furthermore, if only syndrome-nodes exist, the size of $\pre{k-1}\omega$ must equal the sum of sizes of its ancestors $f(\pre{k-1}\omega)$. In other words, the size of the ancestor $\pre{k}\omega_i$ can be represented by the \textbf{fragmentation ratio}
\begin{equation}\label{eq:ratio}
  \pre{k}R_i = \frac{\abs{\pre{k}\omega_i}}{\abs{\pre{k-1}\omega}}, \hspace{0.5cm} \sum_{i=0}^{\mu}{\pre{k}R_i} = 1,
\end{equation}
Secondly, we assume that vertex-trees do not increase in size, such that $\abs{\nset}=\abs{\vset}$. Normally, the number of nodes in a cluster is bounded by the number of vertices $\abs{\nset}\leq \abs{\vset}$, as non-trivial vertices can be added to the node, which increase the node radius. By this assumption, the vertex-tree can only increase in size as the result of a merger between clusters, and nodes are effectively not allowed to increase in radius. While this is not possible in during realistic cluster growth, using this assumption simplifies \eqref{eq:npdc2}, as we will see later, while not compromising its upper bound. 

To find the upper bound in $N_{\text{sus}}$, we are now tasked to find: (a) $\nu$, the number of ancestors in $f_\epsilon$; (b) the fragmentation ratios $\{R_0, ..., R_\nu\}$; and (c) the number of fragmentation generations $\mu$. 

\begin{lemma}\label{lem:evenconstant}
  For constant fragmentation ratios $\pre{k}R_i = R_i$, the sum $\sum_{ \pre{k}\epsilon \in f_\omega^{(k)}(\Omega) }{ \abs{\pre{k}\epsilon}}$ is constant for every $k$. 
\end{lemma}
\begin{proof}
  For $k=1$, there is a single even ancestor $\pre{1}\epsilon$ of size 
  \begin{equation*}
    \abs{\pre{1}\epsilon} = (1 - R_0)\abs{\Omega}.
  \end{equation*}
  For $k=2$, every odd node-tree in $f(\Omega)=\{\pre{1}\omega_0,...,\pre{1}\omega_\nu\}$ is fragmented by $f_\omega$ to an even ancestor $\pre{2}\epsilon_i$ for $i \in \{0,...,\nu \}$, such that 
  \begin{equation*}
    \sum_{i=0}^{\nu}{\abs{\pre{2}\epsilon_i}}  = \sum_{i=0}^{\nu}{R_i(1 - R_0)\abs{\Omega}}= (1 - R_0)\abs{\Omega}.
  \end{equation*}
  The same is true for all subsequent generations $k$. 
\end{proof}

\begin{theorem}\label{the:fragnumber}
  Upper bound for $N_{\text{sus}}$ is obtained by setting $\nu=2$ in \eqref{eq:pfe}. 
\end{theorem}
\begin{proof}
  The sum of even node-tree sizes in every generation is constant per \Cref{lem:evenconstant}. Thus, upper bound in \eqref{eq:npdc2} is obtained by the largest possible $\mu$. As $\nu$ increases the number of odd node-trees in each $f^{(k)}_\omega$, the average size of these odd node-trees decreases. Since the size of a node-tree is proportional to the number of ancestor generations, we find that 
  \begin{equation*}
    \mu \propto \frac{1}{\nu}. 
  \end{equation*}
  Hence, the upper bound in \eqref{eq:npdc2} exists in the minimal value of $\nu$, which is $\nu = 2$.
\end{proof}

Using \Cref{the:fragnumber}, we now find that a fragmentation on an odd cluster $f(\pre{k-1}\omega)$ returns $\{\pre{k}\omega_0, \pre{k}\omega_1, \pre{k}\omega_2\}$, where $\pre{k}\omega_1, \pre{k}\omega_2$ are ancestors of the even node-tree $\pre{k}\epsilon$ returned by $f_\omega(\pre{k-1}\omega)$. 

\begin{lemma}\label{lem:chrono}
  The node-tree size of $\pre{k}\omega_0$ must be smaller than $\pre{k}\omega_1, \pre{k}\omega_2$, such that $R_1 \geq R_0 \leq R_2$. 
\end{lemma}
\begin{proof}
  The partial fragmentations must occur in the order of first \eqref{eq:pfo}, then \eqref{eq:pfe}, as \eqref{eq:pfe} requires an even node-tree that is returned by \eqref{eq:pfo}. In terms of cluster growth, the vertex-trees $\vset_1, \vset_2$, corresponding to $\pre{k}\omega_1, \pre{k}\omega_2$, must merge before the combined vertex-tree can merge with $\vset_0$, which corresponds to $\pre{k}\nset_0$. Declared by Weighted Growth, $\abs{\vset_1}$ and $\abs{\vset_2}$ must be smaller or equal to $\abs{\vset_0}$, such that 
  \begin{equation*}
    \abs{\vset_1}\geq \vset_0 \leq\abs{\vset_2}.
  \end{equation*}
  If this condition is not met, the cluster of $\vset_0$ grows first and merges with either $\vset_1$ or $\vset_2$, and the chronology of events is disturbed. Since we assumed $\abs{\nset}=\abs{\vset}$, this can be translated to 
  \begin{equation*}
    \abs{\nset_1}\geq \nset_0 \leq\abs{\nset_2},
  \end{equation*}
  and subsequently to the fragmentation ratios.
\end{proof}

\begin{theorem}\label{the:ratios}
  Upper bound for $N_{\text{sus}}$ is obtained via the fragmentation ratios $R_0 = R_1 = R_2 = \nicefrac{1}{3}$.
\end{theorem}
\begin{proof}
  The ratios $\{R_0, R_1, R_2\}$ can be found by maximizing the size of the even node-tree $\pre{k}\epsilon$ in each fragmentation, which is 
  \begin{equation*}
    \abs{\pre{k}\epsilon} = (R_1 + R_2)\abs{\pre{k-1}\omega}.
  \end{equation*}
  Since $ R_1 \geq R_0 \leq R_2$ per \Cref{lem:chrono}, the largest values for $R_1, R_2$ possible are equal to $R_0$.
\end{proof}

The last unknown parameter in finding the upper bound of $N_{\text{sus}}$ in Equation \eqref{eq:npdc2} is $\mu$.

\begin{theorem}\label{the:km}
  For $\nu = 2$ and $R_i = \{\nicefrac{1}{3},\nicefrac{1}{3},\nicefrac{1}{3}\}$, the maximum number of fragmentation generations is $\mu = \log_3{\abs{\Omega}}$.
\end{theorem}
\begin{proof}
  In every generation, all node-trees are fragmented into 3 ancestors that are $\nicefrac{1}{3}$ the size of their descendant. The series of $\mu$ fragmentations is thus simply $\mu$ divisions of the node-tree $\Omega$ in 3 parts until all ancestors have size 1, at which point a node-tree cannot be fragmented.
\end{proof}

Collecting \Cref{the:fragnumber,the:ratios,the:km} and filling in Equation \eqref{eq:npdc2} we find that

\begin{align*}
  \nonumber N_{\text{sus}} &\leq 2\sum_{k=1}^\mu{ \sum_{ \pre{k}\epsilon \in f_\omega^{(k)}(\Omega) }{ \abs{\pre{k}\epsilon}}  } \\
  \nonumber         &\leq 2\sum_{k=1}^{\log_3{\abs{\Omega}}} \frac{2}{3}\abs{\Omega}\\
                    &\leq \frac{4}{3}\abs{\Omega}\log_3{\abs{\Omega}}.
\end{align*}

The maximum size of the odd node-tree $\Omega$ is bounded by the system size $n = \abs{V}$. The worst-case time complexity of the suspension calculation is thus $\m{O}(n\log{n})$. 

\subsection{Growth cost}\label{sec:growthcost}

To grow a cluster represented by a node-tree $\nset$, a depth-first search (DFS) is performed on the node-tree to find all nodes that have zero suspension. The total cost of these DFS's are proportional to the total number of nodes encountered during these DFS's, which we dub $N_{\text{grow}}$. Using the definition of fragmentations of \Cref{def:fragmentation}, the cost of growth is proportional to
\begin{equation}\label{eq:ngrow}
  N_{\text{grow}} = 2\sum_{k=1}^\mu{ \sum_{ \pre{k}\omega \in f^{(k)}(\Omega) }{ \abs{\pre{k}\omega}} }.
\end{equation}
We assume again that no trivial vertices are added  to a cluster or $|\nset| = |\vset|$ such that \eqref{eq:ngrow} becomes an upper bound. As a result of Odd-Rooted Join and Root List Replacement, the upper bound is obtained if there are as many fragmentation generations. This is again achieved through $\nu = 2$. For every fragmentation of some odd node-tree $\pre{k-1}\omega$ into $\{\pre{k}\omega_0, \pre{k}\omega_1, \pre{k}\omega_2\}$, all three ancestors add to $N_{\text{grow}}$ if they have grown. According to Weighted Growth, this is the case when $R_0 \approx R_1 \approx R_2\approx \nicefrac{1}{3}$ such that $\abs{\vset_0}\approx \abs{\vset_1}\approx\abs{\vset_2}$. For these values of $\nu$ and $R$, we can apply \Cref{the:km} for $\mu$. For $|\nset| = |\vset|$, the sum of node-tree sizes in every fragmented set $\m{F}_k$ is exactly $\abs{\Omega}$, and we find that
\begin{align*}
  \nonumber N_{\text{grow}} &\leq 2\sum_{k=1}^\mu{ \sum_{ \pre{k}\omega \in f^{(k)}(\Omega) }{ \abs{\pre{k}\omega}}  } \\
  \nonumber         &\leq 2\sum_{k=1}^{\log_3{\abs{\Omega}}} \abs{\Omega}\\
                    &\leq 2\abs{\Omega}\log_3{\abs{\Omega}},
\end{align*}
which again corresponds to a worst-case time complexity $\m{O}(n\log{n})$.
\section{Performance}\label{sec:performance}


We benchmark the performance of the Union-Find Balanced-Bloom decoder of Algorithm \ref{algo:ufbb} using our application in Python3 (see Appendix \ref{ap:oopsurfacecode}). This is done by Monte Carlo simulations of decoding on a simulated lattice and to fit for the code threshold, described in Section \ref{sec:simthres}. For the independent noise model (Definition \ref{def:independent}), we simulate on lattice sizes $L_{small}=[8, 16, 24, 32, 40, 48, 56, 64]$ with a minimum of $96.000$ samples and on $L_{big}=[72, 80, 88, 96]$ with a minimum of $28.800$ samples. For the phenomenological noise model (Definition \ref{def:pheno}), we simulate on lattice sizes $L_{small}=[8,12,16,20,24]$ with a minimum of $105.600$ samples and on lattice sizes $L_{big}=[28, 32, 36, 40, 44]$ with $13.200$ samples. 

\Figure[htb](topskip=0pt, botskip=0pt, midskip=0pt){tikzfigs/threshold_comparison.pdf}{bla\label{thres_comp}}


\subsection{Threshold}

We had initially simulated for the decoding success rates for the range of lattices in $L_{small}$, which is the same range used when benchmarking the various implementations of the Union-Find decoder in Section \ref{sec:ufperformance}. For $L_{small}$, we find that, except for the environment of planar code with independent noise, the thresholds of the Union-Find Balanced-Bloom decoder are increased from the DBUF thresholds, and moves close to the thresholds of the MWPM decoder. We also observe an increase in decoding success rate at the threshold $k_{th}$ from DBUF, which is also the case for the environment of planar code with independent noise. For the range of lattices in $L_{big}$ (Figure \ref{fig:threshold_ufbbbig}), the threshold of the Union-Find Balanced-Bloom decoder decreases to below DBUF thresholds. This does not necessarily mean that it performs worse, as $k_{th}$ is still above DBUF's values, but does raise questions on the scalability of the Union-Find Balanced-Bloom decoder. In fact, if we look closely at the values of $k_C$ (Figure \ref{fig:thres_ufbb_toric_2d_data}), we find that the fit does not accurately represent the underlying data points, which is different behavior from the MWPM and DBUF decoders. The threshold and $k_C$ of the Union-Find Balanced-Bloom decoder may not be accurate for comparison. 

For this reason, we have applied a \emph{sequential fit} to the data acquired in the Monte Carlo simulations of lattices of $L = L_{small} \cup L_{big}$. In the sequential fit, we iterate stepwise in the ordered list of lattice sizes $L$ and fit for the data of $L_i, L_{i+1}$ for $i \in |L|-1$ iterations. The fit thus returns an error threshold in the range of the chosen lattice sizes. The range of thresholds $p_{th}(L_i, L_{i+1})$ for the environment of toric code with independent noise is plotted in Figure \ref{fig:thres_ufbb_toric_2d_seq}. We see that the sequential fits follow a trend where the increase in the input lattice sizes results in a decrease in $p_{th}$ but increase in $k_{th}$. The range of thresholds coordinates $(p_{th}, k_{th})$ is plotted in Figure \ref{fig:thres_ufbb_toric_2d_comp}, together with the data acquired from the simulations for the performance of the MWPM decoder and the DBUF decoder. Similar figures for the Monte Carlo simulations on the planar code and the phenomenological noise model are included in Figures \ref{fig:thres_ufbb_planar_2d}, \ref{fig:thres_ufbb_toric_3d}, and \ref{fig:thres_ufbb_planar_3d}. 

We ascribe the degradation of the threshold error rate to the \emph{parity inversion} effect of Definition \ref{def:parityinversion}. Recall from Lemma \ref{lem:eqstate} that the number of iterations waited before a union $I_t$ is proportional to the number of iterations required to reach the balanced-bloom state (Definition \ref{def:balancedbloom}) $M_{t+1}$, where $(I:M)$ is the equilibrium state of Definition \ref{def:eqstate}. By setting the equilibrium factor to $k_{eq}=\frac{1}{2}$, the equilibrium state is occupied half on average. The degradation is caused by the proportionality of the number of parity inversions and consequently equilibrium-state parameter $M$ to the lattice size. As the lattice size increases, the equilibrium state is still occupied half on average, but the absolute difference in $M-I$ increases. It is thus increasingly more unlikely that the balanced-bloom state is reached. 

Overall, for small lattice sizes, the Union-Find Balanced-Bloom decoder has an increased error threshold $p_{th}$ from the threshold of the Union-Find decoder and is comparable to the threshold values of the Minimum-Weight Perfect Matching decoder. The error threshold decreases for larger lattice sizes, but the Union-Find Balanced-Bloom decoder still has an increased performance, which is now apparent by an increased decoding success rate at the threshold $k_{th}$. The improvement across all lattice sizes is most apparent when comparing the range of threshold coordinates in $(p_X, k_C)$ space, where the coordinates now occupy a range that is not possible with the Union-Find decoder. 

\Figure[htb](topskip=0pt, botskip=0pt, midskip=0pt){tikzfigs/threshold_ufbb.pdf}{bla\label{threshold_ufbb}}


\subsection{Matching weight and running time}

Finally, we plot the average matching weight and running time of the Union-Find Balanced-Bloom (UFBB) compared with the Dynamic-forest Bucket Union-Find decoder (DBUF) and the Minimum-Weight Perfect Matching (MWPM) decoder for data acquired on simulations on a toric code with independent noise and $p_X = 0.1$ in Figure \ref{fig:ufbb_tmwcomp_toric_2d}. We can see from this figure that the Union-Find Balanced-Bloom decoder has a constant decreased weight. As for the running time, the UFBB decoder offers a midway choice between the MWPM decoder and DBUF decoder. We refer to Figures \ref{fig:mwcomp_ufbb} and \ref{fig:tcomp_ufbb} for the same plots but on a planar code or phenomenological noise, for which we observe the same behavior. 

We find that the decrease in weight is constant across the range of values of $p_X$, which is also the case for the planar code and the phenomenological noise model. We can compare the decrease in matching weight as the ratios between the normalized matching weight between the UFBB and DBUF decoders as
\begin{equation}
  r_{\abs{\m{C}}}=\frac{\abs{\m{C}_{UFBB}}/\abs{\m{C}_{MWPM}}}{\abs{\m{C}_{DBUF}}/\abs{\m{C}_{MWPM}}}. 
\end{equation}
We find the averaged values for $r_{\abs{\m{C}}}$ from the Monte Carlo simulations on the toric and planar lattices, and with the independent and phenomenological noise models, in Table \ref{tab:nmwratio}. The Union-Find Balanced-Bloom decoder successfully decreases the matching weight from the Dynamic-forest Bucket Union-Find decoder. The decrease is more apparent under the independent noise model. 

\Figure[htb](topskip=0pt, botskip=0pt, midskip=0pt){tikzfigs/comp_ufbb_toric_2d_p98.pdf}{bla\label{tmw_comp}}
% \Figure[htb](topskip=0pt, botskip=0pt, midskip=0pt){tikzfigs/threshold_comparison_dense.pdf}{bla\label{thres_comp_d}}
% this can be plotted with a shared y-axis
\section{Conclusion}\label{sec:conclusion}

In this paper, we have introduced a modification of the Union-Find (UF) decoder \cite{delfosse2017almost} that selectively grows regions of clusters based on the concept of a potential matching weight. The modified decoder, dubbed the Union-Find Node-Suspension (UFNS) decoder, relies on an additional data structure to facilitate the calculation of the potential matching weight. We have proved analytically that the UFNS decoder has a worst-case time complexity of $\m{O}(n\log{n})$. 

Through Monte Carlo simulations on various decoder types, we have found that the UFNS decoder improves upon the performance of the UF decoder for all tested physical error rates and system sizes. Unfortunately, there is no fixed error threshold due to the Parity Inversion effect, which affects the performance at larger lattice sizes. Nevertheless, the UFNS decoder manages to occupy a region in $(p_X, d)$ space previously reserved to the Minimum Weight Perfect Matching (MWPM) decoder. For the low-error regime, the UFNS combines the advantages of the MWPM decoder's high decoding rates and the UF decoder's low computation time. Future work should focus on finding a way around the Parity Inversion effect and testing the decoder for other error types, such as erasure errors. 

Recent work that includes the Union-Find decoder focuses on bringing the decoder algorithm to the hardware level. Most notably, a scalable decoder micro-architecture has been proposed with a fully pipelined hardware implementation \cite{das2020scalable}. Related work has shown that a reduction in bandwidth is possible provided qubits with a low physical error rate \cite{delfosse2020hierarchical}. Furthermore, another variant of the decoder, dubbed the \emph{Weighted Union-Find} decoder, not to be confused with \emph{Weighted Growth}, promises to increase the code threshold under circuit-level noise \cite{huang2020fault}. This application relies on adopting the decoder to a \emph{weighted} graph. Every edge $e\in\m{E}$ may now have a different length value, and edges are not limited to the growth of half-edges per growth iteration. We believe that the Union-Find Node-Suspension decoder and the Weighted Union-Find decoder are compatible. In the combined decoder, boundary edges in every node are grown with respect to their weights in the weighted graph. 

The Union-Find decoder manages to decode fast and scale almost-linearly with the input system size. However, these speed-ups come at the cost of a decreased decoding performance. With the Union-Find Node-Suspension decoder, we manage to find a middle ground between the two objectives; high decoding performance that runs in worst-case quasilinear time. For these reasons, it may be a great candidate for physical applications in the near future.

\Figure[b!](topskip=0pt, botskip=0pt, midskip=0pt){tikzfigs/comp_lowerror.pdf}{
  The decoding rate $d$ for the low-error regime of phenomenological noise for the MWPM, UFNS and bvUF decoders. The UFNS decoding rates are improved from the UF variants and are very similar to MWPM. All $d$ are obtained by Monte Carlo simulations with a minimum of $100.000$ samples. The x-axis scales linearly with $N = L^3$.\label{comp_lowerror}}

\Figure[b!](topskip=0pt, botskip=0pt, midskip=0pt){tikzfigs/comp_lowerror_time.pdf}{
  The mean computation time of the UFNS, bvUF, and MWPM decoders in the low error regime for phenomenological noise for $p_X = \{0.5\%, 1.2\%, 2\%\}$ of the same simulation as in \Cref{comp_lowerror}. In this regime, the UFNS computation times are very comparable to the bvUF decoder. The x-axis scales linearly with $N = L^3$. \label{comp_lowerror_time}}


\Figure[htb](topskip=0pt, botskip=0pt, midskip=0pt){tikzfigs/threshold_comparison.pdf}{
  Direct comparison of the performance of various decoders covered in this thesis. The data of the original Union-Find (UF) decoder is taken from its publication \cite{delfosse2017almost}. Using the same range of lattice sizes and error rates, we simulate and plot the performance of \emph{(1)} our implementation of the Union-Find decoder with Weighted Growth applied via bucket sort and acyclic vertex-trees maintained during growth, the bvUF decoder, \emph{(2)} the Union-Find Node-Suspension decoder (UFNS), and  \emph{(3)} the Minimum-Weight Perfect Matching (MWPM) decoder.\label{thres_comp}}

\FloatBarrier
\printbibliography
\EOD
\end{document}