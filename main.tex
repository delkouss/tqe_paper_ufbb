\documentclass{ieeeaccess}
\usepackage[OT1]{fontenc} 
\usepackage{amsfonts, amsmath, amssymb, amsthm}
\usepackage{algorithmic}
\usepackage{graphicx}
\usepackage{caption}
\usepackage{placeins}
% \DeclareCaptionFont{ieeeblue}{\color{accessblue}}
% \DeclareCaptionLabelFormat{myformat}{\figcapfont{\textbf{#1}\textbf{#2}}}
% \captionsetup{labelfont={bf,ieeeblue},labelformat=myformat}
\usepackage{setspace}
\usepackage[font={sf,scriptsize,stretch=0.84}, labelfont={bf,color=accessblue}]{caption}

\usepackage{textcomp}
\usepackage{mathtools}
\DeclarePairedDelimiter{\abs}{\lvert}{\rvert}
% \def\BibTeX{{\rm B\kern-.05em{\sc i\kern-.025em b}\kern-.08em
%     T\kern-.1667em\lower.7ex\hbox{E}\kern-.125emX}}
\usepackage[style=ieee, sorting=none]{biblatex}
\addbibresource{cit.bib}

% \usepackage[table]{xcolor}
% % Define colors

\newtheorem{definition}{Definition}[section]
\newtheorem{lemma}{Lemma}[section]
\newtheorem{theorem}{Theorem}[section]
\newtheorem{proposition}{Proposition}[section]

% Set new commands
\newcommand{\codeword}[1]{\texttt{\textcolor{MidnightBlue}{#1}}}
%\newcommand{\codefunc}[1]{\texttt{\textcolor{OliveGreen}{#1}}}
\let\oldemptyset\emptyset
\let\emptyset\varnothing
\newcommand{\codefunc}[1]{\texttt{#1}}
\newcommand{\m}[1]{\mathcal{#1}}
\newcommand{\n}[1]{\mathscr{#1}}
\newcommand{\bound}{\mathscr{B}}
\newcommand{\akker}{\mathscr{A}}
\newcommand{\nset}{\mathcal{N}}
\newcommand{\vset}{\mathcal{V}}
\newcommand{\pre}[1]{ {}^{#1} }
\newcommand{\ceil}[1]{{\left \lceil #1 \right \rceil }}
\newcommand{\floor}[1]{{\left \lfloor #1 \right \rfloor }}

% Set algorithm2e package settings
\usepackage[linesnumbered, ruled, vlined]{algorithm2e}
\SetAlgoCaptionLayout{centerline}
\setlength{\algoheightrule}{1pt}
\setlength{\algotitleheightrule}{1pt}
\setlength{\interspacetitleboxruled}{.5em}
\SetStartEndCondition{ }{}{}
\SetKwProg{Fn}{def}{\string:}{}
\SetKw{KwTo}{in}
\SetKwFor{For}{for}{\string:}{}
\SetKwIF{If}{ElseIf}{Else}{if}{ then}{else if}{else}{}
\SetKwFor{While}{while}{ do}{}
\SetInd{0.1em}{0.5em}
\SetAlgoNoEnd\DontPrintSemicolon
\SetAlFnt{\small}


\begin{document}
\history{Date of publication xxxx 00, 0000, date of current version xxxx 00, 0000.}
\doi{10.1109/TQE.2020.DOI}

\title{Quasilinear Time Decoding Algorithm for \\Topological Codes with High Error Threshold}
\author{
    \uppercase{S. Hu}\authorrefmark{1}, \IEEEmembership{Fellow, IEEE},
    \uppercase{and D. Elkouss\authorrefmark{2}}
}
\address[1]{Department of Physics, Delft University of Technology (email: watermarkhu@outlook.com)}
\address[2]{QuTech, Delft University of Technology, Lorentzweg 1
2628CJ Delft, The Netherlands (email: d.elkousscoronas@qutech.nl)}

\tfootnote{This paragraph of the first footnote will contain support 
information, including sponsor and financial support acknowledgment. For 
example, ``This work was supported in part by the U.S. Department of 
Commerce under Grant BS123456.''}

\markboth
{S. Hu \headeretal: IEEE Transactions on Quantum Engineering}
{S. Hu \headeretal: IEEE Transactions on Quantum Engineering}

\corresp{Corresponding author: First A. Author (email: author@ boulder.nist.gov).}

\begin{abstract}
    Quantum computing has the potential to transcend the information technology as we know it. Small scale quantum systems are already possible today, and the goal is to scale up these quantum architectures to build practical quantum devices. A major limiting factor in quantum computing is the accumulation of errors that may be caused by various sources. A solution is to encode the logical information in a larger amount of physical qubits to increase resilience and to decode the information with a decoder. 
    
    We introduce a modification of the Union-Find decoder that aims to futher decrease the weight of the correction operator, similarly to the Minimum-Weight Perferct Matching decoder. The modified decoder, dubbed the Union-Find Balanced-Bloom decoder, manages to have an increased error threshold compared to the Union-Find decoder of any lattice size. For small lattice sizes, the Union-Find Balanced-Bloom decoder performs near the code threshold of the Minimum-Weight Matching decoder. We manage to maintain a quasilinear worst-case time complexity of $\m{O}(N\log{N})$. 
\end{abstract}

\begin{keywords}
    Quantum Computing, Quantum Error Correction, Surface Code
\end{keywords}

\titlepgskip=-15pt
\maketitle

\section{Introduction}\label{sec:introduction}
One of the most promising approaches for fault-tolerant quantum computation is based on surface quantum error correcting codes \cite{dennis2002topological, kitaev2003fault}. With surface codes, error correction only requires the measurement of local operators on a 2-dimensional lattice of qubits. The measurement outcome, called the syndrome, is passed to the decoding algorithm to deduct the error that has occurred and to supply a correction operator. The resilience against errors can be improved by increasing the system size whilst the physical error rate is below a threshold value $p_{th}$. For this, it is essential that the decoder has low time complexity; if the clock-rate of the quantum computer becomes limited by the decoder, the advantages of increasing the system size could be compromised.

Arguably, the most popular decoder for surface codes is the Minimum-Weight Perfect Matching (MWPM) decoder \cite{dennis2002topological}. The basic principle behind this approach is to identify the \emph{lowest weight} error configuration that can produce the syndrome. In general this is a good approximation to the optimal maximum likelihood decoder \cite{bravyi2014efficient}. For a toric code that only suffers random Pauli noise, the optimal code threshold is $p_{th} = 10.9\%$, whereas the MWPM decoder has $p_{th} = 10.3\%$. The minimum-weight matchings are found by constructing a fully connected graph between nodes of the syndrome, which leads to a cubic worst-case time complexity of $\mathcal{O}(n^3)$, where $n$ is the number of qubits in the system \cite{kolmogorov2009blossom}. Fowler has proved that the matching problem can be solved in average$\mathcal{O}(1)$ time, but only at sufficiently low error rates, and the worst-case complexity remains significant \cite{fowler2013minimum}. 

Many other decoding algorithms have been developed \cites{duclos2013fault, hutter2015improved, watson2015fast, tuckett2018ultrahigh, kubica2019cellular, torlai2017neural, varsamopoulos2017decoding}. Here, we build on top of a recently proposed decoder called the Union-Find (UF) decoder. It combines a very low time complexity with a high threshold \cite{delfosse2017linear, delfosse2017almost} making it a practical solution for real devices. 
The UF decoder maps each syndrome to a vertex in a non-connected graph on the code lattice, and grows clusters of vertices locally by adding iteratively a layer of edges and vertices to existing clusters until all clusters have an even number of non-trivial syndrome vertices. It then trims the clusters until all non-trivial syndrome vertices are paired and linked by a path, which is the correcting operator. By growing the clusters of vertices in order of their sizes, the UF-decoder can be regarded as a heuristic for minimum-weight matching, and has a threshold of $p_{th} = 9.9\%$ for the toric code. The complexity of the UF decoder is driven by the merging between clusters. For this the algorithm uses the Union-Find or disjoint-set data structure \cite{tarjan1975efficiency}, which has worst-cast time complexity $\mathcal{O}(n\alpha(n))$, where $\alpha$ is the inverse of Ackermann's function. For any physical feasible amount of qubits, this value is $\alpha(n) \leq 3$, leading to an ``almost-linear'' time complexity.

We propose here a modification of the UF decoder that improves the heuristic for minimum-weight matching. The modified decoder, which we dub the \emph{Union-Find Balanced Bloom decoder} (UFBB), achieves near MWPM thresholds while retaining a quasilinear time complexity. In section \ref{sec:surfacecode} we introduce the surface code. In sections \ref{sec:matchingweight} and \ref{sec:ufbb} we describe the modified algorithm and its motivation. We discuss the complexity of the algorithm in section \ref{sec:complexity} and compare its performance with other decoders in section \ref{sec:performance}.  
\section{The Surface Code}\label{sec:surfacecode}

The Union-Find Node-Suspension decoder proposed here has the same compatibility as its parent decoder, and is applicable to any surface code of any genus, with or without boundary, and to color codes. For simplicity, we only describe the standard implementation of the surface code without boundary.

\subsection{The toric code}

The \emph{toric code}, a topological code introduced by Kitaev \cite{kitaev2003fault}, is defined by arranging qubits on the edges of a square lattice with periodic boundary conditions. The code is denoted by $V,E,F$, respectively the set of vertices, set of edges, and the set of faces on the lattice. The toric code is defined to be the ground state of the Hamiltonian 
\begin{equation}
    H = -\sum_{v \in V} X_v -\sum_{f \in F} Z_f, 
\end{equation}
where operator $X_v$ is the product of Pauli $X$ operators on the qubits located on edges forming the vertex $v$, \emph{i.e.,} $X_v = \prod_{e \in v} X_e$, and $Z_f = \prod_{e \in f} Z_e$ is the product of Pauli $Z$ operators on the qubits located on edges of face $f$. The code space is spanned by the simultaneous "+1" eigenstate of all operators $X_v$ and $Z_f$. These operators, together with any possible product of them, are the \emph{stabilizers} of the code, and form the stabilizer group $S$. This topology encodes the logical operators in the torus' non-trivial cycles. Errors, below a certain threshold, will only introduce local effects and do not change these cycles.

\subsection{Error model}\label{sec:errormodel}
For simplicity, we will only consider i.i.d. phase-flip errors, where each qubit is subjected to a $Z$ error with probability $p_Z$. Due to \emph{lattice duality}, where the vertices and faces of the lattices can be interchanged, the error detection and correction of bit-flip errors is identical. 
Additionally, any qubit may be \emph{erased} from the system with probability $p_e$. The set of erased qubits is denoted with $\varepsilon$. This \emph{erasure} is detectable, such that we can replace or reinitiate all erased qubits, which corresponds to a random Pauli error after measurement. 

\subsection{Error correction}
Error correction is proceeded by measuring a set of independent stabilizers of the code, \emph{i.e.,} the operators $X_v$ and $Z_f$. For a set of phase-flip errors $E_Z = \{I,Z\}^{\otimes n}$, the stabilizers $X_v$ that anticommute with the error return a non-trivial outcome. The set of non-trivial eigenvalues of the stabilizers is called the syndrome $\sigma$ of the code. Given the measured $\sigma$, and optionally the known erasure $\varepsilon$, it is the task of the decoder to find the correction operator $\mathcal{C}(\sigma, \varepsilon)$. When the correction operator is applied, the code is returned to the code space, \emph{i.e.} $\mathcal{C}(\sigma, \varepsilon)E_Z \in S$. The error is corrected up to a stabilizer. The mapping of measured syndrome to the correction is thus not one-to-one, and it is up to the decoder to choose the most appropriate correction. 
\section{Union-Find decoder}\label{sec:unionfind}

The Union-Find decoder \cite{delfosse2017linear, delfosse2017almost} maps each syndrome $\sigma_i$ to a so-called non-trivial vertex $v_i$ in a non-connected graph on the code lattice, and grows clusters $c_j$ of vertices that form a connected graph locally, by adding iteratively a layer of edges and trivial vertices to existing clusters until all clusters have an even number of non-trivial syndrome vertices. Then, a spanning tree is built for each cluster, and every tree is traversed until all non-trivial syndrome vertices are paired and linked by a path, which is the correcting operator $\mathcal{C}$. By growing the clusters of vertices in order of their sizes, which is dubbed the \emph{weighted growth} version of the Union-Find decoder, the threshold is reported to increase from $9.2\%$ to $9.9\%$ compared to the non-weighted variant for the toric code. 

The complexity of the Union-Find decoder is driven by the merging between clusters. For this the algorithm uses the Union-Find or disjoint-set data structure \cite{tarjan1975efficiency}. The function \codefunc{Find} is used to traverse the vertex-tree of the cluster to the distinctive root element to find the parent cluster of a given vertex. If a newly added vertex has a different root, the vertex trees are merged by \codefunc{Union}. 
\section{Union-Find Node-Suspension decoder}\label{sec:ufbb}

In this section, we describe the \emph{Union-Find Node-Suspension} decoder, which increases the Union-Find decoder's performance by improving its heuristic for minimum-weight matching. We first introduce the concept of the potential matching weight in \Cref{sec:matchingweight}. We describe the data structure required for this decoder in \Cref{sec:nodeset}, and the necessary calculations performed on this data structure in \Cref{sec:paritydelaysus,sec:nodejoin,sec:inversion}. The pseudocode is included in \Cref{sec:pseudocode}. 

\Figure[htb](topskip=0pt, botskip=0pt, midskip=0pt){tikzfigs/tikz-figure0.pdf}{
    A cluster with vertices $\{v_0, v_1, v_2\}$ with potential matching weights $\{2, 3, 2\}$. The line style and color of the colored edges correspond to the matching in the hypothetical union with an external vertex $v'$ of the same line style and color.\label{fig0}}

\subsection{Potential matching weight}\label{sec:matchingweight}
%In the following we give some intuition into the improvement of the Union-Find Balanced Bloom decoder upon the original Union-Find decoder. 
% We compared the ratio of the matchings between the MWPM decoder and our own implementation of the UF decoder, averaged over many simulations, and found that UF matching weight has a constant prefactor of $\sim 1.043$ over the minimum weight for the toric code (\Cref{comp_weight}). From this, we suspected that a decreased matching weight is a heuristic for an increased threshold. Within the context of the UF decoder, the matching weight may be decreased by prioritizing the growth of vertices with low PWM's within the cluster. 

Consider the cluster with index $i$ containing the set of non-trivial vertices $V_i=\{v_0,v_1,v_2\}$ and set of edges $E_i=\{(v_0,v_1), (v_1, v_2)\}$ of \Cref{fig0}. Now let us investigate the weight of a matching if an additional non-trivial vertex $v'$ is connected to the cluster. If $v'$ is connected to $v_0$ or to $v_2$, then the resulting matching has a total weight of 2: $(v',v_0)$ and $(v_1,v_2)$, or $(v_0,v_1)$ and $(v_2,v')$. However, if $v'$ is connected to vertex $v_2$, then the total weight is 3: $(v', v_1)$ and $(v_0, v_2)$. Inspired by this idea, we introduce the concept of potential matching weight (PMW) of a vertex. 

\begin{definition}\label{def:pmw}
    % For, the hypothetical merger with another odd-parity cluster $V_j, E_j$ on the edge $(v_i, v_j)$, with $v_i\in V_i$ and  $v_j \in V_j$, outputs an even-parity cluster with edges $E_{ij} = E_i \cup E_j \cup (v_i, v_j)$ in which there exists a matching $\m{C}_{(v,v')} \subseteq E_{ij}$ between syndromes internal to the cluster.
    % Let the Potential Matching Weight (PMW) of vertex $v_\alpha \in V_\alpha$ in an odd-parity cluster $\alpha$ with vertices $V_\alpha$ and edges $E_\alpha$ be
    % \begin{equation}
    %   PMW(v_\alpha) = \abs{\m{C}_{(v_\alpha,v_\beta)} \cap E_\alpha} + 1,
    % \end{equation}
    % where matching $\m{C}_{(v,v')} \subseteq E_{ij}$ is between the syndrome vertices internal to the even-parity cluster with edges $E_{ij} = E_\alpha \cup E_\beta \cup (v_\alpha, v_\beta)$, after a hypothetical merger of cluster $\alpha$ with another odd-parity cluster $V_\beta, E_\beta$ on the edge $(v_\alpha, v_\beta)$, with $v_\alpha\in V_\alpha$ and  $v_\beta \in V_\beta$
    Let there be a hypothetical merger between odd cluster $\alpha$ of vertices $V_\alpha$ and edges $E_\alpha$, and odd cluster $\beta$ of $V_\beta$ and $E_\beta$, on the edge $(v_\alpha, v_\beta)$, where $v_\alpha \in V_\alpha$ and $v_\beta \in V_\beta$. In the merged even cluster with edges $E_{\gamma} = E_\alpha \cup E_\beta \cup (v_\alpha, v_\beta)$, there is a matching $\m{C}_{(v,v')} \subseteq E_{\gamma}$  between the syndrome vertices internal to the cluster. The \textbf{Potential Matching Weight} (PMW) of vertex $v_\alpha$ is then defined as
    \begin{equation}
      PMW(v_\alpha) = \abs{\m{C}_{(v_\alpha,v_\beta)} \cap E_\alpha} + 1.
    \end{equation}
\end{definition}

In other words, the PMW is a vertex-specific predictive heuristic to the matching weight, assuming a union occurs in the next growth iteration. The PMW can be utilized by prioritizing the growth of vertices with low PMW such that there is an increased probability of mergers between clusters on edges connected to these vertices, and there is an increased probability in a lower matching weight. However, the PMWs' calculation within a cluster is not a trivial task, especially for clusters of increasingly larger size, as all edges of a cluster must be considered in its calculation. Furthermore, the PMWs within a cluster change due to cluster growth and mergers, both of which occur more frequently as the system size increases. For this reason, the scaling of the PMW computation is vital to the decoder. 

\Figure[htb](topskip=0pt, botskip=0pt, midskip=0pt){tikzfigs/tikz-figure1.pdf}{
    The cluster of \Cref{fig0} after two rounds of prioritized growth of $v_0$ and $v_2$. There are regions of vertices that are either interior elements or have equal potential matching weights, represented as nodes with different node radii in the node-tree $\nset$. \label{fig1}}

\subsection{Node-Suspension data structure}\label{sec:nodeset}

Fortunately, the PMW calculation is quite efficient by the introduction of a new data structure. Consider the cluster of non-trivial vertices $V_i=\{v_0,v_1,v_2\}$ and edges $E_i = \{(v_0,v_1), (v_1, v_2)\}$ from \Cref{fig0}. We had found previously that vertices $v_0, v_2$ have a lower PMW compared to $v_1$ by 1 edge. The growth of $v_0$ and $v_2$ are thus prioritized, such that new vertices are added to the cluster on the boundary of $v_0$ and $v_2$. If all newly added vertices are trivial, the cluster is now as in \Cref{fig1}. If we repeat the PMW calculation, we now find that the PMWs in the new vertices connected to $v_0$ are equal. The same is true for vertices connected to $v_2$. 
\begin{definition}\label{def:vertextree}
    Let the vertex-tree $\vset_i$ be a connected acyclic subgraph of the graph of a cluster $G(V_i, E_i)$.   The vertex-tree $\vset_i$ includes all vertices $V_i$ and a minimum number of edges in $E^\vset_i \subseteq E_i$. 
\end{definition}
\begin{definition}
  Let the node-tree $\nset_i$ be a partition of the vertex-tree $\vset_i$, such that each element of the partition --- a \textbf{node} $n$ --- consists of a set of adjacent vertices that lie within a certain distance --- the \textbf{node radius} $r$ --- from the \textbf{primer vertex}, which initializes the node and lies at its center. The node-tree is a directed acyclic graph, and its edges $\m{E}_i$ have lengths equal to the distance between the primer vertices of neighboring nodes. 
\end{definition}

The concept of primer vertices is easily understood when considering non-trivial vertices of the syndrome $\sigma$. Suppose every non-trivial vertex is the primer of a node, the weight of a matching in $\vset_i$ equal to the weight of the same matching in $\nset_i$. Furthermore, for every node of the node-tree, all vertices that lie at distance $r$ to the primer vertex are either boundary vertices to the cluster and have equal PMW, or lie within the radius of another node. For the example in \Cref{fig1}, the PMW of all boundary vertices of $n_0$, for simplicity just the PMW of $n_0$, is $\floor{r_0} + (n_1, n_2) + 1$. The partition from $\vset$ to $\nset$ thus allows us to compute the PMW on a reduced tree. 

\Figure[htb](topskip=0pt, botskip=0pt, midskip=0pt){tikzfigs/tikz-figure2.pdf}{
    Two different types of nodes. Syndrome-nodes $s$ have a non-trivial vertex or syndrome at its center. Vertices that lie on the radii of two existing nodes initialize a junction-node $j$ in the node-tree.\label{fig2}}

\Figure[hbt](topskip=0pt, botskip=0pt, midskip=0pt){tikzfigs/tikz-figure3.pdf}{
    The relevant data structures. \emph{(a)} The cluster-tree of the Union-Find data structure. The path from a vertex to the root of the cluster-tree is traversed to find the root element in order to differentiate between clusters. The root node of the node-tree is now additionally stored at the root of the cluster-tree. \emph{(b)} The vertex-tree $\vset$ with 9 non-trivial vertices. As $\vset$ is strictly acyclic, the cluster's edges must be maintained such that no cycles are created. This is done during growth by removing edges (red dotted lines) if a cycle is detected. \emph{(c)} The node-tree $\nset$, which currently has the same number of elements as $\vset$, as all vertices are non-trivial. Two depth-first searches are required to compute node parities (head recursively) and delays (tail recursively) in $\nset$.\label{fig3}}

All non-trivial vertices serve as primers for nodes that are called \textbf{syndrome-nodes} $s$. However, not all primer vertices are non-trivial vertices of the syndrome. If two non-trivial vertices are located an even Manhattan distance on the lattice, the growth of their clusters can simultaneously reach some vertex that lies on equal radii of the associated nodes, such as in \Cref{fig2}. For this reason, such vertices serve as primers of a different type of node --- a \textbf{junction-node} $j$ --- in the merged node-tree. 

The calculation of the PMW on the node-tree $\nset$ rather than the vertex-tree $\vset$ offers a reduction in the cost. However, it is still no trivial task as the entire tree must be considered for the calculation in every node. Instead, we will compute for the \textbf{node suspension} $n_s$ --- the number of growth iterations needed for a node to reach the maximum PMW in the node-tree --- which relates closely to the PMW. For example, the node suspension for the nodes $\{n_0, n_1, n_2\}$ associated with the vertices $\{v_0, v_1, v_2\}$ in \Cref{fig0} is $\{0, 2, 0\}$, and $\{0, 1, 0\}$ in \Cref{fig1}.

The Node-Suspension data structure does not replace but coexists with the Union-Find data structure. Additional to the the Union-Find data structure's cluster-trees of distinct roots, we store for every cluster the node-tree $\nset_i$ by its root node. For this, we need to maintain the reduced set of edges $E^\vset_i \subseteq E_i$ of the vertex-trees $\vset_i$ for every cluster, which can be done in constant time (see Algorithm \ref{algo:ufbb}). In the UF decoder, vertex-trees $\m{V}_i$ are not maintained, such that the graph associated with each cluster is not acyclic \cite{delfosse2017almost}. Instead, a spanning forest $F$ of all clusters is created \cite{delfosse2017linear} after growth. Each connected element within $F$ is also an acyclic graph. The difference is that while a single depth-first search or breath-first-search creates $F$, $\vset$ is equivalent to multiple breadth-first searches from each non-trivial vertex within the cluster, where the search of every breadth occurs during a growth iteration. The relevant data structures are depicted in \Cref{fig3}. 


\subsection{Node parity, delay, and suspension}\label{sec:paritydelaysus}

The Node-Suspension data structure allows for calculating the node suspension of all nodes in a node-tree $\nset$ by two intermediate steps. In each step, a depth-first-search (DFS) of $\nset$ is applied from its root node $r$ (\Cref{fig3}c).

In the first DFS, we calculate for the \textbf{node parity} $n_p$ --- the number of descendant syndrome-nodes of a node modulo 2 --- via a tail-recursive function, which is only dependent on the node parities of the children nodes of a node. The node parity is defined differently per node type:
\begin{align}\label{eq:nodeparity}
    s_p &= \hspace{.6cm}\big( \sum_{\mathclap{n \in \text{ children of } s}} (1-n_p) \big ) \bmod 2,\\
    j_p &= 1 - \big(\sum_{\mathclap{n \in \text{ children of } j}} (1-n_p) \big) \bmod 2.
\end{align}

In the second DFS, we calculate for the difference in node suspension of a node $n$ with its parent $m$; $\delta = n_s - m_s$. We can choose an arbitrary \textbf{node delay} $n_d$ --- the node suspension minus the maximum node suspension in the node-tree --- for the root node $r$ such as $r_d=0$ and add the suspension difference $\delta$ during each step to obtain $n_d$ for every node. This node delay of a node $n$ is only dependent on the node radii of itself and its parent $m$, the length of edge $(n,m)$, and its parity $n_p$. 
\begin{multline}\label{eq:delayequation}
    n_d = m_d + \bigg \lceil 2C\big(\ceil{n_r} - \floor{m_r + n_r \bmod 1}\\
    - (-1)^{n_p}\abs{(n,m)}\big) - 2(n_r - m_r) \bmod2 \bigg \rceil
\end{multline}
Here, the \textbf{inversion constant} $C$ deals with the inversion of node parities in a node-tree during merges of clusters explained in \ref{sec:nodejoin}. The node suspension is then related to the node delay by
\begin{equation*}
    n_s = n_d - \max_{x \in \nset}{x_d}. 
\end{equation*}
The maximum node delay can be maintained during the second DFS of the node-tree, and the node suspension itself is calculated during cluster growth. A single growth iteration, which is applied in the UF decoder by adding half-edges to all boundary vertices of the cluster, is now replaced by another DFS of $\nset$. During this DFS, we calculate the suspension $n_s$ for a node, and conditionally grow it - adding half-edges to the boundary vertices in the current node and adding 1 to its radius $n_r$ --- if $n_s = 0$. This requires us to save the list of boundary vertices to each node (\Cref{fig3}c). When all $n_s$ in $\nset$ are zero, all nodes are grown simultaneously within the same iteration. 

If the node-tree does not change after a growth iteration, which is the case if no mergers occur between clusters, the node suspensions decrease in an expected manner: For all nodes that are not suspended from growth, their node suspensions decrease with 1 in the next growth iteration. Due to this behavior, we can reuse the node delays $n_d$ to calculate $n_s$ for the next growth iteration by introducing another node parameter $n_w$, the number of iterations a node has \textbf{waited}. Each time a node is suspended from growth, we add 1 to $n_w$. The node suspension in subsequent iterations is then
\begin{equation}\label{eq:suspension}
    n_s = n_d - \max_{x \in \nset}{x_d} - n_w. 
\end{equation}
Note that we have not stated which node in $\nset$ should be the root node. In fact, any node in $\nset$ could have been picked as the root of the node-tree. As long as the DFS of cluster growth is performed in the same direction as the DFSs of the parity and delay calculations. If no cluster mergers occur, the node delays can be reused in the node suspension calculation prior to node growth. 
The node-tree is constructed by storing all neighbors of a node to a list. This way, the DFSs' direction can be determined by simply saving the root node, the starting point of the DFSs, to the cluster. All node variables are depicted in \Cref{fig3}c. 
%In the next section, we expand upon this idea of "reusing" some intermediate parameters to calculate the node suspensions after a cluster merger.  


\subsection{Joining node-trees}\label{sec:nodejoin}

In the Union-Find (UF) algorithm, odd parity clusters of an odd number of non-trivial vertices, --- elements of $\sigma$ --- grow in size repeatedly and merge with other clusters until all clusters are even. During these mergers, the node-trees of the Node-Suspension data structure must also be combined. Let us now first make a clear distinction between the merging protocols of the underlying data structures; the clusters-trees of the UF data structure are merged with the \codefunc{Union} function, whereas the node-trees are merged with a separate \codefunc{Join} function. After a join of multiple node-trees, the node suspensions within the combined node-tree change. Therefore, \codefunc{Join} protocol's focus is to minimize the DFSs of the recalculation of the node parity and delays in the combined node-tree. 

First, note that as a cluster of even parity has an even number of non-trivial vertices, its node-tree has an even number of syndrome-nodes. For these even node-trees, the concept of PMW does not exist, as the matching can be made within the node-tree. Consequently, node suspension, parity, and delays are undefined when two odd node-trees join to an even node-tree. 
%Thus, if two odd clusters merge into an even cluster, we don't know and do not care about its node suspensions. 

\Figure[hbt](topskip=0pt, botskip=0pt, midskip=0pt){tikzfigs/tikz-figure4.pdf}{
    \emph{(a)} An odd cluster $\nset_o=\{n_1, n_2, n_o\}$ with root $n_1$ joins with an even cluster $\nset_e=\{n_3, n_e\}$ with root $n_3$ on nodes $n_o, n_e$, respectively, to a joined node-tree. If we choose to \emph{(b)}, make $n_e$ a child of $n_o$, the parities and delays the sub-tree of $\nset_o$ can unchanged, and we only have to perform partial parity and delay calculations over the sub-tree of $\nset_e$. If we choose to \emph{(c)}, make $n_o$ a child of $n_e$, parities and delays have to be recalculated in the entire joined node-tree. \label{fig4}}

The second type of merger is between an even and an odd cluster. The combined cluster is odd, and its growth is continued. Thus its node suspensions must be computed. Consider the example of odd node-tree $\nset_o$ and even node-tree $\nset_e$ that are to be joined on nodes $n_o\in \nset_o$ and $n_e \in \nset_e$ (\Cref{fig4}\emph{a}). If $\nset_o$'s root is kept as the root of the joined node-tree (\Cref{fig4}\emph{b}), $n_e$ is to be a child node of $n_o$. As $\nset_e$ contains an even number of syndrome-nodes, the node parities in $\nset_o$ do not change. Hence, the node parity DFS is only necessary in the sub-tree $\nset_e$, which now has $n_e$ as sub-root. Furthermore, as the node delay is only dependent on its own properties and its parent's, the node delay DFS is also only required from node $n_e$ and within the sub-tree of $\nset_e$. These so-called \textbf{partial} DFSs of the node-tree are precisely what was required, as the node parity and delays in $\nset_e$ were undefined. Alternatively, if $\nset_e$'s root becomes the root of the combined tree (\Cref{fig4}\emph{c}), an odd number of syndrome-nodes are attached to $n_e$, such that the parities of nodes on the path from $n_e$ to the root are changed. Such a join would require the DFSs on the entire combined node-tree to calculate for node parities and delays. Thus, a simple rule is always to keep the root of the odd node-tree, which we dub \textbf{Odd-Rooted Join}.

In addition, a cluster can be subjected to multiple mergers within the same growth iteration, during which the parity of the merged cluster changes dependent on the number of mergers and the parities of the clusters involved. The DFSs related to the parity and delay calculations must, for this reason, not be initiated directly after the joining of node-trees. After all, it may be possible for the cluster to merge again such that the parities and delays become invalid. To prevent these redundant calculations, sub-roots of the even sub-trees are stored to a list $\m{S}$ at the root of the node-tree (\Cref{fig3}\emph{c}). When multiple mergers occur, the root node that stores the now redundant sub-roots is replaced by a new root with new $\m{S}$. If a cluster is selected for growth, we check for the sub-roots in $\m{S}$ at the new root node and initiate the DFSs from these sub-roots. We call this the \textbf{Root List $\m{S}$ Replacement}. 

\Figure[htb](topskip=0pt, botskip=0pt, midskip=0pt){tikzfigs/tikz-figure5.pdf}{
    The node suspension values for nodes for 3 odd node-trees $\{\nset_1, \nset_2, \nset_3\}$ of 3 nodes that grow and join into a single node-tree. \emph{(a)} Node suspensions are calculated by setting $C=1$ in equation \eqref{eq:delayequation}. In step 1, the growth in each of the three node-trees' outer nodes is prioritized, and the node-trees merge. In step 2, the recalculation of the joined node-tree is performed. Parities within the sub-tree of $\nset_2$ are now inverted, and the suspension in these nodes have doubled. \emph{(b)} Node suspensions are calculated by setting $C=\nicefrac{1}{2}$. Now the increase in node suspensions after parity inversion is halved.\label{fig5}}

\subsection{Parity inversion}\label{sec:inversion}
An unfortunate effect of the Node-Suspension data structure, which we dub \textbf{Parity Inversion}, causes a decrease in the algorithm's performance as the lattice size is increased. We will demonstrate this effect through the example in \Cref{fig5}\emph{a}. Consider three instances of the node-tree of \Cref{fig0}; $\nset_a, \nset_b, \nset_c$, positioned near each other on the lattice. For each node-tree, if the middle node is suspended from growth for two iterations, all nodes have the same Potential Matching Weight. However, in the current example, the node-trees $\nset_a, \nset_b, \nset_c$ merge after 1 iteration. The combined node-tree is odd. Thus, we recalculate the node parities and delays to find that the parities in the partition of the node-tree containing the nodes of $\nset_b$ have been inverted, and the node suspensions in this partition have doubled from before node suspensions before the merger. If the next merging event occurs on the node with the doubled node suspension, the matching weight may be larger compared to the original UF decoder, which defies the goal of Node-Suspension to decrease the weight.

This defines a trade-off in the Node-Suspension data structure; a node must wait as many iterations as it is suspended to reach equilibrium in Potential Matching Weight in the node-tree, but after Parity Inversion, the node suspension for previously prioritized nodes increases linearly with the number of iterations waited by the suspended nodes pre-inversion. As a compromise, we redefine the node suspension as \textbf{half} the number of growth iterations needed for all nodes in the node-tree to reach equal PMW. This can be done by setting $C=0.5$ in Equation \eqref{eq:delayequation}. Nevertheless, as more inversions occur, the maximum node suspension in the node-tree increases, and it becomes more and more unlikely for a cluster to actually reach zero node suspension in all nodes. The number of inversions is directly related to the number of merging events, and thus the size of the lattice. The performance to improve the heuristic for minimum weight matching thus decreases for larger lattices. 


\subsection{Potential matching weight}\label{sec:matchingweight}

In the following we give some intuition into the improvement of the Union-Find Balanced Bloom decoder upon the original Union-Find decoder. Consider the cluster containing the set of edges $\m{E}$ and set of non-trivial vertices $\mathcal{V}=\{v_0,v_1,v_2\}$ in Figure \ref{fig0}. Now let us investigate the weights of a matching if an additional non-trivial vertex $v'$ is connected to the cluster. If $v'$ is connected to $v_0$ or to $v_2$, then the resulting matching has a total weight of 2: $(v',v_0)$ and $(v_1,v_2)$, or $(v_0,v_1)$ and $(v_2,v')$. However, if $v'$ is connected to vertex $v_2$, then the total weight is 3: $(v', v_1)$ and $(v_0, v_2)$. Inspired by this idea, we introduce the concept of potential matching weight (PMW) of a vertex. 

\Figure[htb](topskip=0pt, botskip=0pt, midskip=0pt){tikzfigs/tikz-figure0.pdf}{Unbalanced matching weight in cluster vertex set $\mathcal{V}$. The matching edges (dashed) correspond to the position of $v'$.\label{fig0}}

\begin{definition}\label{def:pmw}
    Consider an odd-parity cluster $c_i$ containing a vertex $v$. The Potential Matching Weight (PMW) of the vertex $v$ is the matching weight in the subset of edges of an odd-parity cluster $c_i$ in a hypothetical union with another cluster $c_j$ in the next growth iteration, where the merging boundary edge is supported by $v$. 
    \begin{equation}
      PMW(v) = \abs{\m{C} \cap \m{E}} + 1 |\text{ merge to even cluster on } v
    \end{equation}
\end{definition}
In other words, the potential matching weight is a vertex-specific predictive heuristic to the matching weight assuming a union in the next growth iteration. However, the calculation of the potential matching weight is seemingly not as straight forward, especially for clusters of increasingly larger size. Furthermore, if the potential matching weight is to be calculated for every vertex with boundary edges in all clusters in every growth iteration, the algorithm's time complexity would increase dramatically. Luckily, this calculation can be performed on a reduced graph of the cluster, which requires an additional data structure outlined in the next section. 


\subsection{Node set data structure}\label{sec:nodeset}

The calculation of the PMW requires to introduce a new data structure that we call the node-set of a cluster. The node-set of a cluster is a partition of the cluster such that each element of the partition consists of a set of adjacent vertices that are either interior elements in the associated graph or have equal PMW (Figure \ref{fig1}). The calculation of the PMW is performed by two depth-first searches of the node-set (Figure \ref{fig2}), where the PMW is translated into a node delay that determines the priority for cluster growth, such that the growth of a cluster moves towards equal PMW in the cluster (Balanced Bloom).

\Figure[htb](topskip=0pt, botskip=0pt, midskip=0pt){tikzfigs/tikz-figure1.pdf}{A node-set $\mathcal{N}$ vs. a vertex set $\mathcal{V}$, both representing the same cluster. Each shaded area covers the vertices of a different  node.\label{fig1}}

\Figure[htb](topskip=0pt, botskip=0pt, midskip=0pt){tikzfigs/tikz-figure2.pdf}{Two depth-first searches on $\mathcal{N}$ to compute node parities (head recursively) and delays (tail recursively).\label{fig2}}


To efficiently calculate the potential matching weights in a cluster, we introduce here an additional data structure, the \emph{node-tree} of a cluster, that coexists with the Union-Find data structure. We consider the case of independent noise. After syndrome identification, all identified clusters consist of a single syndrome-vertex $v_\sigma \in \sigma$. Note that with erasure noise, the initially identified clusters may be of larger size, where each connected graph of erased edges belonging to the same cluster. This set of clusters is equivalent to the syndrome set $\sigma$. Within syndrome validation, these clusters are subjected to growth and merge events with other clusters. During growth, all vertices added to some cluster $c_j$ have some closest syndrome vertex $v_\sigma$ within $c_j$ in the syndrome set $\sigma$, if a dynamic tree of the cluster is maintained. Recall from section \ref{sec:dynamicforest} that a such a cluster is always a connected acyclic graph. Even after cluster merges, newly added vertices have some closely located syndrome-vertex. The growth of a cluster can thus be interpreted to be \emph{seeded} in the syndrome vertices $v_\sigma \in \sigma$. Thus, the growth of a single cluster containing multiple syndrome vertices is related to multiple seeded growths. 

\begin{theorem}\label{the:nodepmw}
  All vertices in the subset of boundary vertices seeded in the same syndrome-vertex, $ \{v_1, v_2,...\}_{v_\sigma}$, have the same potential matching weight if a dynamic forest is maintained. 
\end{theorem}
\begin{proof}
  All vertices $v_i \in \{v_1, v_2,...\}_{v_\sigma} \subseteq \delta\vset_j$ of an odd-parity cluster $c_j$ have the same syndrome-vertex $v_\sigma$, which is located at minimum distance $d_i = |(v_i, v_\sigma)|$ on a path supported by edges in $\m{E}_j$. As a growth iteration means to grow all boundaries, all distances $d_i$ have the same value. A hypothetical matching $\m{C}$ with another odd-parity cluster on vertex $v_i$ must contain edges $(v_i, v_\sigma)$ since $c_j$ is a tree. Furthermore, $\m{E}_j\cap (\m{C} \setminus (v_i, v_\sigma))$ is independent of which vertex $v_i$ as long as they have the same seed. Thus, the potential matching weight
  \begin{equation}
    PMW(v_i) = \abs{(v_i, v_\sigma)} + \abs{\m{E_j}\cap (\m{C} \setminus (v_i, v_\sigma))} = \text{ constant } \forall v_i \in \{v_1, v_2,...\}_{v_\sigma}. 
  \end{equation}
\end{proof}

\begin{definition}\label{def:node}
  Let a node $nn$ represent a subset of vertices of a cluster for which each vertex is seeded in the same seed vertex $v_{seed}$, which is denoted as $n.\vset$ in object notation. 
\end{definition}

\begin{definition}\label{def:nodeset}
  Let a cluster $c_j$ also be represented by a \emph{node-set} $snodeset_j = \{n_1, n_2, ...\}$, stored as a tree by its root node at the cluster $c.n_r$. The subset of $n_i.\vset$ containing vertices in the boundary $\delta\vset_j$ is denoted $n_i.\delta\vset$, where $\delta\vset_j \supseteq n_i.\delta\vset \subseteq n_i.\vset$. Let the combined set of all nodes on a graph be denoted as $\nset$.
\end{definition}

Per Theorem \ref{the:nodepmw} and Definition \ref{def:nodeset}, all boundary vertices of a node have the same potential matching weight. The calculation of the potential matching weights within a cluster can thus be limited to its node-tree $\nset_j$. From our previous example, each vertex in cluster $c_e$ is a syndrome-vertex. For each of the vertices, their seed syndrome vertices are themselves. The node-tree is thus $\nset_e = \{n_1, n_2, n_3\}$ where $v_1 \in n_1.\vset$, $v_2 \in n_2.\vset$, and $v_3 \in n_3.\vset$. As this cluster grows in size, the number of vertices in $\vset_e$ increases in each round, while the number of nodes in $\nset_e$ remains the same at three nodes (Figure \ref{fig:nodesetpmw}). The node-tree is thus a \emph{reduced tree} of cluster $c_i$ where each node contains a subset of vertices in $\vset_j$, and each edge of the \emph{reduced tree} is equivalent to one or more edges in $\m{E}_j$. Furthermore, as every node needs to be seeded in some vertex, the number of nodes $|\nset|$ is limited by the number of vertices on the lattice. 
\begin{equation}\label{eq:sets}  
  \abs{\nset} \leq \abs{\vset} 
\end{equation}
% \input{tikzfigs/nodesetpmw}

\subsection{Node types}

There are various types of nodes that behave slightly differently. In this section, we introduce the \emph{syndrome-node} and the \emph{linking-node}, required for decoding on a toric code. For bounded surfaces such as the planar code, the \emph{boundary-node} additionally required, which is covered in Section \ref{sec:ufbbbound}. 

\begin{definition}\label{def:syndromenode}
  Let a syndrome-node $nsyndromenode$ denote a node that is seeded in a syndrome-vertex. 
\end{definition}

The node type that we have described in the previous section is a syndrome-node. Boundary vertices $v_i$ of a syndrome-node have a single seed syndrome-vertex for which there exists a minimum distance $d_i$, as stated in the proof of Theorem \ref{the:nodepmw}. This is true if all syndrome vertices are located an odd distance from each other. But this is not the case at all, as the distance between syndrome vertices is only limited by the discrete nature of the lattice and the size and boundary (if it exists) of the lattice itself. For two syndrome vertices $v_1, v_2$ located an even distance from each other, each seeds a syndrome-node $s_1, s_2$, and there exists some vertex $v_{l}$ that lie in equal distance to both syndromes. If the clusters of $s_1, s_2$ grow and reach vertex $v_{l}$ in the same growth iterations, it is not clear to which syndrome-node $v_l$ belongs, or which vertex $v_1$ or $v_2$ seeds $v_l$. 

\begin{definition}\label{def:linkingnode}
  Let a linking-node $nlinkingnode$ denote a node that is seeded in a vertex that lies in equal distance to two or more seeds of other nodes. 
\end{definition}

This problem is solved by initiating a linking-node $l$ with the vertex $v_l$ as its seed. Every boundary vertex of the nodes $s_1$ and $s_2$ is limited to having a single nearest syndrome-vertex, which are $v_1$ and $v_2$, respectively. For the linking-node, every boundary vertex in $l.\delta \vset$ is limited to having a single nearest \emph{linking-vertex} $v_l$, which is its seed. We can replace every instance of $v_\sigma$ in Theorem \ref{the:nodepmw} and its proof with $v_l$ to see that the theorem also holds for linking-nodes. Thus, a linking-node also has the property that its boundary vertex set $l.\delta \vset$ has the same potential matching weight. Note that a linking-node initiated on a vertex that lies in equal distance to the seeds of \emph{any} node, thus including other linking-nodes. 

Consider our example cluster $c_e$ of 3 nodes $\{n_1, n_2, n_3\}$ again. We slightly alter this cluster by increasing the distance between the seeds $v_1, v_2$ and $v_2, v_3$ to two edges. This means that cluster $c_e$ is only established after two growth iterations of the three previous separate cluster of node-trees $\{n_1\}, \{n_2\}, \{n_3\}$, and has a total size of 13 vertices (Figure \ref{fig:linkingode}). Now consider the vertices $v_{12}$ and $v_{23}$ that lie between $v_1, v_2$ and $v_2, v_3$, respectively. These are linking-vertices as they lie in equal distance to two seeds. Thus, in the node-tree of the merged cluster $\nset_e$, linking-nodes $l_{12}$ and $l_{23}$ are initiated. 

% \input{tikzfigs/linkingnode}

\subsection{Balanced-bloom}

The node-tree data structure can be utilized to delay the growth of boundaries with a high potential matching weight, or prioritize the growth of boundaries with a low potential matching weight, as the boundaries confined in each node have the same potential matching weight per Theorem \ref{the:nodepmw}. In order to do so, the cluster growth must be separated for the nodes in its node-tree. 

\begin{definition}\label{def:bloom}
  Let the \emph{bloom} of a node $n$ refer to the growth of the boundaries $n.\delta\vset$. The growth for all boundaries of a cluster $
  \delta\vset_j$ is the equivalent to the combined bloom of all nodes in its node-tree $\nset_j$. Let the radius of a node $nnradius$ be the number of iterations it has bloomed. 
\end{definition}

In search of a minimal weight matching, the growth of a cluster can thus prioritize the bloom of nodes with the lowest potential matching weight, and delay the bloom of nodes with larger potential matching weight. As these prioritized nodes bloom and increase in radius, the cluster moves towards equal potential matching weight across all nodes, where the number of delayed nodes decreases in each iteration. Once the equilibrium is reached, no nodes are delayed.
\begin{definition}\label{def:balancedbloom}
  Balanced-bloom is the state of growth of an odd-parity cluster $c_j$ when all nodes in its node-tree $\nset_j$ have the same potential matching weight, and thus all nodes in $\nset_j$ are bloomed. This state can be reached by prioritizing the growth of nodes with the lowest potential matching weight. 
\end{definition}
\begin{lemma}\label{lem:calconce}
  Between union events, the potential matching weight of nodes in a cluster need only to be calculated once. The delayed node can be queued for some iterations based on the difference of its potential matching weight and the minimal potential matching weight in the cluster.
\end{lemma}
\begin{proof}
  While no unions between clusters occur, the same set of nodes will define the cluster. The potential matching weight of nodes in the cluster is then defined by a potential matching $\m{C}$ (Theorem \ref{the:nodepmw}). The changes to the potential matching weight of a node $n_i$ due to the growth of the cluster, or some iterations of bloom, is directly related to its radius $n_i.r$. As we can store the radius as an attribute of the node, the altered potential matching weight is then simply an $\m{O}(1)$ calculation involving its old value and $n_i.r$. 
\end{proof}

The node-tree $\nset_j$ of a cluster $c_j$ is a reduced tree of the graph formed by $\vset_j$ and $\m{E}_j$, and is thus also a connected acyclic graph. The node-tree is stored as its root node $n_r \in \nset_j$ at the cluster $c_j.n_r$. As node-trees merge and linking-nodes are initiated, children nodes added to the set by connecting them to the parent nodes by \emph{undirected} edges. This is different from the vertex set $\vset$, which utilizes the Union-Find data structure (Section \ref{sec:ufdata}), which has \emph{directed} edges that point to the root. We will see in the next section why this is the case. 

\section{Node parity and delay}\label{sec:nodedelay}
The node-tree data structure allows for a reduction in the calculation of the potential matching weight, as the value for boundary vertices within the node are equal. However, if this calculation is done naively by calculating the potential matching weight for each node individually, where the entire node-tree is traversed in each calculation, the full calculation runs in quadratic time. Luckily, as we will explore in this section, the node-tree data structure allows us to calculate several values that relate closely to the potential matching weight, the \emph{node parity} and \emph{node delay}, by two depth-first searches from the root node. 

\begin{definition}\label{def:nodedelay}
  Let the \emph{node delay} $nndelay$ be the difference in the number of bloom delay iterations of a node $n$ and the root node $n_r$ in the node-tree of an odd-parity cluster.
\end{definition}

\begin{definition}\label{def:nodeparity}
  Let the \emph{node parity} $nnparity$ be an indicator for whether a node $n_\beta$ has a larger delay than its parent in an odd-parity cluster. For even parity $n_\beta.p=0$, then $n_\beta.d < n_\alpha.d$. For odd parity $n_\beta.p=1$, then $n_\beta.d > n_\alpha.d$. Even nodes are relatively prioritized, and odd nodes are relatively delayed.
\end{definition}

\begin{theorem}\label{the:delayequation}
  The node parity of a node $n_\beta$ is only dependent on its own attributes and its children $\{n_{\gamma,1}, ...\}$:
  \begin{equation}\label{eq:nodeparity}
    n_\beta.p =
    \begin{cases}
      \big( \sum_{n_\gamma} (1-n_\gamma.p) \big ) \bmod 2 \hspace{1em} | \hspace{1em} n_\gamma \text{ child node of } n_\beta & n_\beta \equiv s_\beta \\
      1 - \big( \sum_{n_\gamma} (1-n_\gamma.p) \big ) \bmod 2 \hspace{1em} | \hspace{1em} n_\gamma \text{ child node of } n_\beta & n_\beta \equiv l_\beta.
    \end{cases} 
  \end{equation}
  The node delay of a node $n_\beta$ is only dependent on its own attributes and its parent $n_\alpha$:
  \begin{multline}\label{eq:delayequation}
    s_\beta.d = s_\alpha.d + \Bigg \lceil k_{eq} \Bigg( 2\bigg(\ceil{\frac{s_\beta.r}{2}} - \floor{\frac{s_\alpha.r + s_\beta.r \bmod 2}{2}} - (-1)^{s_\beta.p}\abs{(s_\beta,s_\alpha)}\bigg)
    \Bigg) - \\
    (s_\beta.r - s_\alpha.r) \bmod 2 \Bigg \rceil \hspace{1em} | \hspace{1em} s_\beta \neq s_r,
  \end{multline}
  where $n.r$ denotes the node radius (Definition \ref{def:bloom}) and $k_{eq}$ is an optimization parameter.
\end{theorem}
\begin{proof}
  Lemmas \ref{lem:nodeparitypart} and \ref{lem:nodeparity} prove Equation \ref{eq:nodeparity}. The analyses leading up to Equations \eqref{eq:1ddelay} and \eqref{eq:2ddelay}, and Lemma \ref{lem:keq} prove Equation \eqref{eq:delayequation}.
\end{proof}

We will prove Theorem \ref{the:delayequation} throughout the following sections. In Section \ref{sec:1dnodetree}, we introduce the concept of node delays and parities on syndrome-nodes through an example of a one-dimensional node-tree. In Section \ref{sec:realisticnodetree}, the same concept is applied to realistic node-trees. These concepts are extended to linking-nodes in Section \ref{sec:linkparitydelay}. In Section \ref{sec:eqstate}, we introduce the concept of the \emph{equilibrium-state} of a node-tree that optimizes the minimal weight behavior through the $k_{eq}$ parameter. Finally, the pseudo-codes for the calculation of node delays are listed in Section \ref{sec:pdccalc}.

\subsection{One-dimensional node-tree parity and delay}\label{sec:1dnodetree}
% \input{tikzfigs/onedimensialtree}

We introduce the concepts of node parity and node delay from Definitions \ref{def:nodeparity} and \ref{def:nodedelay} through a one-dimensional node-tree $\nset_{1D}$ of exclusively syndrome-nodes. In this simplification, all nodes lie on a horizontal line from $s_1$ to $s_{|\nset_{1D}|}$ (Figure \ref{fig:1dnodetree}). Let us calculate the potential matching weights for the nodes in this cluster. Recall from Definition \ref{def:bloom} that the radius of the node $s.r$ is equal to the number of bloom iterations, one half-edge on the boundaries per iteration. This means that if a merge with some other cluster occurs on a boundary edge of $s$, the weight of the matching edges within the node $s$ is equal to $\floor{s.r/2}+1$ or. For a merge on $s_1$, the matching weight $|\m{C}|$ is the sum of $\floor{s.r/2}+1$, the length of edges $(s_2,s_3), (s_4,s_5)$, and some value $k$ corresponding to the weight of matching edges in the remainder of the cluster. This calculation can be continued for other nodes:
\begin{align*}
% \nonumber % Remove numbering (before each equation)
  PMW(s_1) &= \floor{s_1.r/2}+1 + \abs{(s_2,s_3)} + \abs{(s_4,s_5)} + k \\
  PMW(s_2) &= \floor{s_2.r/2}+1 + \abs{(s_1,s_2)} + \abs{(s_2,s_3)} + \abs{(s_4,s_5)} + k \\
  PMW(s_3) &= \floor{s_3.r/2}+1 + \abs{(s_1,s_2)} + \abs{(s_4,s_5)} + k\\
  &\vdots
\end{align*}
The difference in the potential matching weight of a node $s_i$ and its parent $s_{i-1}$ has a more constant definition that is only dependent on the radii of $s_i$,  $s_{i-1}$, and the length of the edge connecting the two:
\begin{align*}
% \nonumber % Remove numbering (before each equation)
  PMW(s_2) - PMW(s_1) &= \floor{s_2.r/2} - \floor{s_1.r/2} + \abs{(s_1,s_2)} \\
  PMW(s_3) - PMW(s_2) &= \floor{s_3.r/2} - \floor{s_2.r/2} - \abs{(s_2,s_3)} \\
  &\vdots
\end{align*}

There is a trend in which the contribution of the edge length to the difference in the potential matching weight is dependent on the \emph{parity} of the node number $i$. The difference $ PMW(s_{2i}) - PMW(s_{2i-1})$ for some integer $i$ has the positive addition of $|(s_{2i}, s_{2i-1})|$, whereas the difference $ PMW(s_{2i+1}) - PMW(s_{2i})$ has the subtraction of $|(s_{2i}, s_{2i-1})|$. Thus, we can generalize the difference as
\begin{equation}\label{eq:pmwdiff}
  PMW(s_i) - PMW(s_{i-1}) = \floor{\frac{s_i.r}{2}} - \floor{\frac{s_{i-1}.r}{2}} + (-1)^{i}\abs{(s_i,s_{i-1})} \hspace{1em} | \hspace{1em} i\geq 2.
\end{equation}

\begin{lemma}
  The difference in delay between a node $s_i$ and its parent $s_{i-1}$ is related to the difference in potential matching weight by 
  \begin{equation}\label{eq:delaydiff}
    s_i.d - s_{i-1}.d =2\big(PMW(s_i) - PMW(s_{i-1})\big) + f_{deg}(s_i.r, s_{i-1}.r) \hspace{1em} | \hspace{1em} i\geq 2,
  \end{equation}
  where $f_{deg}$ is a repair function that accounts for the degeneracy of the potential matching weight with
  \begin{equation}\label{eq:degenrepair}
    f_{deg}(R_i, R_{i-1}) = (R_i - R_{i-1}) \bmod 2 \cdot \left(\frac{R_i - R_{i-1}}{\abs{R_i - R_{i-1}}}\right) \cdot (-1)^{\left(\frac{R_i+R_{i-1}-1}{2}\right)\bmod 2} .
  \end{equation}
\end{lemma}
\begin{proof}
  As the boundary edges grow only a half-edge per bloom, the difference in the node delays between a node $s_i$ and its parent $s_{i-1}$ is thus twice the difference in their potential matching weights. But also due to this discrete multiplication factor of 2 between the delay and the potential matching weight, there is a degeneracy when calculating the potential matching weights from the node radii. For example, the radii $s_i.r = s_{i-1}.r = 2k$ for some integer $k$ yields the same potential matching weight as $s_i.r = 2k$, $s_{i-1}.r = 2k + 1$.

  The degeneracy between the node radius $R_i$ and the parent node radius $R_{i-1}$ exists only if the difference between the radii is odd. This is due to the division by 2 and the subsequent floor function. Thus, the degeneracy repair function $f_{deg}$ acts only when $(R_i - R_{i-1}) \bmod 2$ is 1. 
  
  Disregarding the length of edges between two subsequent nodes, for nodes $s_i, s_{i-1}$ with radii $R_i-R_{i-1}=1$, node $s_i$ is thus larger and should have delay $+1$ compared with node $s_{i-1}$. For radii $R_i-R_{i-1}=-1$, node $s_i$ should have delay $-1$ compared with node $s_{i-1}$. This can be simplified with
  \begin{equation}\label{eq:nodediff}
    s_i.d - s_{i-1}.d = \frac{R_i - R_{i-1}}{\abs{R_i - R_{i-1}}} \hspace{1em} | \hspace{1em} \abs{R_i - R_{i-1}} = 1.
  \end{equation}
  
  Furthermore, we find that the degeneracy is caused by a non-linearity in the difference of the potential matching weights:
  \begin{equation}\label{eq:nonlinear}
    \floor{R_i/2}-\floor{R_{i-1}/2} = 
    \begin{cases}
      \pm 2\abs{R_i - R_{i-1}} & \text{if } \left(\frac{R_i+R_{i-1}-1}{2}\right)\bmod 2 = 1 \\
      \pm 2(\abs{R_i - R_{i-1}} - 1) & \text{else}.
    \end{cases}
  \end{equation}
   The non-linearity can be accounted for by combining Equation \eqref{eq:nodediff} with the condition of Equation \eqref{eq:nonlinear} to obtain the repair function of Equation \eqref{eq:degenrepair}, which proves the lemma. 
\end{proof}

Combining Equations \eqref{eq:pmwdiff} and \eqref{eq:delaydiff}, we find that the delay of a node is defined as
\begin{multline}\label{eq:1ddelaycomp}
  s_i.d = s_{i-1}.d + 2\bigg(\floor{\frac{s_i.r}{2}} - \floor{\frac{s_{i-1}.r}{2}} + (-1)^{i}\abs{(s_i,s_{i-1})}\bigg) + \\
  (s_i.r - s_{i-1}.r) \bmod 2 \cdot \left(\frac{s_i.r - s_{i-1}.r}{\abs{s_i.r - s_{i-1}.r}}\right) \cdot (-1)^{\left(\frac{s_i.r+s_{i-1}.r-1}{2}\right)\bmod 2} \hspace{1em} | \hspace{1em} i\geq 2,
\end{multline}
which can be further simplified to 
\begin{multline}\label{eq:1ddelay}
  s_i.d = s_{i-1}.d + 2\Bigg(\ceil{\frac{s_i.r}{2}} - \floor{\frac{s_{i-1}.r + s_i.r \bmod 2}{2}} + (-1)^{i}\abs{(s_i,s_{i-1})}\Bigg) - \\
  (s_i.r - s_{i-1}.r) \bmod 2 \hspace{1em} | \hspace{1em} i\geq 2,
\end{multline}
where the repair function $f_{deg}$ has been partially moved into the main part of the function. We will not provide a description of this simplification, but Equation \eqref{eq:1ddelay} has the exact same output as Equation \eqref{eq:1ddelaycomp}. 

Using equation \eqref{eq:1ddelay}, we can calculate all the node delays in the one-dimensional node-tree by setting some initial delay for $s_1$, for example, $s_1.d=0$. This is why the node delay is defined as the difference in the bloom delay iterations between a node and the root node, which is $n_r=s_1$ in the one-dimensional node-tree. The node delay can thus also take negative values, as the choice for $s_1.d$ is arbitrary. The absolute delay, the number of iterations for a node to wait, can then be calculated by subtracting the minimum delay in the node-tree $\min \{s.d | s \in \nset_{1D}\}$. Not to mention, as the potential matching weight does not change between union events (Lemma \ref{lem:calconce}), the node delays do not have to be recalculated in every iteration. This means that it is necessary to add to the number of iterations a node has waited.
\begin{definition}\label{def:absolutedelay}
  Let $nnwait$ denote the number of bloom iterations a node $n$ has already \emph{waited}, then the \emph{absolute delay} $gls{nndelaya}$ of a node $n$ in a cluster $c_j$ with node-tree $\nset_j$ is the actual number of blooms to wait at any given moment. The absolute delay is calculated with
  \begin{equation}\label{eq:absulutedelay}
    n_i.D = n_i.d - c_j.d - n.w, 
  \end{equation}
  where $c_j.d$ is the minimum delay value in the cluster
  \begin{equation}\label{eq:cd}
    c_j.d = \min \{n.d \hspace{.5em} | \hspace{.5em} n\in \nset_j\}.
  \end{equation}
\end{definition}

Note that in Definition \ref{def:absolutedelay}, the general node element $n$ is used instead of the syndrome-node $s$. This definition also holds for other types of nodes, such as linking-nodes (Section \ref{sec:linkparitydelay}) or boundary-nodes (Section \ref{sec:ufbbbound}). The balanced-bloom state (Definition \ref{def:balancedbloom}) is thus reached when $n_i.D = 0$ for all nodes in the node-tree. 

\subsection{Realistic node-tree parity and delay}\label{sec:realisticnodetree}

The one-dimensional node-tree from the previous section does not accurately represent node-trees that occupy a real lattice. On a two-dimensional lattice (independent noise) and a three-dimensional lattice (phenomenological noise), the node-tree $\nset$ is allowed to form in the same dimensions as an acyclic graph, instead of a linear set with index number $i$. The delay calculation on an entire node tree is not a sequence of calculations from node $s_1$ to $s_{|\nset_{1D}|}$, but a depth-first search from the root node $s_r$. Just as the previous section, we assume that $\nset$ has excursively syndrome-nodes. Using the same strategy as in the previous section, we find that the equation for calculating the node delays is quite similar. The delay calculation is performed on a node $s_\beta$ comparatively with the parent node $s_\alpha$, which means that there must be some directed path within $\nset$, such that there is a clear direction, and the calculation is started from the root node $s.r$ by setting $s.r.d=0$.

The edge contribution to the node parity $|(s_\beta, s_\alpha)|$, whose sign was previously determined by the node index $i$, is now set by the node parity (Definition \ref{def:nodeparity}). 
\begin{lemma}\label{lem:nodeparitypart}
  For a node-tree of exclusively syndrome-nodes, the node parity concept can be defined as the number of descendant nodes modulo 2 (see Figure \ref{fig:parities}). It can be calculated without counting the number of descendants for every node by using the recursive relation where the parity of a node $n_\beta$ is only dependent on the parities of its immediate children $n_\gamma$:
  \begin{equation}\label{eq:nodeparitypart}
    s_\beta.p = \big( \sum_{s_\gamma} (1-s_\gamma.p) \big ) \bmod 2 \hspace{1em} | \hspace{1em} s_\gamma \text{ child node of } s_\beta.
  \end{equation}
\end{lemma}
\begin{proof}
  For a node $s_\beta$ with a set of children nodes $\{s_\gamma, ...\}$, the node parity $s_\beta.p$ can only be even if it has an even number of children nodes with even parity $s_\gamma.p = 0$, and an even number of children nodes with odd parity $s_\gamma.p=1$. This is accomplished by Equation \ref{eq:nodeparitypart}. 
\end{proof}
Note that this definition of the node parity is identical as in a one-dimensional syndrome-node-tree, where a node with an odd index effectively has an even number of descendant nodes and results in a contribution $-|(s_i, s_{i-1})|$ and an even indexed node results in a contribution $+|(s_i, s_{i-1})|$. The parity calculation thus requires the parity of every child node to be known. This means that the parity calculation of $\nset$ is related to a depth-first search from the root node $s_r$, with a tail-recursive function to calculate the parities from the bottom up. To calculate the node delays within $\nset$, a second depth-first search is applied with

\begin{multline}\label{eq:2ddelay}
  s_\beta.d = s_\alpha.d + 2\Bigg(\ceil{\frac{s_\beta.r}{2}} - \floor{\frac{s_\alpha.r + s_\beta.r \bmod 2}{2}} - (-1)^{s_\beta.p}\abs{(s_\beta,s_\alpha)}\Bigg) - \\
  (s_\beta.r - s_\alpha.r) \bmod 2 \hspace{1em} | \hspace{1em} s_\beta \neq s_r,
\end{multline}
where $n_\beta$ is the node of interest, and $n_\alpha$ is a parent of $n_\beta$, and the sign of the edge component is now dependent on the node parity $s.p$.

% \input{tikzfigs/parities}

\subsection{Linking-node parity and delay}\label{sec:linkparitydelay}

Up until now, the existence of linking-nodes has been neglected in the node parity and delays calculations. In this section, we will extend upon the previous equations for node parity and delay to include linking-nodes. Luckily, the delay calculation of Equation \eqref{eq:2ddelay} still holds for linking-nodes. However, the parity of a linking-node is calculated differently. Consider an example node-tree $\nset_s$ with five syndrome-nodes $\{s_1,...,s_5\}$ lined up linearly with distance 1 between them and $n_r = s_1$ (Figure \ref{fig:linkingparity}a). Let us consider a delay $n.d^*$ from Equation \eqref{eq:2ddelay} but leaving out the node radius components as we are now only interested in the parity component $- (-1)^{s_\beta.p}\abs{(s_\beta,s_\alpha)}$. The parity of $s_4$ is odd, therefore
\begin{equation*}
  s_4.d^* = s_3.d^* + 2(s_3, s_4).
\end{equation*}

% \input{tikzfigs/linkingparity}

Consider now a second example node-tree $\nset_l$ with three syndrome-nodes and two linking-nodes $\{s_1, l_2, s_3, l_4, s_5\}$ (Figure \ref{fig:linkingparity}b). Recall that a linking-node does not have a syndrome-vertex as seed, and thus matching must occur between seeds of the syndrome-nodes. The potential matching weights without the radius component $PMW^*$ in $\nset_l$ are
\begin{align*}
  PMW^*(s_1) &= \abs{(s_1, l_2)} + \abs{(l_1, s_3)} + \abs{(l_3, l_4)} + \abs{(l_4, s_5)} \\
  PMW^*(l_2) &= \abs{(s_1, l_2)} + \abs{(l_3, l_4)} + \abs{(l_4, s_5)} \\
  PMW^*(s_3) &= \abs{(s_1, l_2)} + \abs{(l_1, s_3)} + \abs{(l_3, l_4)} + \abs{(l_4, s_5)} \\
  PMW^*(l_4) &= \abs{(s_1, l_2)} + \abs{(l_1, s_3)} + \abs{(l_4, s_5)} \\
  PMW^*(s_5) &= \abs{(s_1, l_2)} + \abs{(l_1, s_3)} + \abs{(l_3, l_4)} + \abs{(l_4, s_5)},
\end{align*}
and the delays in $\nset_l$ are 
\begin{align*}
  l_2.d^* &= s_1.d^* + 2(l_2, s_1)\\
  s_3.d^* &= l_2.d^* + 2(s_3, l_2)\\
  l_4.d^* &= s_3.d^* - 2(l_4, s_3)\\
  s_5.d^* &= l_4.d^* - 2(s_5, l_4).
\end{align*}

\begin{lemma}\label{lem:nodeparity}
  The parity equation \eqref{eq:nodeparitypart} can be altered to apply for both syndrome-nodes and linking-nodes by   
  \begin{equation}
    n_\beta.p =
    \begin{cases}
      \big( \sum_{n_\gamma} (1-n_\gamma.p) \big ) \bmod 2 \hspace{1em} | \hspace{1em} n_\gamma \text{ child node of } n_\beta & n_\beta \equiv s_\beta \\
      1 - \big( \sum_{n_\gamma} (1-n_\gamma.p) \big ) \bmod 2 \hspace{1em} | \hspace{1em} n_\gamma \text{ child node of } n_\beta & n_\beta \equiv l_\beta.
    \end{cases}\tag{\ref{eq:nodeparity}}
  \end{equation}
\end{lemma}
\begin{proof}
  The node parities of subsequent syndrome-nodes in a node tree should be independent on the number of intermediate linking-nodes, as the matching only occurs between the syndrome-vertex seeds of the syndrome-nodes. The parities of the intermediate linking-nodes should thus satisfy this requirement. By applying 1 minus the definition for the parity of a syndrome-node, the parity of the nearest descendant syndrome-node is effectively passed on to the linking-node, so that the parity flip only occurs at the next syndrome-node when moving upwards in the node-tree. 
\end{proof}

\subsection{Node tree ancestry}
Recall from the last paragraph of Section \ref{sec:nodeset} that the edges of the node tree are \emph{undirected}. However, the depth-first searches to calculate the node parities and delays indicate that there is some ancestry in the node tree. In this section, we will clarify this feature of the node-tree. 
\begin{lemma}\label{lem:anynoderoot}
  Any node $n_i \in \m{N}$ is a valid root. The root $n_r$, which has parity $n_r.p=0$, determines the node parities within the node-tree. 
\end{lemma}
\begin{proof}
  Since the node parities are calculated from the descendants to the root and the node delays are subjected to an arbitrarily chosen delay for the root $n_r.d$, any node in $\nset$ can be chosen as the root. Recall from Definition \ref{def:nodeparity} that the node parity is only defined for an odd-parity cluster. For a node-tree of exclusively syndrome-nodes, $n_r$ must have an even number of descendant nodes, and thus per Lemma \ref{lem:nodeparitypart}, it must be that $s_r.p=0$. From a node-tree of mixed syndrome-nodes and linking-nodes, recall from Lemma \ref{lem:nodeparity} that linking-nodes always copies the parity of the nearest descendant syndrome-node, thus $n_r.p=0$. Choosing which node $n_i \in \nset$ is the root node $n_r$, for this reason, determines the parities in the node-tree (see Figure \ref{fig:parities}). 
\end{proof}

The node-tree has undirected edges, such that it is not set in stone which node is the root. When to clusters merge into one, their respective node trees need to be merged too. As the edges in the node-tree reflect one or many physical edges on the lattice, the merge of node-trees can not be applied by simply pointing one root to another, such as in the Union-Find data structure. Instead, the node-trees $\nset_i, \nset_j$ are joined on the nodes $n_i \in \nset_i, n_j\in \nset_j$ containing the boundary vertices that support the newly grown edge that links the clusters. This can be done by setting one of the nodes $n_i$ or $n_j$ as the \emph{subroot} if its tree, and connecting it with the other. This motivates the use of undirected edges. New roots can be chosen that allow for the union of node-trees. More on the union of node-trees is described in Section \ref{sec:nodejoin}. 

\begin{lemma}\label{lem:nodecalc_ancestrypath}
  The calculated node delays $n_i.d$ are only valid, while node parities have been calculated with the same root node $n_r$. The absolute delay $n_i.D$ is independent of the selected root node. 
\end{lemma}
\begin{proof}
  Since both the calculation of the node parities and node delays are performed by a depth-first search of the node tree, and the node parities are dependent on which node is set as root (Lemma \ref{lem:anynoderoot}), it is trivial that the node delay calculation should follow the same depth-first search as the parity calculation. The absolute delay $n_i.D$ is independent of the root node, as it is the node delay $n_i.d$ minus the minimal delay in the cluster $c.d$ (Definition \ref{def:absolutedelay}). Recall that the node delay value is the difference in delay with the root node $n_r.d$, whose value is arbitrary. By subtracting the minimal delay value in the cluster, this arbitrariness is accounted for. 
\end{proof}


%  An interesting aspect of the node delays is that the differential delays $\delta(n.d)$ are indifferent for which node is set as root $n_r = n$. The root delay value $n.d$ however may differ for different roots as de delay value for the root node is arbitrary. But as we subtract by the minimal delay $C.d$ to find the normalized delay, the root dependance of node PMW and node PNW is accounted for. This fact strengthens Lemma \ref{lem:anynoderoot}.

\subsection{Equilibrium optimization}\label{sec:eqstate}

In this section, we alter the delay equation \eqref{eq:2ddelay} with a new parameter $k_{eq}$ to optimize a trade-off in this algorithm. This trade-off occurs in about $50\%$ of the node-tree unions in the event that we dub \emph{parity-inversion}. Recall from Lemma \ref{lem:calconce} that after a union, the potential matching weight within the node-tree changes, and the parities and delays may have to be recalculated. We will describe in Section \ref{sec:growingcluster} necessary steps to grow a cluster with the node-tree data structure, and in Section \ref{sec:nodejoin}, we describe how to merge node-trees. In this section, the focus is on what happens to the potential matching weight and the subsequently required recalculation of the node parities and delays. 

When clusters grow in size, their nodes are delayed such that the equilibrium in the potential matching weight can be reached. Because of this, the prioritized nodes have larger radii than the delayed nodes. As clusters merge, their node-trees are also joined on the nodes that contain the vertices supporting the connecting edge. Due to the merges, the parities of nodes in parts of the joined node-tree may flip. 
\begin{definition}\label{def:parityinversion}
  Parity inversion is the event of that the parities within a part of a node-tree flip, which may happen as a result of the merging of multiple node-trees. 
\end{definition}
This means that the previously prioritized nodes become the nodes to be delayed, and the previously delayed nodes are to be prioritized. As these nodes have already grown in different radii, the parity inversion causes that after the flip in priority, it takes twice as many iterations to reach the equilibrium in potential matching weight. As more and more unions occur, the number of parity inversions increases, and so does the number of iterations needed to reach equilibrium. 

\begin{definition}\label{def:eqstate}
  The equilibrium-state $(I:M)$ of cluster describes the degree of potential matching weight equilibrium in the cluster with node-tree $\nset$, where $M$ is the number of iterations with delayed blooms needed to reach equal potential matching weight, and $I\leq M$ is the number of iterations grown while equal potential matching weight has not been reached (Figure \ref{fig:eqstate}). The $(M:M)$ equilibrium-state is maximally occupied when all nodes in the node-tree have equal potential matching weights, which is equivalent to the balanced-bloom state of Definition \ref{def:balancedbloom}. 
\end{definition}
% \begin{figure}
%   \centering
%     \begin{tikzpicture}
%       \DSPECTRUM{4}{2}{1}
%       \draw (-1.5,.5) node[align=right] {Unbalanced} ++(6.6,0) node[align=left] {Balanced};
%     \end{tikzpicture}
%   \caption{Visual representation of the equilibrium-state $(2:4)$. The size of the full x-axis is $M=4$ and the length of the bar is $I=2$. The left side of the spectrum is equivalent to the unbalanced equilibrium-state, and the right the balanced state.}\label{fig:eqstate}
% \end{figure}
For example, a cluster with $M=4$ requires 4 growth iterations to reach an equilibrium in potential matching weight in all nodes in the cluster. The equilibrium-state thus gives us an indication of how near balanced-bloom a cluster performs. 
\begin{lemma}\label{lem:eqstate}
  Let $(I_t, M_t)$ denote the equilibrium-state of a cluster just before a union with another cluster that causes a parity inversion, and $(I_{t+1}, M_{t+1})$ the equilibrium-state after the union, then $I_t \propto M_{t+1}$.
\end{lemma}
\begin{proof}
  In the context of the equilibrium-state, the delayed bloom of nodes in cluster growth is equivalent to increasing the value of $I$ in the equilibrium-state. As $I_t\to M_t$, the difference between the radii of the prioritized and delayed nodes increases. Thus, the iterations $M_a$ needed after the union and parity inversion also increases. 
\end{proof}
Subsequent parity inversions cause a gradual but certain increase in $M$ of the equilibrium-state, depending on $(I_t:M_t)$ during the parity inversion at the union, requiring a growing number of growth iterations $I_{t+1}$ to reach the equilibrium-state $(M:M)_{t+1}$. As the lattice size is increased, the total number of unions of a cluster with other clusters also increases, leading to a growing number of parity inversions. Thus increasing the lattice size has the consequence that more growth iterations $I$ are needed to reach equilibrium-state $(M:M)$. This is the trade-off in the effectiveness of this algorithm. On the one hand, it is preferred that $I\to M$ to maximally occupy the equilibrium-state that is a heuristic for minimum-weight, but on the other, $I$ is also proportional to the number of iterations needed to actually reach $(M:M)$ due to parity inversions. 

\begin{definition}\label{def:keq}
  Let the \emph{equilibrium factor} $zkeq\in [0,1]$ be a target factor $I/M$ to the node delay. 
\end{definition}
\begin{lemma}\label{lem:keq}
  The delay equation where the delays have a factor $k_{eq}\in [0,1]$ minimizes the trade-off caused by parity inversion. 
  \begin{multline}
    s_\beta.d = s_\alpha.d + \Bigg \lceil k_{eq} \Bigg( 2\bigg(\ceil{\frac{s_\beta.r}{2}} - \floor{\frac{s_\alpha.r + s_\beta.r \bmod 2}{2}} - (-1)^{s_\beta.p}\abs{(s_\beta,s_\alpha)}\bigg)
    \Bigg) - \\
    (s_\beta.r - s_\alpha.r) \bmod 2 \Bigg \rceil \hspace{1em} | \hspace{1em} s_\beta \neq s_r. \tag{\ref{eq:delayequation}}
  \end{multline}
\end{lemma}
\begin{proof}
  For any $k_{eq} < 1$, a cluster will never actually reach the $(M:M)$ equilibrium-state, only $(k_{eq}M:M)$. Consequently, after a parity inversion, the difference in node radii between prioritized and delayed nodes is decreased, such $I\to k_{eq}M$ can be reached in a lower amount of growth iterations. 
\end{proof}

In Figure \ref{fig:kbloom} and \ref{fig:kbloom2}, a comparison is made between the growth of a set of node-trees using Equation \eqref{eq:delayequation} with $k_{eq}=1$ (same as Equation \eqref{eq:2ddelay}) and with $k_{eq}=1/2$. Here, the equilibrium-state is defined as $(I:k_{eq}M)$. We see that the number of iterations needed to maximally occupy the equilibrium state using $k_{eq}$ is halved before and after the union with parity inversion when using $k_{eq} = 1/2$. 

The optimal value of $k_{eq}$ may be dependent on the number of parity inversions, the lattice size, growth iteration, and the node-tree and cluster sizes $|\nset|, |\vset|$, with the goal of maximally occupying the equilibrium-state after the last parity inversion. We suspect that because $M$ doubles after parity inversion, a constant factor of $k_{eq}=1/2$ should behave well on average, as the equilibrium state is occupied half on average. However, other values of $k_{eq}$ should be explored, and optimizations dependent on these variables could be possible. 

% \input{tikzfigs/equilibrium_state}
 
\subsection{Parity and delay calculations}\label{sec:pdccalc}

With equation \eqref{eq:nodeparity} and \eqref{eq:delayequation}, we now finally have the tools to formulate the algorithms to calculate the node parities and delays. For a node-tree with root $n_r$, we can calculate the parities by calling the \emph{head recursive} function \codefunc{Calcparity} on $n_r$ in Algorithm \ref{algo:calcparity}, where we perform a reverse depth-first search of the node-tree. The node delays are calculated by calling the \emph{tail recursive} function \codefunc{Calcdelay} in Algorithm \ref{algo:calcdelay}, where we perform a second depth-first search of the node-tree. This parity and delay calculation will from this point sometimes be abbreviated as PDC. A schematic of the directions of these calculations in an example node-tree is included in Figure \ref{fig:2dfs}.

% \input{pseudocodes/calcparity}
% \input{pseudocodes/calcdelay}
% \subsection{Node parity, delay and suspension}\label{sec:paritydelaysus}

The Node-Suspension data structure allows for the calculation of the node suspension of all nodes in a node-tree $\nset$ by two intermediate steps. In each step a depth-first-searches (DFS) of $\nset$ is applied, such that the calculation can be performed in linear time to the node-tree dimension. In the first DFS, we calculate for the \textbf{node parity} - the number of child syndrome-nodes of a node modulo 2 - $np$ via a tail recursive function, which is only dependant on the node parities of the children nodes $\{n_{\gamma,1}, ...\}$ of a node $n_\beta$:
\begin{equation}\label{eq:nodeparity}
np_\beta =
\begin{cases}
    \big( \sum_{n_\gamma} (1-np_\gamma) \big ) \bmod 2 \hspace{1em} | \hspace{1em} n_\beta \equiv sn_\beta \\
    1 - \big( \sum_{n_\gamma} (1-np_\gamma) \big ) \bmod 2 \hspace{1em} | \hspace{1em} n_\beta \equiv jn_\beta.
\end{cases} 
\end{equation}

In the second DFS, we calculate for the difference in node suspension $\delta ns$ of a node $n_\beta$ with its parent $n_\alpha$. We can choose an arbitrary \textbf{node delay} $nd$ for the root node such as $nd_r=0$ and add the difference $\delta ns$ during each step to obtain $nd$ for every node. This node delay of a node $n_\beta$ is only dependent on the node radii of itself $nr_\beta$ and its parent $nr_\alpha$, the length of node-tree edge $(n_\beta, n_\alpha)$ and its parity $np_\beta$. 
\begin{multline}\label{eq:delayequation}
    nd_\beta = nd_\alpha + \Bigg \lceil 2k_{eq}\bigg(\ceil{\frac{nr_\beta}{2}} - \floor{\frac{nr_\alpha + nr_\beta \bmod 2}{2}}\\
    - (-1)^{np_\beta}\abs{(n_\beta,n_\alpha)}\bigg) - (nr_\beta - nr_\alpha) \bmod 2 \Bigg \rceil
\end{multline}
Here, the \textbf{equilibrium factor} $k_{eq}$ is a constant that deals with the inversion of node parities in a node-tree during merges of clusters, detailed in Section \ref{sec:nodejoin}. 

\Figure[htb](topskip=0pt, botskip=0pt, midskip=0pt){tikzfigs/tikz-figure3.pdf}{Two depth-first searches on $\mathcal{N}$ to compute node parities (head recursively) and delays (tail recursively).\label{fig3}}

There is a final step in calculating the node suspension from the node delay, which are related by
\begin{equation}
    ns = nd - \max_{nd_i \in \nset} nd_i - nw, 
\end{equation}
where $nw$ is equal to the number of iterations a node has \textbf{waited} or has been suspended from growth. The maximum node delay can be maintained during the second DFS of the node-tree. The node suspension itself is calculated dynamically during the growth of the cluster. 

A single growth iteration, which is applied in the original UF decoder by adding half-edges to all boundary vertices of the cluster, is now replaced by another DFS of $\nset$. During this DFS, a node is conditionally bloomed - adding half-edges to the boundary vertices in the current node and adding 1 to the node radius $nr$ - if $ns$ is equal to zero. If not, add 1 to $nw$ and continue with the DFS. A subsequent growth iteration now does not require the two DFS related to the calcualtion of $np$ and $nd$, provided that no union between clusters has occured. The Node-Suspension data structure enables us to calculate the node suspension across multiple growth iterations based on a single calculation of the node parity and delay. When all $ns$ in $\nset$ are zero, all nodes are bloomed simultaenously within the same iteration. This process of reaching equal $ns$ or PMW in the cluster is called the \textbf{Balanced Bloom} of nodes. 

Note that we hadn't stated which node in $\nset$ should be the root node. In fact, any node in $\nset$ could have been picked as the root of the node-tree. This property, together with the constancy of $np$ and $nd$ in betwen cluster unions, allows us to define a set of rules for the merges of node-trees.  

\section{Growing a cluster}\label{sec:growingcluster}
With the knowledge of the previous section, we now have the equations and algorithms available to describe the steps to grow a cluster in the context of Union-Find Balanced-Bloom. Previously, in the Union-Find decoder, a cluster is grown with $\codefunc{Grow}(c_j, \m{L}_m)$ (Algorithm \ref{algo:ufgrow}). Here, the boundary edges connected to the vertices in $c_j.\delta\vset$ are grown by increasing the value of $e.support$. If $e.support = 2$, $e$ is added to the merging list $\m{L}_m$ to merge the vertex-trees at some later moment. 

In the node-tree data structure, the growth of a cluster is equivalent to a depth-first search of the node-tree, which will now be performed by $\codefunc{Ngrow}$ (Algorithm \ref{algo:bbgrow}). The boundary list for each cluster is not stored at the cluster $c_j$, but separately stored at each of the nodes $n_i$ in $\m{N}_j$ by $n_i.\delta\vset$. We travel to all $n_i \in \m{N}_j$ from the root $n_r$ and apply $\codefunc{Bloom}(n_i)$ (Algorithm \ref{algo:bloom}), which grows the boundaries for each node individually. Again, if an edge on the boundary are grown to $e.support = 2$, $e$ is added to the merging list $\m{L}_m$. Elements of $\m{L}_m$ are then iterated over to merge the vertex-trees and node-trees at some later moment. The merging of node-trees is considered in Section \ref{sec:nodejoin}. 

Recall from Theorem \ref{def:balancedbloom} that with Balanced-Bloom, the bloom of node with the lowest potential matching weight in the cluster are prioritized, whereas the bloom of other nodes are delayed. Also, from Lemma \ref{lem:calconce}, in the absence of unions, the delays are not recalculated after every growth iteration, but stored in memory at the nodes. Definition \ref{def:absolutedelay} introduced the absolute delay $n_i.D$, where the actual number of iterations to delay is updated via the minimal delay value $c_j.d$ in the cluster, and the number of iterations already waited for $n.w$. Thus, when performing the depth-first search in \codefunc{Ngrow}, a node should be conditionally bloomed if only $n_i.D = 0$ is satisfied. If not, node $n_i$ is skipped, the wait $n.w$ is increased, and the depth-first search continues recursively on its children nodes. 

% \input{pseudocodes/bloom}
% \input{pseudocodes/ngrow}

\section{Joining node-trees}\label{sec:nodejoin}
Within the vertex-tree $\m{V}$, which utilizes the Union-Find data structure, \emph{path compression} and \emph{union by weight} or \emph{union by rank} are applied to minimize the depth of the tree. These rules minimize the calls to the \codefunc{Find} function. Similarly, in the node-tree $\m{N}$, we would also like to apply a set of rules to reduce the calls to \codefunc{Calcparity} and \codefunc{Calcdelay}, which we will dub the parity and delay calculation minimization.

\begin{definition}\label{def:partialpdc}
  A partial parity or partial delay calculation, which will often be abbreviated to a partial calculation, is associated with a depth-first search that is not initiated from the root node $n_r$, but some descendant node of $n_r$ in the node-tree. 
\end{definition}

This minimization is achieved by preserving the node parities and delays in subsets of the merged node-tree after union, and applying a partial calculation of the parities and delays in the remaining subsets if required. The recursiveness of both \codefunc{Calcparity} and \codefunc{Calcdelay} (Algorithms \ref{algo:calcparity} and \ref{algo:calcdelay}) ensures that this is possible. The tail-recursive parity calculation stops at the node where the depth-first search is started, and the head-recursive delay calculation now has a non-arbitrarily node delay. 

With the addition of the node-tree data structure, during the merge of clusters $c_\alpha$ and $c_\beta$, we have to additionally merge the node-trees $\m{N}_{\alpha}$ and $\m{N}_\beta$ that require its own set of rules that we will explain in this section. Let us first make a clear distinction between the various methods. For the merge of vertex-trees $\vset_\alpha, \vset_\beta$ we apply $\codefunc{Union}(v_\alpha, v_\beta)$ (Algorithms \ref{algo:unionweight} or \ref{algo:unionrank}), with the two vertices spanning the edge connecting two clusters as arguments. For the merge of node-trees $\nset_\alpha, \nset_\beta$, we introduce here $\codefunc{Join}(n^\alpha, n^\beta)$ (Algorithm \ref{algo:join}), which is called on the two nodes $n_\alpha, n_\beta$ that seed vertices $v^\alpha, v^\beta$, respectively. During a merge of two clusters, these routines are both applied to their respective sets. Within the context of the Union-Find Balanced-Bloom decoder, when either one of the expressions ``merge clusters $C_\alpha$ and $C_\beta$'', ``the union of vertex-trees $\m{V}_\alpha$ and $\m{V}_\beta$'' or the ``join of node-trees $\m{N}_{\alpha}$ and $\m{N}_\beta$'' is mentioned, it is always implied that both routines are executed.

\begin{definition}\label{def:nodesetparity}
  Let the parity of a node-tree be the number of syndrome-nodes in the node-tree modulo 2. The parity of the node-tree is thus equivalent to the parity of its cluster. 
\end{definition}
Let us categorize the joins of two node-trees into two types: even-joins and odd-joins, depending on the parity of the node-tree after the join. 
\begin{definition}\label{def:oddevenjoin}
  An even-join may be the result of the join of two even node-trees or two odd node-trees, whereas an odd-join is the result of the join of one odd node-tree and one even node-tree.
\end{definition}

\begin{lemma}\label{lem:nodecalc_even}
  If node-trees merge into an even node-tree $\nset^e$, all node parities and delays within $\nset^e$ become invalid or \emph{undefined}. 
\end{lemma}
\begin{proof}
  Recall from Definitions \ref{def:nodeparity} and \ref{def:nodedelay} that the node parity and delay are only defined for odd-parity clusters. An even-parity cluster does not have a potential matching weight, as the matching within the cluster is already defined. However, $\nset^e$ can merge with another odd-parity cluster with node-tree $\nset^o$ in a larger odd-join. In that case, we might think about ``reusing'' some node parities and delays that were already calculated in $\nset^e$. To reuse prior calculated parities and delays, a depth-first search on $\nset^e$ is needed to find which sections are still valid, and which sections are not. This is especially the case when the clusters in the even-join are the results of joins within the same growth iteration. Checking the validity to reuse prior parities and delays then acquires the same complexity as redoing the calculation of parity and delays over the subtree $\nset^e$. Hence, the node parities and delays in the joined set after an even-join are \emph{undefined}.
\end{proof}

\begin{lemma}\label{lem:nodecalc_odd}
  Consider an odd-join on nodes $n_j^e \in \nset^e, n_j^o\in \nset^o$, belonging to an even and an odd node-tree, respectively. Parity and delay calculations are minimized if the node-trees are always joined by setting $n_j^e$ as the child of $n_j^o$. 
\end{lemma}
\begin{proof}
  If $n_j^e$ is made a child of $n_j^o$, $n_j^e$ is the new subroot of subtree $\nset^e$, and an even number of syndrome-nodes are now descendants of $n_j^o$, and parities within $\nset^o$ and its root are unchanged. Recall from Lemma \ref{lem:nodecalc_ancestrypath} that thus the delays in $\nset^o$ are also unchanged. A partial parity and delay calculation can now be initiated from $n_j^e$ and is proportional to $|\nset^e|$ (Figure \ref{fig:joinrules}b). If $n_j^o$ is made a child of $n_j^e$, an odd number of syndrome-nodes are descendants of $n_j^e$ and change the parities in the ancestors of $n_j^o$ up to the root of the joined tree. The parities and delays now need to be recalculated in the entire tree, which is proportional to $|\nset^e| + |\nset^o|$ (Figure \ref{fig:joinrules}c). 
\end{proof}

% \input{tikzfigs/oddevenjoin}

From Lemmas \ref{lem:nodecalc_even} and \ref{lem:nodecalc_odd}, we can define a simple rule that determines how node-trees are joined.

\begin{definition}\label{def:joinbyparity}
  Let the \emph{join by parity} rule govern how to join node-trees in the event of clusters merging. For even-joins between two even or two odd node-trees, the parent and child node-trees can be picked at random. For odd-joins between nodes $n_j^e \in \nset^e, n_j^o \in \nset^o$, always make the even node-tree a child of the odd node-tree, where $n_j^e$ is now the subroot of the subtree $\nset^e$.
\end{definition}
The \emph{join by parity} rule ensures that the parities and delays in $\nset^o$ are preserved and that only a partial calculation, equivalent to the depth-first search from node $n_j^e$, is needed. Note the concept of a \emph{partial} calculation is somewhat redundant. Using these rules for the joins of node-trees, the parity and delay calculations are never calculated on a full node-tree, except for the initial round. 

Recall from Definition \ref{def:nodeset} that the node-tree $\nset_j$ of cluster $c_j$ is stored as its root node at $c_j.n_r$, which sets the ancestry in the node-tree. In a join of two node-sets, the \emph{join by parity} rule requires to conditionally set the ancestry in the joined node-set. This can simply be done by connecting the node-trees with a new edge, and selecting the correct root node to be stored in the merged cluster (see Algorithm \ref{algo:join}). Also, due to the use of undirected edges, it is required to store the direction of the partial parity and delay calculation.

\begin{definition}
  Let us make a distinction between the \emph{final odd-join} between an odd node-tree $'\nset^o$ and an even node tree $'\nset^e$ to a joined node-tree $\nset$, and all others odd-joins that joined to $'\nset^e$ within the same round which we dub \emph{intermediate odd-joins}. 
\end{definition}

\begin{lemma}\label{lem:delaywhengrown}
  Redundant partial parity and delay calculations over even subtrees in intermediate odd-joins are prevented by applying the calculation directly before the growth of the cluster. 
\end{lemma}
\begin{proof}
  Consider the case when partial delay and parity calculations are initiated from a node $n_j^e \in \nset^e \subset \nset$ directly after the join of $\nset^e$ and $\nset^o$ to the joined node-tree $\nset$ while applying the \emph{join by parity} rule of Definition \ref{def:joinbyparity}. If there are many odd-joins (and even-joins) within the same round of growth, that at the end of round all joins to a single cluster with node-tree $\nset$, every odd-join will require the partial calculation over the even subtree. There may thus be many even subtrees where multiple partial calculations are performed within the same round before the final cluster $\nset$ is constructed. All but the final calculation will lead to the correct parities and delays in $\nset$. To circumvent any redundant calculations on the even subtrees of intermediate odd-joins, the partial calculation is suspended as much as possible, until just before a cluster is grown.
\end{proof}

Consider an example with five odd node-trees $\nset_1, ...,  \nset_5$ (Figure \ref{fig:redundantpdc}) that join to a single node-tree, where the partial calculation is applied directly after each join. The join of $\nset_1$ and $\nset_2$ to $\nset_{12}$ is an even-join and requires no partial calculation. The join of $\nset_{12}$ and $\nset_3$ is an odd-join, and we apply partial calculations in $\nset_{12}$. The join of $\nset_{123}$ and $\nset_4$ is an even-join and the join of $\nset_{1234}$ and $\nset_5$ is an odd-join, with partial calculations in $\nset_{1234}$. The earlier computation in $\nset_{12}$ is thus redundant. 

% \input{tikzfigs/partialcalculations}

The only task now is to store the subroot of the even subtree $n_j^e \in '\nset^e$ of the final odd-join, as this subroot is the starting point of the depth-first searches of the partial parity and delay calculation. For every odd-join between odd node-tree $'\m{N}^o$ and even node-tree $'\m{N}^e$ on nodes $'n_j^o, 'n_j^e$ to a cluster $c_j$, store the subroot $'n^e_j$ at the cluster as the \emph{undefined node subroot} $c_j.u$ (Algorithm \ref{algo:join}). If $c_j$ is selected for growth, and has an undefined node subroot $c_j.n_u$, we apply $\codefunc{Calcparity}(c_j.n_u)$ and $\codefunc{Calcdelay}(c_j.n_u)$ (Algorithms \ref{algo:calcparity}, \ref{algo:calcdelay}) to calculate parities and delays in undefined subtrees. We then call $\codefunc{Bloom}(c_j.n_r)$ (Algorithm \ref{algo:bloom}) to grow the cluster. 

% This data structure dynamically saves the root of the undefined part of a cluster to the root node. For any IO-join, we don't know yet whether another O-join will occur, thus each IO-join to cluster $''\nset^o$ is treated as a FO-join. For a IO-join, we thus also store the undefined subroot $u_1$ at the root $R_1=''n_R_{-1}$. If $''\nset^o$ joins with other clusters in subsequent E-join to cluster $'\nset^e$ and lastly the ``real final'' FO-join with $'\nset^o$ to $\nset^o$, we again store the undefined subroot $u_2='n_r^e$ at the new root of $R_2='n_R_{-1}$. Due to Lemma \ref{lem:nodecalc_odd}, it is certain that $u_2$ is an ancestor of $u_1$, and the PDC will traverse over all undefined regions of the set.

% \begin{theorem}\label{the:delayonce}
%   Undefined region of an odd cluster $\nset^o$ is defined as the subroot $u$ for which all children nodes including $u$ have undefined parities and delays, and is stored at root node $n^o_r$. PDC is performed for $n^o_r.u$ and its children before cluster $\nset^o$ is grown.
% \end{theorem}
% \input{pseudocodes/join}

\section{Pseudo-code}
Now we have the full description of the modification of the Union-Find decoder, which we dub the \emph{Union-Find Balanced-Bloom} decoder. Recall from Theorem \ref{the:nodepmw} that the potential matching weight is only defined if a dynamic forest of clusters is maintained. Recall also from Section \ref{sec:ufperformance} that weighted growth improves the code threshold of the Union-Find Decoder. Thus, the modification will be applied to the Dynamic-forest Bucket Union-Find decoder of Algorithm \ref{algo:dbuf}. 

In the Union-Find Balanced-Bloom decoder of Algorithm \ref{algo:ufbb}, partial parity and delay calculations are applied if a cluster $c_i$ has an undefined node subroot $c_i.n_u$, and \codefunc{Grow} (Algorithm \ref{algo:ufgrow}) is replaced with \codefunc{Ngrow} (Algorithm \ref{algo:bbgrow}). Furthermore, when iterating over the edges of the merging list $\m{L}_m$, if the vertex-tree roots of the supporting vertices do not belong to the same cluster, it either means that a new vertex is added to the cluster, or that two clusters are merged. In the first case, the new vertex is added to the node, whereas two node-trees are joined in the second case. To differentiate between these cases, we need to additionally store the node $n$ containing the vertex $v\in v.\vset$ at the vertex as $v.n$. With this data structure, two node-trees have to be joined on $v.n$ and $u.n$ if they both exist. Otherwise, the node is to be saved to the newly added vertex. 

% \section{Union-Find Node-Suspension decoder}\label{sec:ufbb}

In this section, we describe the \emph{Union-Find Node-Suspension} decoder, which increases the Union-Find decoder's performance by improving its heuristic for minimum-weight matching. We first introduce the concept of the potential matching weight in \Cref{sec:matchingweight}. We describe the data structure required for this decoder in \Cref{sec:nodeset}, and the necessary calculations performed on this data structure in \Cref{sec:paritydelaysus,sec:nodejoin,sec:inversion}. The pseudocode is included in \Cref{sec:pseudocode}. 

\Figure[htb](topskip=0pt, botskip=0pt, midskip=0pt){tikzfigs/tikz-figure0.pdf}{
    A cluster with vertices $\{v_0, v_1, v_2\}$ with potential matching weights $\{2, 3, 2\}$. The line style and color of the colored edges correspond to the matching in the hypothetical union with an external vertex $v'$ of the same line style and color.\label{fig0}}

\subsection{Potential matching weight}\label{sec:matchingweight}
%In the following we give some intuition into the improvement of the Union-Find Balanced Bloom decoder upon the original Union-Find decoder. 
% We compared the ratio of the matchings between the MWPM decoder and our own implementation of the UF decoder, averaged over many simulations, and found that UF matching weight has a constant prefactor of $\sim 1.043$ over the minimum weight for the toric code (\Cref{comp_weight}). From this, we suspected that a decreased matching weight is a heuristic for an increased threshold. Within the context of the UF decoder, the matching weight may be decreased by prioritizing the growth of vertices with low PWM's within the cluster. 

Consider the cluster with index $i$ containing the set of non-trivial vertices $V_i=\{v_0,v_1,v_2\}$ and set of edges $E_i=\{(v_0,v_1), (v_1, v_2)\}$ of \Cref{fig0}. Now let us investigate the weight of a matching if an additional non-trivial vertex $v'$ is connected to the cluster. If $v'$ is connected to $v_0$ or to $v_2$, then the resulting matching has a total weight of 2: $(v',v_0)$ and $(v_1,v_2)$, or $(v_0,v_1)$ and $(v_2,v')$. However, if $v'$ is connected to vertex $v_2$, then the total weight is 3: $(v', v_1)$ and $(v_0, v_2)$. Inspired by this idea, we introduce the concept of potential matching weight (PMW) of a vertex. 

\begin{definition}\label{def:pmw}
    % For, the hypothetical merger with another odd-parity cluster $V_j, E_j$ on the edge $(v_i, v_j)$, with $v_i\in V_i$ and  $v_j \in V_j$, outputs an even-parity cluster with edges $E_{ij} = E_i \cup E_j \cup (v_i, v_j)$ in which there exists a matching $\m{C}_{(v,v')} \subseteq E_{ij}$ between syndromes internal to the cluster.
    % Let the Potential Matching Weight (PMW) of vertex $v_\alpha \in V_\alpha$ in an odd-parity cluster $\alpha$ with vertices $V_\alpha$ and edges $E_\alpha$ be
    % \begin{equation}
    %   PMW(v_\alpha) = \abs{\m{C}_{(v_\alpha,v_\beta)} \cap E_\alpha} + 1,
    % \end{equation}
    % where matching $\m{C}_{(v,v')} \subseteq E_{ij}$ is between the syndrome vertices internal to the even-parity cluster with edges $E_{ij} = E_\alpha \cup E_\beta \cup (v_\alpha, v_\beta)$, after a hypothetical merger of cluster $\alpha$ with another odd-parity cluster $V_\beta, E_\beta$ on the edge $(v_\alpha, v_\beta)$, with $v_\alpha\in V_\alpha$ and  $v_\beta \in V_\beta$
    Let there be a hypothetical merger between odd cluster $\alpha$ of vertices $V_\alpha$ and edges $E_\alpha$, and odd cluster $\beta$ of $V_\beta$ and $E_\beta$, on the edge $(v_\alpha, v_\beta)$, where $v_\alpha \in V_\alpha$ and $v_\beta \in V_\beta$. In the merged even cluster with edges $E_{\gamma} = E_\alpha \cup E_\beta \cup (v_\alpha, v_\beta)$, there is a matching $\m{C}_{(v,v')} \subseteq E_{\gamma}$  between the syndrome vertices internal to the cluster. The \textbf{Potential Matching Weight} (PMW) of vertex $v_\alpha$ is then defined as
    \begin{equation}
      PMW(v_\alpha) = \abs{\m{C}_{(v_\alpha,v_\beta)} \cap E_\alpha} + 1.
    \end{equation}
\end{definition}

In other words, the PMW is a vertex-specific predictive heuristic to the matching weight, assuming a union occurs in the next growth iteration. The PMW can be utilized by prioritizing the growth of vertices with low PMW such that there is an increased probability of mergers between clusters on edges connected to these vertices, and there is an increased probability in a lower matching weight. However, the PMWs' calculation within a cluster is not a trivial task, especially for clusters of increasingly larger size, as all edges of a cluster must be considered in its calculation. Furthermore, the PMWs within a cluster change due to cluster growth and mergers, both of which occur more frequently as the system size increases. For this reason, the scaling of the PMW computation is vital to the decoder. 

\Figure[htb](topskip=0pt, botskip=0pt, midskip=0pt){tikzfigs/tikz-figure1.pdf}{
    The cluster of \Cref{fig0} after two rounds of prioritized growth of $v_0$ and $v_2$. There are regions of vertices that are either interior elements or have equal potential matching weights, represented as nodes with different node radii in the node-tree $\nset$. \label{fig1}}

\subsection{Node-Suspension data structure}\label{sec:nodeset}

Fortunately, the PMW calculation is quite efficient by the introduction of a new data structure. Consider the cluster of non-trivial vertices $V_i=\{v_0,v_1,v_2\}$ and edges $E_i = \{(v_0,v_1), (v_1, v_2)\}$ from \Cref{fig0}. We had found previously that vertices $v_0, v_2$ have a lower PMW compared to $v_1$ by 1 edge. The growth of $v_0$ and $v_2$ are thus prioritized, such that new vertices are added to the cluster on the boundary of $v_0$ and $v_2$. If all newly added vertices are trivial, the cluster is now as in \Cref{fig1}. If we repeat the PMW calculation, we now find that the PMWs in the new vertices connected to $v_0$ are equal. The same is true for vertices connected to $v_2$. 
\begin{definition}\label{def:vertextree}
    Let the vertex-tree $\vset_i$ be a connected acyclic subgraph of the graph of a cluster $G(V_i, E_i)$.   The vertex-tree $\vset_i$ includes all vertices $V_i$ and a minimum number of edges in $E^\vset_i \subseteq E_i$. 
\end{definition}
\begin{definition}
  Let the node-tree $\nset_i$ be a partition of the vertex-tree $\vset_i$, such that each element of the partition --- a \textbf{node} $n$ --- consists of a set of adjacent vertices that lie within a certain distance --- the \textbf{node radius} $r$ --- from the \textbf{primer vertex}, which initializes the node and lies at its center. The node-tree is a directed acyclic graph, and its edges $\m{E}_i$ have lengths equal to the distance between the primer vertices of neighboring nodes. 
\end{definition}

The concept of primer vertices is easily understood when considering non-trivial vertices of the syndrome $\sigma$. Suppose every non-trivial vertex is the primer of a node, the weight of a matching in $\vset_i$ equal to the weight of the same matching in $\nset_i$. Furthermore, for every node of the node-tree, all vertices that lie at distance $r$ to the primer vertex are either boundary vertices to the cluster and have equal PMW, or lie within the radius of another node. For the example in \Cref{fig1}, the PMW of all boundary vertices of $n_0$, for simplicity just the PMW of $n_0$, is $\floor{r_0} + (n_1, n_2) + 1$. The partition from $\vset$ to $\nset$ thus allows us to compute the PMW on a reduced tree. 

\Figure[htb](topskip=0pt, botskip=0pt, midskip=0pt){tikzfigs/tikz-figure2.pdf}{
    Two different types of nodes. Syndrome-nodes $s$ have a non-trivial vertex or syndrome at its center. Vertices that lie on the radii of two existing nodes initialize a junction-node $j$ in the node-tree.\label{fig2}}

\Figure[hbt](topskip=0pt, botskip=0pt, midskip=0pt){tikzfigs/tikz-figure3.pdf}{
    The relevant data structures. \emph{(a)} The cluster-tree of the Union-Find data structure. The path from a vertex to the root of the cluster-tree is traversed to find the root element in order to differentiate between clusters. The root node of the node-tree is now additionally stored at the root of the cluster-tree. \emph{(b)} The vertex-tree $\vset$ with 9 non-trivial vertices. As $\vset$ is strictly acyclic, the cluster's edges must be maintained such that no cycles are created. This is done during growth by removing edges (red dotted lines) if a cycle is detected. \emph{(c)} The node-tree $\nset$, which currently has the same number of elements as $\vset$, as all vertices are non-trivial. Two depth-first searches are required to compute node parities (head recursively) and delays (tail recursively) in $\nset$.\label{fig3}}

All non-trivial vertices serve as primers for nodes that are called \textbf{syndrome-nodes} $s$. However, not all primer vertices are non-trivial vertices of the syndrome. If two non-trivial vertices are located an even Manhattan distance on the lattice, the growth of their clusters can simultaneously reach some vertex that lies on equal radii of the associated nodes, such as in \Cref{fig2}. For this reason, such vertices serve as primers of a different type of node --- a \textbf{junction-node} $j$ --- in the merged node-tree. 

The calculation of the PMW on the node-tree $\nset$ rather than the vertex-tree $\vset$ offers a reduction in the cost. However, it is still no trivial task as the entire tree must be considered for the calculation in every node. Instead, we will compute for the \textbf{node suspension} $n_s$ --- the number of growth iterations needed for a node to reach the maximum PMW in the node-tree --- which relates closely to the PMW. For example, the node suspension for the nodes $\{n_0, n_1, n_2\}$ associated with the vertices $\{v_0, v_1, v_2\}$ in \Cref{fig0} is $\{0, 2, 0\}$, and $\{0, 1, 0\}$ in \Cref{fig1}.

The Node-Suspension data structure does not replace but coexists with the Union-Find data structure. Additional to the the Union-Find data structure's cluster-trees of distinct roots, we store for every cluster the node-tree $\nset_i$ by its root node. For this, we need to maintain the reduced set of edges $E^\vset_i \subseteq E_i$ of the vertex-trees $\vset_i$ for every cluster, which can be done in constant time (see Algorithm \ref{algo:ufbb}). In the UF decoder, vertex-trees $\m{V}_i$ are not maintained, such that the graph associated with each cluster is not acyclic \cite{delfosse2017almost}. Instead, a spanning forest $F$ of all clusters is created \cite{delfosse2017linear} after growth. Each connected element within $F$ is also an acyclic graph. The difference is that while a single depth-first search or breath-first-search creates $F$, $\vset$ is equivalent to multiple breadth-first searches from each non-trivial vertex within the cluster, where the search of every breadth occurs during a growth iteration. The relevant data structures are depicted in \Cref{fig3}. 


\subsection{Node parity, delay, and suspension}\label{sec:paritydelaysus}

The Node-Suspension data structure allows for calculating the node suspension of all nodes in a node-tree $\nset$ by two intermediate steps. In each step, a depth-first-search (DFS) of $\nset$ is applied from its root node $r$ (\Cref{fig3}c).

In the first DFS, we calculate for the \textbf{node parity} $n_p$ --- the number of descendant syndrome-nodes of a node modulo 2 --- via a tail-recursive function, which is only dependent on the node parities of the children nodes of a node. The node parity is defined differently per node type:
\begin{align}\label{eq:nodeparity}
    s_p &= \hspace{.6cm}\big( \sum_{\mathclap{n \in \text{ children of } s}} (1-n_p) \big ) \bmod 2,\\
    j_p &= 1 - \big(\sum_{\mathclap{n \in \text{ children of } j}} (1-n_p) \big) \bmod 2.
\end{align}

In the second DFS, we calculate for the difference in node suspension of a node $n$ with its parent $m$; $\delta = n_s - m_s$. We can choose an arbitrary \textbf{node delay} $n_d$ --- the node suspension minus the maximum node suspension in the node-tree --- for the root node $r$ such as $r_d=0$ and add the suspension difference $\delta$ during each step to obtain $n_d$ for every node. This node delay of a node $n$ is only dependent on the node radii of itself and its parent $m$, the length of edge $(n,m)$, and its parity $n_p$. 
\begin{multline}\label{eq:delayequation}
    n_d = m_d + \bigg \lceil 2C\big(\ceil{n_r} - \floor{m_r + n_r \bmod 1}\\
    - (-1)^{n_p}\abs{(n,m)}\big) - 2(n_r - m_r) \bmod2 \bigg \rceil
\end{multline}
Here, the \textbf{inversion constant} $C$ deals with the inversion of node parities in a node-tree during merges of clusters explained in \ref{sec:nodejoin}. The node suspension is then related to the node delay by
\begin{equation*}
    n_s = n_d - \max_{x \in \nset}{x_d}. 
\end{equation*}
The maximum node delay can be maintained during the second DFS of the node-tree, and the node suspension itself is calculated during cluster growth. A single growth iteration, which is applied in the UF decoder by adding half-edges to all boundary vertices of the cluster, is now replaced by another DFS of $\nset$. During this DFS, we calculate the suspension $n_s$ for a node, and conditionally grow it - adding half-edges to the boundary vertices in the current node and adding 1 to its radius $n_r$ --- if $n_s = 0$. This requires us to save the list of boundary vertices to each node (\Cref{fig3}c). When all $n_s$ in $\nset$ are zero, all nodes are grown simultaneously within the same iteration. 

If the node-tree does not change after a growth iteration, which is the case if no mergers occur between clusters, the node suspensions decrease in an expected manner: For all nodes that are not suspended from growth, their node suspensions decrease with 1 in the next growth iteration. Due to this behavior, we can reuse the node delays $n_d$ to calculate $n_s$ for the next growth iteration by introducing another node parameter $n_w$, the number of iterations a node has \textbf{waited}. Each time a node is suspended from growth, we add 1 to $n_w$. The node suspension in subsequent iterations is then
\begin{equation}\label{eq:suspension}
    n_s = n_d - \max_{x \in \nset}{x_d} - n_w. 
\end{equation}
Note that we have not stated which node in $\nset$ should be the root node. In fact, any node in $\nset$ could have been picked as the root of the node-tree. As long as the DFS of cluster growth is performed in the same direction as the DFSs of the parity and delay calculations. If no cluster mergers occur, the node delays can be reused in the node suspension calculation prior to node growth. 
The node-tree is constructed by storing all neighbors of a node to a list. This way, the DFSs' direction can be determined by simply saving the root node, the starting point of the DFSs, to the cluster. All node variables are depicted in \Cref{fig3}c. 
%In the next section, we expand upon this idea of "reusing" some intermediate parameters to calculate the node suspensions after a cluster merger.  


\subsection{Joining node-trees}\label{sec:nodejoin}

In the Union-Find (UF) algorithm, odd parity clusters of an odd number of non-trivial vertices, --- elements of $\sigma$ --- grow in size repeatedly and merge with other clusters until all clusters are even. During these mergers, the node-trees of the Node-Suspension data structure must also be combined. Let us now first make a clear distinction between the merging protocols of the underlying data structures; the clusters-trees of the UF data structure are merged with the \codefunc{Union} function, whereas the node-trees are merged with a separate \codefunc{Join} function. After a join of multiple node-trees, the node suspensions within the combined node-tree change. Therefore, \codefunc{Join} protocol's focus is to minimize the DFSs of the recalculation of the node parity and delays in the combined node-tree. 

First, note that as a cluster of even parity has an even number of non-trivial vertices, its node-tree has an even number of syndrome-nodes. For these even node-trees, the concept of PMW does not exist, as the matching can be made within the node-tree. Consequently, node suspension, parity, and delays are undefined when two odd node-trees join to an even node-tree. 
%Thus, if two odd clusters merge into an even cluster, we don't know and do not care about its node suspensions. 

\Figure[hbt](topskip=0pt, botskip=0pt, midskip=0pt){tikzfigs/tikz-figure4.pdf}{
    \emph{(a)} An odd cluster $\nset_o=\{n_1, n_2, n_o\}$ with root $n_1$ joins with an even cluster $\nset_e=\{n_3, n_e\}$ with root $n_3$ on nodes $n_o, n_e$, respectively, to a joined node-tree. If we choose to \emph{(b)}, make $n_e$ a child of $n_o$, the parities and delays the sub-tree of $\nset_o$ can unchanged, and we only have to perform partial parity and delay calculations over the sub-tree of $\nset_e$. If we choose to \emph{(c)}, make $n_o$ a child of $n_e$, parities and delays have to be recalculated in the entire joined node-tree. \label{fig4}}

The second type of merger is between an even and an odd cluster. The combined cluster is odd, and its growth is continued. Thus its node suspensions must be computed. Consider the example of odd node-tree $\nset_o$ and even node-tree $\nset_e$ that are to be joined on nodes $n_o\in \nset_o$ and $n_e \in \nset_e$ (\Cref{fig4}\emph{a}). If $\nset_o$'s root is kept as the root of the joined node-tree (\Cref{fig4}\emph{b}), $n_e$ is to be a child node of $n_o$. As $\nset_e$ contains an even number of syndrome-nodes, the node parities in $\nset_o$ do not change. Hence, the node parity DFS is only necessary in the sub-tree $\nset_e$, which now has $n_e$ as sub-root. Furthermore, as the node delay is only dependent on its own properties and its parent's, the node delay DFS is also only required from node $n_e$ and within the sub-tree of $\nset_e$. These so-called \textbf{partial} DFSs of the node-tree are precisely what was required, as the node parity and delays in $\nset_e$ were undefined. Alternatively, if $\nset_e$'s root becomes the root of the combined tree (\Cref{fig4}\emph{c}), an odd number of syndrome-nodes are attached to $n_e$, such that the parities of nodes on the path from $n_e$ to the root are changed. Such a join would require the DFSs on the entire combined node-tree to calculate for node parities and delays. Thus, a simple rule is always to keep the root of the odd node-tree, which we dub \textbf{Odd-Rooted Join}.

In addition, a cluster can be subjected to multiple mergers within the same growth iteration, during which the parity of the merged cluster changes dependent on the number of mergers and the parities of the clusters involved. The DFSs related to the parity and delay calculations must, for this reason, not be initiated directly after the joining of node-trees. After all, it may be possible for the cluster to merge again such that the parities and delays become invalid. To prevent these redundant calculations, sub-roots of the even sub-trees are stored to a list $\m{S}$ at the root of the node-tree (\Cref{fig3}\emph{c}). When multiple mergers occur, the root node that stores the now redundant sub-roots is replaced by a new root with new $\m{S}$. If a cluster is selected for growth, we check for the sub-roots in $\m{S}$ at the new root node and initiate the DFSs from these sub-roots. We call this the \textbf{Root List $\m{S}$ Replacement}. 

\Figure[htb](topskip=0pt, botskip=0pt, midskip=0pt){tikzfigs/tikz-figure5.pdf}{
    The node suspension values for nodes for 3 odd node-trees $\{\nset_1, \nset_2, \nset_3\}$ of 3 nodes that grow and join into a single node-tree. \emph{(a)} Node suspensions are calculated by setting $C=1$ in equation \eqref{eq:delayequation}. In step 1, the growth in each of the three node-trees' outer nodes is prioritized, and the node-trees merge. In step 2, the recalculation of the joined node-tree is performed. Parities within the sub-tree of $\nset_2$ are now inverted, and the suspension in these nodes have doubled. \emph{(b)} Node suspensions are calculated by setting $C=\nicefrac{1}{2}$. Now the increase in node suspensions after parity inversion is halved.\label{fig5}}

\subsection{Parity inversion}\label{sec:inversion}
An unfortunate effect of the Node-Suspension data structure, which we dub \textbf{Parity Inversion}, causes a decrease in the algorithm's performance as the lattice size is increased. We will demonstrate this effect through the example in \Cref{fig5}\emph{a}. Consider three instances of the node-tree of \Cref{fig0}; $\nset_a, \nset_b, \nset_c$, positioned near each other on the lattice. For each node-tree, if the middle node is suspended from growth for two iterations, all nodes have the same Potential Matching Weight. However, in the current example, the node-trees $\nset_a, \nset_b, \nset_c$ merge after 1 iteration. The combined node-tree is odd. Thus, we recalculate the node parities and delays to find that the parities in the partition of the node-tree containing the nodes of $\nset_b$ have been inverted, and the node suspensions in this partition have doubled from before node suspensions before the merger. If the next merging event occurs on the node with the doubled node suspension, the matching weight may be larger compared to the original UF decoder, which defies the goal of Node-Suspension to decrease the weight.

This defines a trade-off in the Node-Suspension data structure; a node must wait as many iterations as it is suspended to reach equilibrium in Potential Matching Weight in the node-tree, but after Parity Inversion, the node suspension for previously prioritized nodes increases linearly with the number of iterations waited by the suspended nodes pre-inversion. As a compromise, we redefine the node suspension as \textbf{half} the number of growth iterations needed for all nodes in the node-tree to reach equal PMW. This can be done by setting $C=0.5$ in Equation \eqref{eq:delayequation}. Nevertheless, as more inversions occur, the maximum node suspension in the node-tree increases, and it becomes more and more unlikely for a cluster to actually reach zero node suspension in all nodes. The number of inversions is directly related to the number of merging events, and thus the size of the lattice. The performance to improve the heuristic for minimum weight matching thus decreases for larger lattices. 


\subsection{Node parity, delay and suspension}\label{sec:paritydelaysus}

The Node-Suspension data structure allows for the calculation of the node suspension of all nodes in a node-tree $\nset$ by two intermediate steps. In each step a depth-first-searches (DFS) of $\nset$ is applied, such that the calculation can be performed in linear time to the node-tree dimension. In the first DFS, we calculate for the \textbf{node parity} - the number of child syndrome-nodes of a node modulo 2 - $np$ via a tail recursive function, which is only dependant on the node parities of the children nodes $\{n_{\gamma,1}, ...\}$ of a node $n_\beta$:
\begin{equation}\label{eq:nodeparity}
np_\beta =
\begin{cases}
    \big( \sum_{n_\gamma} (1-np_\gamma) \big ) \bmod 2 \hspace{1em} | \hspace{1em} n_\beta \equiv sn_\beta \\
    1 - \big( \sum_{n_\gamma} (1-np_\gamma) \big ) \bmod 2 \hspace{1em} | \hspace{1em} n_\beta \equiv jn_\beta.
\end{cases} 
\end{equation}

In the second DFS, we calculate for the difference in node suspension $\delta ns$ of a node $n_\beta$ with its parent $n_\alpha$. We can choose an arbitrary \textbf{node delay} $nd$ for the root node such as $nd_r=0$ and add the difference $\delta ns$ during each step to obtain $nd$ for every node. This node delay of a node $n_\beta$ is only dependent on the node radii of itself $nr_\beta$ and its parent $nr_\alpha$, the length of node-tree edge $(n_\beta, n_\alpha)$ and its parity $np_\beta$. 
\begin{multline}\label{eq:delayequation}
    nd_\beta = nd_\alpha + \Bigg \lceil 2k_{eq}\bigg(\ceil{\frac{nr_\beta}{2}} - \floor{\frac{nr_\alpha + nr_\beta \bmod 2}{2}}\\
    - (-1)^{np_\beta}\abs{(n_\beta,n_\alpha)}\bigg) - (nr_\beta - nr_\alpha) \bmod 2 \Bigg \rceil
\end{multline}
Here, the \textbf{equilibrium factor} $k_{eq}$ is a constant that deals with the inversion of node parities in a node-tree during merges of clusters, detailed in Section \ref{sec:nodejoin}. 

\Figure[htb](topskip=0pt, botskip=0pt, midskip=0pt){tikzfigs/tikz-figure3.pdf}{Two depth-first searches on $\mathcal{N}$ to compute node parities (head recursively) and delays (tail recursively).\label{fig3}}

There is a final step in calculating the node suspension from the node delay, which are related by
\begin{equation}
    ns = nd - \max_{nd_i \in \nset} nd_i - nw, 
\end{equation}
where $nw$ is equal to the number of iterations a node has \textbf{waited} or has been suspended from growth. The maximum node delay can be maintained during the second DFS of the node-tree. The node suspension itself is calculated dynamically during the growth of the cluster. 

A single growth iteration, which is applied in the original UF decoder by adding half-edges to all boundary vertices of the cluster, is now replaced by another DFS of $\nset$. During this DFS, a node is conditionally bloomed - adding half-edges to the boundary vertices in the current node and adding 1 to the node radius $nr$ - if $ns$ is equal to zero. If not, add 1 to $nw$ and continue with the DFS. A subsequent growth iteration now does not require the two DFS related to the calcualtion of $np$ and $nd$, provided that no union between clusters has occured. The Node-Suspension data structure enables us to calculate the node suspension across multiple growth iterations based on a single calculation of the node parity and delay. When all $ns$ in $\nset$ are zero, all nodes are bloomed simultaenously within the same iteration. This process of reaching equal $ns$ or PMW in the cluster is called the \textbf{Balanced Bloom} of nodes. 

Note that we hadn't stated which node in $\nset$ should be the root node. In fact, any node in $\nset$ could have been picked as the root of the node-tree. This property, together with the constancy of $np$ and $nd$ in betwen cluster unions, allows us to define a set of rules for the merges of node-trees.  
\Figure[ht](topskip=0pt, botskip=0pt, midskip=0pt){tikzfigs/tikz-figure4.pdf}{
    (a) An odd cluster $\nset^o=\{n_1, n_2, n_o\}$ with root $n_1$ joins with an even cluster $\nset^e=\{n_3, n_e\}$ with root $n^e_r=n_3$ on nodes $n_o, n_e$, respectively, to a joined node-tree. If we choose to (b), make $n_e$ a child of $n_o$, the parities and delays the sub-tree of $\nset^o$ can unchanged, and we only have to perform partial parity and delay calculations over the sub-tree of $\nset^e$. If we choose to (c), make $n_o$ a child of $n_e$, parities and delays have to be recalculated in the entire joined node-tree. \label{fig4}}

\subsection{Joining node-trees}\label{sec:nodejoin}

In the Union-Find algorithm, clusters of an odd number of non-trivial vertices or elements of $\sigma$ grow in size repetitively and merge with other clusters until all clusters are of even parity. During this process, the node-trees of the Node-Suspension data structure must also be merged. Let us now first make a clear distinction between the merging protocols of the underlying data structures; the vertex-trees of the UF data structure are merged with \codefunc{Union}, whereas the node-trees are merged with \codefunc{Join}. After a join of multiple node-trees, the node suspensions within the combined node-tree change. The focus of the \codefunc{Join} protocol is therefore to minimize the DFS's of the recalculation of the node parity and delays in the combined node-tree. 

First, note that as a cluster of even parity has an even number of non-trivial vertices, its node-tree has an even number of syndrome-nodes. The concept of PMW does not exist for an even node-tree as the matching can be made within the node-tree. Consequently, node suspension, parity and delays are undefined for an even node-tree. 
%Thus, if two odd clusters merge into an even cluster, we don't know and do not care about its node suspensions. 

A second type of merge is the between an even and an odd cluster. The combined cluster is odd and its growth must be continued, thus its node suspensions must be computed. Now, consider the example of odd node-tree $\nset_o$ and even node-tree $\nset_e$ that are to be joined on nodes $n_o\in \nset_o$ and $n_e \in \nset_e$ (Figure \ref{fig4}\emph{a}). If the root of $\nset_o$ is kept as the root of the joined node-tree (Figure \ref{fig4}\emph{b}), $n_e$ is to be a child node of $n_o$. As $\nset_e$ contains an even number of syndrome-nodes, the node parities in $\nset_e$ do not change. Hence, the DFS of the node parity recalculation is only necessary in the sub-tree of $\nset_e$, which now has $n_e$ as sub-root. Furthermore, as the node delay is only dependent on the properties of a node and of its parent node, the DFS of the node delay recalculation is also only required from node $n_e$ and is performed within the sub-tree of $\nset_e$. These partial DFS's of the node-tree are exactly what was required as the node parity and delays in $\nset_e$ were undefined. If the root of $\nset_e$ takes the role of the root of the combined tree (Figure \ref{fig4}\emph{c}), an odd number of syndrome-nodes are attached to $n_e$, such that the parities of nodes on the path from $n_e$ to the root are changed. Such a join would require the DFS's for the entire combined node-tree of the recalculation of node parities and delays. A simple rules is thus to always keep the root of the odd node-tree. 

Finally, a cluster can be subjected to mergers with multiple other clusters within the same growth iteration, during which the merged cluster may switch parity multiple times. The DFS's related to the recalculation of the node parities and delay must for this reason not be initiated directly after the joining of node-trees. Instead, a pointer to the sub-root of the even sub-tree in the most recent odd-even join is stored at the root of the node-tree. The recalculation from the sub-root is then initiated just before cluster growth to prevent multiple recalculations over the same partitions of the node-tree. 

% to even -> node suspension undefined


% \Figure[htb](topskip=0pt, botskip=0pt, midskip=0pt){tikzfigs/tikz-figure5.pdf}{bla\label{fig5}}

% Equilibruim
\subsection{Parity inversion}\label{sec:inversion}

A unfortunate effect of the Node-Suspension data structure, which we dub parity-inversion, causes a decrease in the performance of the algorithm as the lattice size is increased (see Figure \ref{threshold_ufbb}). We will demonstrate this effect through the example in Figure \ref{fig5}\emph{a}. Consider three instances of the node-tree of Figure \ref{fig0} $\nset_a, \nset_b, \nset_c$, positioned closely to each other on the lattice. Every node in all node-trees have radius 1, and the node suspension in $n_1$ in each node-tree is 2. Thus means that if $n_1$ is suspended for two growth iterations, such as in Figure \ref{fig1}, all nodes have the same PMW. However, in this example, the node-trees $\nset_a, \nset_b, \nset_c$ merge after 1 iteration. The merged cluster is odd, thus we recalculate the node parities and delays per the rules set in the previous section. Now, the parities in the partition of the node-tree containing the nodes of $\nset_b$ have been inverted , and the calculated node suspensions in this partition increases dramatically. 

If the next merging event occurs on the node with the increased node suspension, the matching weight may be increased compared to the original UF decoder. This defines a trade-off in the node suspension; a node must wait as many iterations as it is suspended to reach equal PMW in the node-tree, but after a parity-inversion the node suspension for previously prioritized nodes increase linearly with the number of iterations waited by the suspended nodes pre-inversion. As a comprimise, we redefine the node suspension as \textbf{half} the number of growth iterations needed for all nodes in the node-tree to reach equal PMW. This can be done in Equation \eqref{eq:delayequation} by setting $k_{inv}=0.5$.

Nevertheless, as more parity inversions occur, the maximum node suspension in the node-tree increases, and it becomes more and more unlikely for a cluster to actually reach zero node suspension in all nodes. The number of parity inversions is directly related to the number of merging events, and thus the size of the lattice. The performance to improve the heuristic for minimum weight matchings thus decreases for larger lattices. 

\Figure[htb](topskip=0pt, botskip=0pt, midskip=0pt){tikzfigs/tikz-figure5.pdf}{
    (a) The delay values $n_i.d$ and the equilibrium-states $(I:k_{eq}M)$ for 3 odd clusters $\{\nset_1, \nset_2, \nset_3\}$ of 3 nodes that grow and join into a size-9 cluster. (Top) Initially, parity and delay calculations  are performed with delay equation \eqref{eq:2ddelay} on each odd cluster which have equilibrium-states $(0:2)$, with delay $2$ in the middle node. (Middle) The clusters are grown, where the middle node is delayed, such that it's delay value decreases to $1$, and the clusters have equilibrium-states $(1:2)$. (Bottom) The clusters join to a single odd cluster, which is selected for growth. Hence, parity and delay calculation is performed again, and the equilibrium-state is $(0:4)$, thus requiring 4 growth iterations before equal potential matching weight is reached in all nodes.  (b) The same clusters, growths and joins, but now with delay equation \eqref{eq:delayequation} for $k_{eq} = 1/2$. With the equilibrium factor, $(k_{eq}M:k_{eq}M)$ can be reached in fewer growth iterations; e.g. after 1 round (middle), $(1:1)$ is reached. Also, after join to a single cluster (bottom), fewer iterations are needed (2 compared to 4 in Figure \ref{fig:kbloom}).\label{fig5}}
\subsection{Pseudocode}\label{sec:pseudocode}
\FloatBarrier
The full version of the algorithm we have described is given in Algorithm \ref{algo:ufbb}. Note that this pseudocode includes instructions that are shortened versions of the pseudocode of the Union-Find decoder \cite{delfosse2017almost}. This is done for clarity on the additions of the Node-Suspension data structure and protocols on top of the Union-Find pseudocode. The first block 1-4 initializes the clusters and described the loop of cluster growth. Block 2 contains lines 5-8 and describes the DFS's related to the calculation of node parities and delays, and the DFS of the cluster growth. Block 3 contains lines 9-17 described the combined merging protocols of the Union-Find and Node-Suspension data structures. Node that lines 16-17 contain an extra step to ensure that the vertex-trees are always acyclic. The final block in line 17 is the peeling decoder \cite{delfosse2017linear}. 

\begin{algorithm}[htb]
    \BlankLine
    \KwData{A graph $G=(\m{V},\m{E})$, an erasure $\m{R} \subseteq \m{E}$ and syndrome $\sigma \subseteq \m{V}$}
    \KwResult{Correction set $\m{C}$}
    \BlankLine
    Initialize cluster vertex-trees, node-trees and other of UF.\;
    Create the list $\m{L}$ of odd clusters.\;
    \While(){$\m{L}$ is not empty}{
      Initialize the fusion list $\m{F}$ as an empty list.\;
      \For(){cluster $c\in\m{L}$}{
        \If(){pointer to even sub-root $n_e$ exists}{
          Apply DFS's to calculate node parities and delays (Equations \eqref{eq:nodeparity}, \eqref{eq:delayequation}) in partition of even sub-tree;
        }
        Apply DFS from root $n_r$ to all leafs. If $ns=0$ (Equation \eqref{eq:suspension}), grow all boundary edges of vertices in the node a half-edge per the Union-Find decoder, such that grown edges are added to $\m{F}$. If $ns\neq0$, apply $nw=nw+1$ and continue the DFS.\;
      }
      \For(){edge $(u,v) \in \m{F}$}{
        \eIf(){$\codefunc{Find}(u)\neq\codefunc{Find}(v)$}{
          Merge vertex-trees by weighted \codefunc{Union}.\;
          \eIf(){$u \in n_v$ and $v \in n_v$}{
            Merge node-trees by \codefunc{Join}.\;
          }(){
            Add $u$ or $v$ to $n_v$ or $n_u$, respectively.\;
          }
        }(){
          Subtract 1 from $(u,v)$ in \emph{Support}.\;
        }
      }
    }
    Apply the peeling decoder \cite{delfosse2017linear}.
    \caption{Union-Find Node-Suspension decoder}\label{algo:ufbb}
  \end{algorithm}
    

\clearpage

\section{Complexity of Node-Suspension}\label{sec:complexity}

In this section, we will find the worst-cast time complexity of the Union-Find Node-Suspension decoder. The addition cost of the original Union-Find decoder can be split in two parts: (A) the depth-first-searches (DFS's) related to the (re)calculation of the node parities and node delays in line \ref{algo:pdc}, and (B) the DFS related to the growth of a cluster in line \ref{algo:grow}. We dub the two parts the \textbf{suspension cost} and the \textbf{growth cost}, respectively. The \codefunc{Join} operation in lines \ref{algo:joina}-\ref{algo:joinb} only has a linear addition to the cost.

\subsection{Suspension cost}\label{sec:suscomplexity}

The cost of node suspension calculation is proportional to $N_{\text{sus}}$, the number of nodes traversed in the DFS's of the node parities and node delays. As a result of Odd-Rooted Join, $N_{\text{sus}}$ is proportional to the sum of sizes of all even node-trees. Root List Replacement decreases the sum to the most recent even node-trees that are ancestors of grown odd node-trees, which we dub $\mathbf{\Delta}$ \textbf{node-trees}. To find the worst-case time complexity, we maximize $N_{\text{sus}}$, which is proportional to the computation time. We take a time-reversed approach of analyzing a cluster; starting from a single cluster that maximally occupies the lattice at the end of growth, and move back in time to find its ancestor clusters. In this process that we call \textbf{cluster fragmentation}, we aim to find the set of cluster mergers that maximizes the number and sizes of $\Delta$ node-trees. 

% The maximization of $N_{\text{sus}}$ is in the repetitiveness of the recalculation over some parition of the final node-tree. 


% \begin{enumerate}[label=P\arabic*,ref=P\arabic*]
%   \item Joins between the node-trees of odd and even clusters retain the root node of the odd cluster. \label{rjoin}
%   \item The DFS's of the node parity and delay calculation are performed just before cluster growth in an even partition of the node-tree starting from the pointer saved at the root node. \label{rgrow}
%   \item The smallest clusters, measured by the number of verties in $\vset$, are grown first. \label{rweight}
% \end{enumerate}

% From properties \ref{rjoin} and \ref{rgrow}, it can be $N_{\text{sus}}$ is proportional to the sum of sizes of all even node-trees during all mergers of the growth process on the lattice. Note that if many clusters merge within the same growth iteration, only the last even cluster counts towards the cost, since \ref{rgrow} ensures that the calculation is not performed on intermediate even partitions.

\begin{definition}\label{def:fragmentation}
  Let the \textbf{fragmentation} function $f$ on an odd node-tree $\omega$ return the set of $\nu + 1$ of its most recent odd ancestor node-trees on which suspension calculations were performed. Let the prefix on the node-tree indicate the \textbf{fragmentation generation}, such that $f(\pre{k-1}\omega)$ returns $\nu + 1$ node-trees of generation $k$, denoted by $\pre{k}\m{F}$:
  \begin{equation}\label{eq:fstep}
    f(\{\pre{k-1}\omega\}) = \{\pre{k}\omega_0,\pre{k}\omega_1,...,\pre{k}\omega_{\nu}\}.
  \end{equation}
  Let $f$ be the combination of \textbf{partial fragmentation} functions $f_\omega$ and $f_\epsilon$, where $f_\omega$ fragments an odd node-tree $\pre{k-1}\omega$ into an odd ancestor $\pre{k}\omega_0$ and an even ancestor node-tree $\pre{k}\epsilon$: 
  \begin{equation}\label{eq:pfo}
    f_\omega(\{\pre{k-1}\omega\}) = \{\pre{k}\epsilon, \pre{k}\omega_0 \},
  \end{equation}
  and where $f_\epsilon$ further fragments $\pre{k}\epsilon$ into $\nu$ odd ancestors
  \begin{equation}\label{eq:pfe}
    f_\epsilon(\{\pre{k}\epsilon\}) =\{\pre{k}\omega_1,...,\pre{k}\omega_{\nu}\}.
  \end{equation}
  % of an odd cluster with node-tree $\pre{k-1}\omega$ split it into a set of its ancester node-trees. Here $k-1$ indicates the \emph{fragmentation step number}, where larger step number $k$ refers to an ancestor node-tree of smaller size. Let the fragmentation $f$ be the combination of \emph{intermediate fragmentations} (IF) $f_\omega$, which fragments an odd node-tree into an even ancestor and an odd ancestor

  % and $f_\epsilon$, which fragments even node-trees into $\nu$ odd ancestors
  % \begin{equation}\label{eq:pfe}
  %   f_\epsilon(\{\pre{k}\epsilon\}) = \m{F}^e_k=\{\pre{k}\omega_1,...,\pre{k}\omega_{\nu}\},
  % \end{equation}
  % such that a fragmentation is
\end{definition}

The fragmentation function can be applied consecutively, such that a set of odd node-trees of generation $k$ is fragmented to ancestors of generation $k+1$. We use the notation $f^{(2)}(\pre{k-1}\omega)$ to indicate that two fragmentations are applied on $\pre{k-1}\omega$ to obtain ancestors of generation $k+1$. Furthermore, let $f_\omega^{(i)}$ and be equivalent to $f_\omega f^{(i-1)}$. 

According to Odd-Rooted Join and Root List Replacement, if the cluster of node-tree $\pre{k-1}\omega$ is grown, the DFS's of the parity and delay calculations are performed within $\pre{k}\epsilon$. Therefore, every even node-tree returned by $f_\omega$ is a $\Delta$ node-tree. The value of $N_{\text{sus}}$ can thus be obtained by taking the sum of sizes of all even node-tree in $\pre{k}\m{F}_\omega$ over a series of $\mu$ fragmentations of some odd node-tree $\Omega = \pre{0}\omega$ until there are no more ancestor node-trees:
\begin{equation}\label{eq:npdc}
  N_{\text{sus}} = 2\sum_{k=1}^\mu{ \sum_{ \pre{k}\epsilon \in f_\omega^{(k)}(\Omega) }{ \abs{\pre{k}\epsilon}} }.
\end{equation}

To find $N_{\text{sus}}$, we are going to make two assumptions to simplify \eqref{eq:npdc}. Junction-nodes are initiated on the tangent of two node radii belonging to separate node-trees when merging into one. For increasing fragmentation generation, the total number of nodes in the fragmented set must therefore decrease. By neglecting their existence, \eqref{eq:npdc} becomes
\begin{equation}\label{eq:npdc2}
  N_{\text{sus}} \leq 2\sum_{k=1}^\mu{ \sum_{ \pre{k}\epsilon \in f_\omega^{(k)}(\Omega) }{ \abs{\pre{k}\epsilon}} }.
\end{equation}
Furthermore, if only syndrome-nodes exist, the size of $\pre{k-1}\omega$ must equal the sum of sizes of its ancestors $f(\pre{k-1}\omega)$. In other words, the size of the ancestor $\pre{k}\omega_i$ can be represented by the \textbf{fragmentation ratio}
\begin{equation}\label{eq:ratio}
  \pre{k}R_i = \frac{\abs{\pre{k}\omega_i}}{\abs{\pre{k-1}\omega}}, \hspace{0.5cm} \sum_{i=0}^{\mu}{\pre{k}R_i} = 1,
\end{equation}
Secondly, we assume that vertex-trees do not increase in size, such that $\abs{\nset}=\abs{\vset}$. Normally, the number of nodes in a cluster is bounded by the number of vertices $\abs{\nset}\leq \abs{\vset}$, as non-trivial vertices can be added to the node, which increase the node radius. By this assumption, the vertex-tree can only increase in size as the result of a merger between clusters, and nodes are effectively not allowed to increase in radius. While this is not possible in during realistic cluster growth, using this assumption simplifies \eqref{eq:npdc2}, as we will see later, while not compromising its upper bound. 

To find the upper bound in $N_{\text{sus}}$, we are now tasked to find: (a) $\nu$, the number of ancestors in $f_\epsilon$; (b) the fragmentation ratios $\{R_0, ..., R_\nu\}$; and (c) the number of fragmentation generations $\mu$. 

\begin{lemma}\label{lem:evenconstant}
  For constant fragmentation ratios $\pre{k}R_i = R_i$, the sum $\sum_{ \pre{k}\epsilon \in f_\omega^{(k)}(\Omega) }{ \abs{\pre{k}\epsilon}}$ is constant for every $k$. 
\end{lemma}
\begin{proof}
  For $k=1$, there is a single even ancestor $\pre{1}\epsilon$ of size 
  \begin{equation*}
    \abs{\pre{1}\epsilon} = (1 - R_0)\abs{\Omega}.
  \end{equation*}
  For $k=2$, every odd node-tree in $f(\Omega)=\{\pre{1}\omega_0,...,\pre{1}\omega_\nu\}$ is fragmented by $f_\omega$ to an even ancestor $\pre{2}\epsilon_i$ for $i \in \{0,...,\nu \}$, such that 
  \begin{equation*}
    \sum_{i=0}^{\nu}{\abs{\pre{2}\epsilon_i}}  = \sum_{i=0}^{\nu}{R_i(1 - R_0)\abs{\Omega}}= (1 - R_0)\abs{\Omega}.
  \end{equation*}
  The same is true for all subsequent generations $k$. 
\end{proof}

\begin{theorem}\label{the:fragnumber}
  Upper bound for $N_{\text{sus}}$ is obtained by setting $\nu=2$ in \eqref{eq:pfe}. 
\end{theorem}
\begin{proof}
  The sum of even node-tree sizes in every generation is constant per \Cref{lem:evenconstant}. Thus, upper bound in \eqref{eq:npdc2} is obtained by the largest possible $\mu$. As $\nu$ increases the number of odd node-trees in each $f^{(k)}_\omega$, the average size of these odd node-trees decreases. Since the size of a node-tree is proportional to the number of ancestor generations, we find that 
  \begin{equation*}
    \mu \propto \frac{1}{\nu}. 
  \end{equation*}
  Hence, the upper bound in \eqref{eq:npdc2} exists in the minimal value of $\nu$, which is $\nu = 2$.
\end{proof}

Using \Cref{the:fragnumber}, we now find that a fragmentation on an odd cluster $f(\pre{k-1}\omega)$ returns $\{\pre{k}\omega_0, \pre{k}\omega_1, \pre{k}\omega_2\}$, where $\pre{k}\omega_1, \pre{k}\omega_2$ are ancestors of the even node-tree $\pre{k}\epsilon$ returned by $f_\omega(\pre{k-1}\omega)$. 

\begin{lemma}\label{lem:chrono}
  The node-tree size of $\pre{k}\omega_0$ must be smaller than $\pre{k}\omega_1, \pre{k}\omega_2$, such that $R_1 \geq R_0 \leq R_2$. 
\end{lemma}
\begin{proof}
  The partial fragmentations must occur in the order of first \eqref{eq:pfo}, then \eqref{eq:pfe}, as \eqref{eq:pfe} requires an even node-tree that is returned by \eqref{eq:pfo}. In terms of cluster growth, the vertex-trees $\vset_1, \vset_2$, corresponding to $\pre{k}\omega_1, \pre{k}\omega_2$, must merge before the combined vertex-tree can merge with $\vset_0$, which corresponds to $\pre{k}\nset_0$. Declared by Weighted Growth, $\abs{\vset_1}$ and $\abs{\vset_2}$ must be smaller or equal to $\abs{\vset_0}$, such that 
  \begin{equation*}
    \abs{\vset_1}\geq \vset_0 \leq\abs{\vset_2}.
  \end{equation*}
  If this condition is not met, the cluster of $\vset_0$ grows first and merges with either $\vset_1$ or $\vset_2$, and the chronology of events is disturbed. Since we assumed $\abs{\nset}=\abs{\vset}$, this can be translated to 
  \begin{equation*}
    \abs{\nset_1}\geq \nset_0 \leq\abs{\nset_2},
  \end{equation*}
  and subsequently to the fragmentation ratios.
\end{proof}

\begin{theorem}\label{the:ratios}
  Upper bound for $N_{\text{sus}}$ is obtained via the fragmentation ratios $R_0 = R_1 = R_2 = \nicefrac{1}{3}$.
\end{theorem}
\begin{proof}
  The ratios $\{R_0, R_1, R_2\}$ can be found by maximizing the size of the even node-tree $\pre{k}\epsilon$ in each fragmentation, which is 
  \begin{equation*}
    \abs{\pre{k}\epsilon} = (R_1 + R_2)\abs{\pre{k-1}\omega}.
  \end{equation*}
  Since $ R_1 \geq R_0 \leq R_2$ per \Cref{lem:chrono}, the largest values for $R_1, R_2$ possible are equal to $R_0$.
\end{proof}

The last unknown parameter in finding the upper bound of $N_{\text{sus}}$ in Equation \eqref{eq:npdc2} is $\mu$.

\begin{theorem}\label{the:km}
  For $\nu = 2$ and $R_i = \{\nicefrac{1}{3},\nicefrac{1}{3},\nicefrac{1}{3}\}$, the maximum number of fragmentation generations is $\mu = \log_3{\abs{\Omega}}$.
\end{theorem}
\begin{proof}
  In every generation, all node-trees are fragmented into 3 ancestors that are $\nicefrac{1}{3}$ the size of their descendant. The series of $\mu$ fragmentations is thus simply $\mu$ divisions of the node-tree $\Omega$ in 3 parts until all ancestors have size 1, at which point a node-tree cannot be fragmented.
\end{proof}

Collecting \Cref{the:fragnumber,the:ratios,the:km} and filling in Equation \eqref{eq:npdc2} we find that

\begin{align*}
  \nonumber N_{\text{sus}} &\leq 2\sum_{k=1}^\mu{ \sum_{ \pre{k}\epsilon \in f_\omega^{(k)}(\Omega) }{ \abs{\pre{k}\epsilon}}  } \\
  \nonumber         &\leq 2\sum_{k=1}^{\log_3{\abs{\Omega}}} \frac{2}{3}\abs{\Omega}\\
                    &\leq \frac{4}{3}\abs{\Omega}\log_3{\abs{\Omega}}.
\end{align*}

The maximum size of the odd node-tree $\Omega$ is bounded by the system size $n = \abs{V}$. The worst-case time complexity of the suspension calculation is thus $\m{O}(n\log{n})$. 

\subsection{Growth cost}\label{sec:growthcost}

To grow a cluster represented by a node-tree $\nset$, a depth-first search (DFS) is performed on the node-tree to find all nodes that have zero suspension. The total cost of these DFS's are proportional to the total number of nodes encountered during these DFS's, which we dub $N_{\text{grow}}$. Using the definition of fragmentations of \Cref{def:fragmentation}, the cost of growth is proportional to
\begin{equation}\label{eq:ngrow}
  N_{\text{grow}} = 2\sum_{k=1}^\mu{ \sum_{ \pre{k}\omega \in f^{(k)}(\Omega) }{ \abs{\pre{k}\omega}} }.
\end{equation}
We assume again that no trivial vertices are added  to a cluster or $|\nset| = |\vset|$ such that \eqref{eq:ngrow} becomes an upper bound. As a result of Odd-Rooted Join and Root List Replacement, the upper bound is obtained if there are as many fragmentation generations. This is again achieved through $\nu = 2$. For every fragmentation of some odd node-tree $\pre{k-1}\omega$ into $\{\pre{k}\omega_0, \pre{k}\omega_1, \pre{k}\omega_2\}$, all three ancestors add to $N_{\text{grow}}$ if they have grown. According to Weighted Growth, this is the case when $R_0 \approx R_1 \approx R_2\approx \nicefrac{1}{3}$ such that $\abs{\vset_0}\approx \abs{\vset_1}\approx\abs{\vset_2}$. For these values of $\nu$ and $R$, we can apply \Cref{the:km} for $\mu$. For $|\nset| = |\vset|$, the sum of node-tree sizes in every fragmented set $\m{F}_k$ is exactly $\abs{\Omega}$, and we find that
\begin{align*}
  \nonumber N_{\text{grow}} &\leq 2\sum_{k=1}^\mu{ \sum_{ \pre{k}\omega \in f^{(k)}(\Omega) }{ \abs{\pre{k}\omega}}  } \\
  \nonumber         &\leq 2\sum_{k=1}^{\log_3{\abs{\Omega}}} \abs{\Omega}\\
                    &\leq 2\abs{\Omega}\log_3{\abs{\Omega}},
\end{align*}
which again corresponds to a worst-case time complexity $\m{O}(n\log{n})$.
\section{Performance}\label{sec:performance}


We benchmark the performance of the Union-Find Balanced-Bloom decoder of Algorithm \ref{algo:ufbb} using our application in Python3 (see Appendix \ref{ap:oopsurfacecode}). This is done by Monte Carlo simulations of decoding on a simulated lattice and to fit for the code threshold, described in Section \ref{sec:simthres}. For the independent noise model (Definition \ref{def:independent}), we simulate on lattice sizes $L_{small}=[8, 16, 24, 32, 40, 48, 56, 64]$ with a minimum of $96.000$ samples and on $L_{big}=[72, 80, 88, 96]$ with a minimum of $28.800$ samples. For the phenomenological noise model (Definition \ref{def:pheno}), we simulate on lattice sizes $L_{small}=[8,12,16,20,24]$ with a minimum of $105.600$ samples and on lattice sizes $L_{big}=[28, 32, 36, 40, 44]$ with $13.200$ samples. 

\Figure[htb](topskip=0pt, botskip=0pt, midskip=0pt){tikzfigs/threshold_comparison.pdf}{bla\label{thres_comp}}


\subsection{Threshold}

We had initially simulated for the decoding success rates for the range of lattices in $L_{small}$, which is the same range used when benchmarking the various implementations of the Union-Find decoder in Section \ref{sec:ufperformance}. For $L_{small}$, we find that, except for the environment of planar code with independent noise, the thresholds of the Union-Find Balanced-Bloom decoder are increased from the DBUF thresholds, and moves close to the thresholds of the MWPM decoder. We also observe an increase in decoding success rate at the threshold $k_{th}$ from DBUF, which is also the case for the environment of planar code with independent noise. For the range of lattices in $L_{big}$ (Figure \ref{fig:threshold_ufbbbig}), the threshold of the Union-Find Balanced-Bloom decoder decreases to below DBUF thresholds. This does not necessarily mean that it performs worse, as $k_{th}$ is still above DBUF's values, but does raise questions on the scalability of the Union-Find Balanced-Bloom decoder. In fact, if we look closely at the values of $k_C$ (Figure \ref{fig:thres_ufbb_toric_2d_data}), we find that the fit does not accurately represent the underlying data points, which is different behavior from the MWPM and DBUF decoders. The threshold and $k_C$ of the Union-Find Balanced-Bloom decoder may not be accurate for comparison. 

For this reason, we have applied a \emph{sequential fit} to the data acquired in the Monte Carlo simulations of lattices of $L = L_{small} \cup L_{big}$. In the sequential fit, we iterate stepwise in the ordered list of lattice sizes $L$ and fit for the data of $L_i, L_{i+1}$ for $i \in |L|-1$ iterations. The fit thus returns an error threshold in the range of the chosen lattice sizes. The range of thresholds $p_{th}(L_i, L_{i+1})$ for the environment of toric code with independent noise is plotted in Figure \ref{fig:thres_ufbb_toric_2d_seq}. We see that the sequential fits follow a trend where the increase in the input lattice sizes results in a decrease in $p_{th}$ but increase in $k_{th}$. The range of thresholds coordinates $(p_{th}, k_{th})$ is plotted in Figure \ref{fig:thres_ufbb_toric_2d_comp}, together with the data acquired from the simulations for the performance of the MWPM decoder and the DBUF decoder. Similar figures for the Monte Carlo simulations on the planar code and the phenomenological noise model are included in Figures \ref{fig:thres_ufbb_planar_2d}, \ref{fig:thres_ufbb_toric_3d}, and \ref{fig:thres_ufbb_planar_3d}. 

We ascribe the degradation of the threshold error rate to the \emph{parity inversion} effect of Definition \ref{def:parityinversion}. Recall from Lemma \ref{lem:eqstate} that the number of iterations waited before a union $I_t$ is proportional to the number of iterations required to reach the balanced-bloom state (Definition \ref{def:balancedbloom}) $M_{t+1}$, where $(I:M)$ is the equilibrium state of Definition \ref{def:eqstate}. By setting the equilibrium factor to $k_{eq}=\frac{1}{2}$, the equilibrium state is occupied half on average. The degradation is caused by the proportionality of the number of parity inversions and consequently equilibrium-state parameter $M$ to the lattice size. As the lattice size increases, the equilibrium state is still occupied half on average, but the absolute difference in $M-I$ increases. It is thus increasingly more unlikely that the balanced-bloom state is reached. 

Overall, for small lattice sizes, the Union-Find Balanced-Bloom decoder has an increased error threshold $p_{th}$ from the threshold of the Union-Find decoder and is comparable to the threshold values of the Minimum-Weight Perfect Matching decoder. The error threshold decreases for larger lattice sizes, but the Union-Find Balanced-Bloom decoder still has an increased performance, which is now apparent by an increased decoding success rate at the threshold $k_{th}$. The improvement across all lattice sizes is most apparent when comparing the range of threshold coordinates in $(p_X, k_C)$ space, where the coordinates now occupy a range that is not possible with the Union-Find decoder. 

\Figure[htb](topskip=0pt, botskip=0pt, midskip=0pt){tikzfigs/threshold_ufbb.pdf}{bla\label{threshold_ufbb}}


\subsection{Matching weight and running time}

Finally, we plot the average matching weight and running time of the Union-Find Balanced-Bloom (UFBB) compared with the Dynamic-forest Bucket Union-Find decoder (DBUF) and the Minimum-Weight Perfect Matching (MWPM) decoder for data acquired on simulations on a toric code with independent noise and $p_X = 0.1$ in Figure \ref{fig:ufbb_tmwcomp_toric_2d}. We can see from this figure that the Union-Find Balanced-Bloom decoder has a constant decreased weight. As for the running time, the UFBB decoder offers a midway choice between the MWPM decoder and DBUF decoder. We refer to Figures \ref{fig:mwcomp_ufbb} and \ref{fig:tcomp_ufbb} for the same plots but on a planar code or phenomenological noise, for which we observe the same behavior. 

We find that the decrease in weight is constant across the range of values of $p_X$, which is also the case for the planar code and the phenomenological noise model. We can compare the decrease in matching weight as the ratios between the normalized matching weight between the UFBB and DBUF decoders as
\begin{equation}
  r_{\abs{\m{C}}}=\frac{\abs{\m{C}_{UFBB}}/\abs{\m{C}_{MWPM}}}{\abs{\m{C}_{DBUF}}/\abs{\m{C}_{MWPM}}}. 
\end{equation}
We find the averaged values for $r_{\abs{\m{C}}}$ from the Monte Carlo simulations on the toric and planar lattices, and with the independent and phenomenological noise models, in Table \ref{tab:nmwratio}. The Union-Find Balanced-Bloom decoder successfully decreases the matching weight from the Dynamic-forest Bucket Union-Find decoder. The decrease is more apparent under the independent noise model. 

\Figure[htb](topskip=0pt, botskip=0pt, midskip=0pt){tikzfigs/comp_ufbb_toric_2d_p98.pdf}{bla\label{tmw_comp}}
% \Figure[htb](topskip=0pt, botskip=0pt, midskip=0pt){tikzfigs/threshold_comparison_dense.pdf}{bla\label{thres_comp_d}}
% this can be plotted with a shared y-axis
\section{Conclusion}\label{sec:conclusion}

In this paper, we have introduced a modification of the Union-Find (UF) decoder \cite{delfosse2017almost} that selectively grows regions of clusters based on the concept of a potential matching weight. The modified decoder, dubbed the Union-Find Node-Suspension (UFNS) decoder, relies on an additional data structure to facilitate the calculation of the potential matching weight. We have proved analytically that the UFNS decoder has a worst-case time complexity of $\m{O}(n\log{n})$. 

Through Monte Carlo simulations on various decoder types, we have found that the UFNS decoder improves upon the performance of the UF decoder for all tested physical error rates and system sizes. Unfortunately, there is no fixed error threshold due to the Parity Inversion effect, which affects the performance at larger lattice sizes. Nevertheless, the UFNS decoder manages to occupy a region in $(p_X, d)$ space previously reserved to the Minimum Weight Perfect Matching (MWPM) decoder. For the low-error regime, the UFNS combines the advantages of the MWPM decoder's high decoding rates and the UF decoder's low computation time. Future work should focus on finding a way around the Parity Inversion effect and testing the decoder for other error types, such as erasure errors. 

Recent work that includes the Union-Find decoder focuses on bringing the decoder algorithm to the hardware level. Most notably, a scalable decoder micro-architecture has been proposed with a fully pipelined hardware implementation \cite{das2020scalable}. Related work has shown that a reduction in bandwidth is possible provided qubits with a low physical error rate \cite{delfosse2020hierarchical}. Furthermore, another variant of the decoder, dubbed the \emph{Weighted Union-Find} decoder, not to be confused with \emph{Weighted Growth}, promises to increase the code threshold under circuit-level noise \cite{huang2020fault}. This application relies on adopting the decoder to a \emph{weighted} graph. Every edge $e\in\m{E}$ may now have a different length value, and edges are not limited to the growth of half-edges per growth iteration. We believe that the Union-Find Node-Suspension decoder and the Weighted Union-Find decoder are compatible. In the combined decoder, boundary edges in every node are grown with respect to their weights in the weighted graph. 

The Union-Find decoder manages to decode fast and scale almost-linearly with the input system size. However, these speed-ups come at the cost of a decreased decoding performance. With the Union-Find Node-Suspension decoder, we manage to find a middle ground between the two objectives; high decoding performance that runs in worst-case quasilinear time. For these reasons, it may be a great candidate for physical applications in the near future.

\Figure[b!](topskip=0pt, botskip=0pt, midskip=0pt){tikzfigs/comp_lowerror.pdf}{
  The decoding rate $d$ for the low-error regime of phenomenological noise for the MWPM, UFNS and bvUF decoders. The UFNS decoding rates are improved from the UF variants and are very similar to MWPM. All $d$ are obtained by Monte Carlo simulations with a minimum of $100.000$ samples. The x-axis scales linearly with $N = L^3$.\label{comp_lowerror}}

\Figure[b!](topskip=0pt, botskip=0pt, midskip=0pt){tikzfigs/comp_lowerror_time.pdf}{
  The mean computation time of the UFNS, bvUF, and MWPM decoders in the low error regime for phenomenological noise for $p_X = \{0.5\%, 1.2\%, 2\%\}$ of the same simulation as in \Cref{comp_lowerror}. In this regime, the UFNS computation times are very comparable to the bvUF decoder. The x-axis scales linearly with $N = L^3$. \label{comp_lowerror_time}}


\Figure[htb](topskip=0pt, botskip=0pt, midskip=0pt){tikzfigs/threshold_comparison.pdf}{
  Direct comparison of the performance of various decoders covered in this thesis. The data of the original Union-Find (UF) decoder is taken from its publication \cite{delfosse2017almost}. Using the same range of lattice sizes and error rates, we simulate and plot the performance of \emph{(1)} our implementation of the Union-Find decoder with Weighted Growth applied via bucket sort and acyclic vertex-trees maintained during growth, the bvUF decoder, \emph{(2)} the Union-Find Node-Suspension decoder (UFNS), and  \emph{(3)} the Minimum-Weight Perfect Matching (MWPM) decoder.\label{thres_comp}}

\FloatBarrier
\printbibliography
\EOD
\end{document}